We first go over the solution to the second part of the first problem in the homework set - classifying the path-connected covers of $S^3 / Q_8$ based on the different subgroups of $Q_8$, which is the fundamental group of $S^3/Q_8$.
\par Theorem: Let $(X_\alpha, x_\alpha)$, $\alpha \in J$ be a pointed topological group, suvh that there is a contractible neighborhood of $x_\alpha$ for eavh $\alpha$.. There is an isomorphism 
\[*_{\alpha \in J}\pi_i(X_\alpha, x_\alpha) \to \pi_1 (\bigvee_{\alpha \in J} (X_\alpha, x_\alpha), \overline {x_\alpha})\]
\par Corollary: $\pi_1\left( \bigvee_{i=1}^n S_i^1, (1,0) \right) \simeq F_n$, the free group on $n$ letters. 
\par Midterm review and description available on Monday. Carl says we won't need too much geometric intuition. 
\par Going over problem 1 on the problem set, Carl first gives a geometric description of the solution, and then we see that Van Kampen's theorem can also be applied to give the same answer.
\par Now, Carl reviews Van Kampen's theorem, from class on Wednesday, which I missed because I was pretending to be sick. 
\par I've felt just awful recently. Surprisingly enough, buying an exorbitantly expensive tiny keyboard didn't seem to help all that much. I spend a lot of time lying around and worrying that things are going badly, and then most of the rest of my time playing video games and sleeping. Recently I've been trying to learn to type quickly on a 40\% ortholinear keyboard, because if I don't learn to type well on this thing I'll have paid a bunch for a useless piece of tech, which of course I am
loath to do.
\par 
\par The reason that the fundamental group of the wedge sum of spaces is the free product of the fundamental groups of those spaces is essentially just because of Van Kampen's theorem. Assuming that all spaces involved are locally contractible, there exists a contractible neighborhood of the basepoint of the wedge sum of these spaces, which has trivial fundamental group; combined with Van Kampen's theorem, this gives the fundamental group of the resulting space as a simple free
product of the individual spaces.
\par Someday my prints will come.
\par \

