% --------------------------------------------------------------
% Andrew Tindall
% --------------------------------------------------------------
 
\documentclass[12pt]{article}
 
\usepackage[margin=1in]{geometry} 
\usepackage{amsmath,amsthm,amssymb,enumitem,hyperref,tikz-cd}

\newcommand{\N}{\mathbb{N}}
\newcommand{\Q}{\mathbb{Q}}
\newcommand{\Z}{\mathbb{Z}}
\newcommand{\C}{\mathbb{C}}
\newcommand{\R}{\mathbb{R}}
\newcommand{\mc}[1]{\mathcal{#1}}
\newcommand{\e}{\varepsilon}
\newcommand{\bs}{\backslash}
\newcommand{\PGL}{\text{PGL}}
\newcommand{\Sp}{\text{Sp}}
\newcommand{\tr}{\text{tr}}
\newcommand{\Lie}{\text{Lie}}
\newcommand{\rec}[1]{\frac{1}{#1}}
\newcommand{\toinf}{\rightarrow \infty}


\theoremstyle{definition}
\newtheorem{proofpart}{Part}
\newtheorem{theorem}{Theorem}
\makeatletter
\@addtoreset{proofpart}{theorem}
\makeatother


\newenvironment{problem}[2][Problem]{\begin{trivlist}
\item[\hskip \labelsep {\bfseries #1}\hskip \labelsep {\bfseries #2.}]}{\end{trivlist}}
 
\begin{document}
 
%\renewcommand{\qedsymbol}{\filledbox}
 
\title{Homework 10}
\author{Andrew Tindall \\
Topology 2}
\begin{problem}{1}
	Hatcher, 2.2.9: Compute the homology groups of the following $2$-complexes:
	\begin{enumerate}[label=(\alph*)]
		\item The quotient of $S^2$ obtained by identifying north and south poles to a point.
			\begin{proof}
				We can form this $2$-complex out of two $2$-simplices, three $1$-simplices, and two $0$-simplices, as follows: letting $A$ and $B$ be the $2$ simplices, $a,b,c$ the $1$-simplices, and $x$ and $y$ the $0$-simplices, we have
				\begin{align*}\delta A = \delta B = a + b  -c\\
				\delta a = y - x, \delta b = x - y, \delta c = 0 \end{align*}
				This is just the $\Delta$-structure for $S^2$, with two of the $0$-simplices identified. The chain complex corresponding to this space is
				\[\cdots \to 0 \to \Z^2 \to^{\delta_2} \Z^3 \to^{\delta_1} \Z^2 \to 0 \]
				So, there can be at most three nontrivial homology groups, $H_0$, $H_1$, and $H_2$. Calculating $\delta_1$, we see that it is the linear map corresponding to the matrix $\begin{bmatrix}
					1 & -1 & 0\\
					-1 & 1 & 0
				\end{bmatrix}$, whose image is a one-dimensional submodule of $\Z_2$ generated by $\langle 1, -1\rangle$, and whose kernel is a two-dimensional submodule of $\Z^3$ which can be generated by the elements $\langle 1, 1, 0\rangle$ and $\langle 0,0,1\rangle$. Looking at $\delta_2$, we see that it is the linear map corresponding to the matrix $\begin{bmatrix}
					1 & 1 \\ 1 & 1 \\ -1 & -1
				\end{bmatrix}$, whose kernel is generated by $\langle 1, -1\rangle$ and whose image is generated by $\langle 1, 1, -1\rangle$. 
				\par Since the kernel of $\delta_1$ is generated by $(a + b)$ and $c$, and the image of $\delta_2$ is generated by $a + b - c$, we see that $\text{ker}(\delta_1) / \text{im}(\delta_2)$ is generated by the images of $(a + b)$ and $c$ modulo $\text{im}(\delta_2)$, and both $(a + b)$ and $c$ are equal modulo $\text{im}(\delta_2)$, so the homology group $H_1$ is cyclic with one generator: $H_1 \simeq \Z$.
				\par As $\delta_3$ is trivial, the group $H_2$ is simply equal to $\text{ker}(\delta_2)$, which we have seen is cyclic with one generator, $A - B$. So both $H_1$ and $H_2$ are equal to $\Z$.
			\end{proof} 
		\item $S^1 \times (S^1 \vee S^1)$.
			\begin{proof}
				The K\"unneth formula tells us that, for the product of arbitrary spaces, there is a natural short exact sequence
				\[0 \to \bigoplus_{i + j = k} H_i(X) \otimes H_j(Y) \to H_k(X \times Y) \to \bigoplus_{i + j = k - 1} \text{Tor}_1(H_i(X), H_j(Y)) \to 0.\]
				The nontrivial homology groups of $S^1$ are
				\begin{align*}
					H_1(S^1) \simeq \Z\\
					H_0(S^1) \simeq \Z,
				\end{align*}
				While the homology groups of $S_1 \vee S_1$ are 
				\begin{align*}
					H_1(S^1 \vee S^1) \simeq \Z \oplus \Z\\
					H_0(S^1 \vee S^1) \simeq \Z
				\end{align*}
				Because all of these groups are free, hence flat, the $\text{Tor}_1$ terms in the above exact sequence are trivial. Therefore, we can calculate the homology groups of $S^1 \times (S^1 \vee S^1)$ directly:
				\begin{align*}
					H_2(S^1 \times (S^1 \vee S^1)) &= H_1(S^1) \otimes H_1(S^1 \vee S^1) = \Z \otimes (\Z \oplus \Z) = \Z,\\
					H_1(S^1 \times (S^1 \vee S^1)) &= (H_1(S^1) \otimes H_0(S^1 \vee S^1)) \oplus (H_0(S^1) \otimes H_1(S^1 \vee S^1))\\
					&= (\Z \otimes \Z ) \oplus (\Z \otimes (\Z \oplus \Z)) &= \Z \oplus \Z \oplus \Z,\\
					H_0(S^1 \times (S^1 \vee S^1)) &= H_1(S^1) \otimes H_0(S^1 \vee S^1) = \Z \otimes \Z = \Z
				\end{align*}
			\end{proof}
	\end{enumerate}
\end{problem}
\begin{problem}{2}
	Hatcher 2.2.14: A map $f: S^n \to S^n$ satisfying $f(x) = f(-x)$ is called an \textit{even map}. Show that an even map $S^n \to S^n$ must have even degree, and that the degree must in fact be zero when $n$ is even. When $n$ is odd, show that there exist even maps of any given degree.
	\begin{proof}
		\textit{incomplete}
	\end{proof}
\end{problem}
\maketitle
\end{document}
