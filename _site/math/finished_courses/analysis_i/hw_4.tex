% --------------------------------------------------------------
% Andrew Tindall
% --------------------------------------------------------------
 
\documentclass[12pt]{article}
 
\usepackage[margin=1in]{geometry} 
\usepackage{amsmath,amsthm,amssymb,enumitem}

\newcommand{\N}{\mathbb{N}}
\newcommand{\Q}{\mathbb{Q}}
\newcommand{\Z}{\mathbb{Z}}
\newcommand{\R}{\mathbb{R}}
\newcommand{\mc}[1]{\mathcal{#1}}
\newcommand{\e}{\varepsilon}
\newcommand{\bs}{\backslash}
\newcommand{\PGL}{\text{PGL}}
\newcommand{\Sp}{\text{Sp}}
\newcommand{\tr}{\text{tr}}
\newcommand{\Lie}{\text{Lie}}
\newcommand{\rec}[1]{\frac{1}{#1}}
\newcommand{\toinf}{\rightarrow \infty}


\theoremstyle{definition}
\newtheorem{proofpart}{Part}
\newtheorem{theorem}{Theorem}
\makeatletter
\@addtoreset{proofpart}{theorem}
\makeatother


\newenvironment{problem}[2][Problem]{\begin{trivlist}
\item[\hskip \labelsep {\bfseries #1}\hskip \labelsep {\bfseries #2.}]}{\end{trivlist}}
 
\begin{document}
 
%\renewcommand{\qedsymbol}{\filledbox}
 
\title{Homework 3}
\author{Andrew Tindall\\
Analysis I}
 
\maketitle
\begin{problem}{1}
	Let $f: \R^n \to \R^m$ be differentiable at $x_0 \in \R^n$.
	\begin{enumerate}[label=(\roman*)]
		\item Prove that $f'(x_0)$ is the linear, automatically continuous, operator from $\R^n$ to $\R^m$ that is given by the matrix $A \in \R^{n\times m}$ whose components are partial derivatives: $A_{ij} = \frac{\partial f^i}{\partial x_j}(x_0)$. This is the Jacobi matrix.
		\item Prove that if all partial derivatives of $f$ exist and are continuous, then $f$ is strongly differentiable at each $x_0 \in \R^n$.
	\end{enumerate}
\end{problem}
\begin{proof}
	\begin{enumerate}[label=(\roman*)]
		\item \textit{incomplete}
		\item \textit{incomplete}
	\end{enumerate}
\end{proof}
\begin{problem}
Let $E$ be a Banach space and let $F$ be a finite-dimensional subspace of $E$. Show that there exists a closed subspace $G \subset E$ such that:
\[
F + G = E \;\text{and}\; F \cap G= \{0\}.
\]
\end{problem}
\begin{proof}
This proof is adapted from a Stackexchange answer given at \cite{se} Assuming that $F$ is $n$-dimensional, let $e_1, ... e_n$ be a basis of $F$. Define the linear functionals $f_i$ on the basis $e_i$ by $f_i(e_i) = 1$, and $f_i(e_j) = 0$ if $i \neq j$. Then by Hahn-Banach, each of these functionals may be extended to functionals $g_i$ defined on the whole of $E$. Let the space $G$ be defined as
\[ G = \bigcap_{i = 1}^n \text{ker}(g_i).\]
\par We now show that this space is closed, that $F+G=E$, and that $F \cap G = \left\{ 0 \right\}$. First, $G$ is closed because the kernel of any functional is closed, and therefore so is their intersection.
\par Now, let $x \in E$ be arbitrary; we decompose it into elements of $F$ and $G$. Let $y = g_1(x)e_1 + \dots + g_n(x)e_n$, and $z = x - (g_1(x)e_1 + \dots + g_n(x)e_n)$. It is clear that $y \in F$, and we see that $z \in G$ because, for any $g_i$, 
\begin{align*}g_i(z) & = g_i(x - (g_1(x))e_1 + \dots + g_n(x)e_n)\\
&= g_i(x) - g_i(x)g_i(e_i)
&= 0\end{align*}
Therefore, $g$ is in the intersection of the kernels of all $g_i$, meaning it is in $G$, and so any arbitrary element of $E$ may be decomposed as $y+z$, with $y \in F$ and $z \in G$. So $E = F + G$.
\par Now, we show that $F \cap G = \left\{ 0 \right\}$. Let $x$ be an arbitary element of $F \cap G$. Because $x \in F$, we may write it as $x = x_1 e_1 + \dots + x_n e_n$; therefore for any $g_i$,
\[g_i(x) = x_ng_i(e_i) = x_i.\] However, $x \in \text{ker}(g_i)$, so each component $x_i$ is  $0$.
\end{proof}
\begin{problem}{3}
	\begin{enumerate}[label=(\roman*)]
		\item Let $(E, \left \lVert { \cdot } \right \lVert )$ be the Banach space defined in problem $5$ of Homework 3 (the space of Lipchitz functions $f: X \to \R$, where $f(x_0) = 0$, and $\left \lVert { f } \right \lVert $ is the lowest Lipschitz constant of $f$). For each $x \in X$, let $T_x(f) = f(x)$. Prove that each $T_x \in E^*$, and that $\left \lVert { T_x - T_y } \right \lVert = d(x,y)$. Deduce that $X$ is therefore isometric to a subset of $E^*$.
		\item Let $E$ be a linear normed space. Prove that it is isometric with a linear subspace of the Banach space $\mathcal B(B_{E^*}(1))$ of bounded real functions on the closed unit ball in the dual space $E^{*}$: $B_{E^*}(1) = \left\{ T \in E^{*}; \left \lVert { T } \right \lVert \leq 1 \right\}$.
	\end{enumerate}
\end{problem}
\begin{proof}
	\begin{enumerate}[label=(\roman*)]
		\item Let $x$ in $X$ be arbitrary. We show first that $T_x \in E^{*}$; i.e. that it is a bounded real functional on $E$. 
			\begin{itemize}
				\item $T_x(f + g) = T_x(f) + T_x(g)$: This follows from direct calculation, as 
					\begin{align*}
						T_x(f + g) &= (f + g)(x) \\
						&= f(x) + g(x)\\
						&= T_x(f) + T_x(g)
					\end{align*}
				\item $T_x(\alpha f) = \alpha T_x(f)$: This too, follows from calculation:
					\begin{align*}
						T_x(\alpha f) &= (\alpha f)(x)\\
						&= \alpha (f(x))\\
						&= \alpha T_x(f)
					\end{align*}
				\item $\left \lVert { T_x } \right \lVert  < \infty$: We show that the norm of $T_x$ is bounded by $d(x,x_0)$. For if $f$ is an arbitrary element of $E$ with lowest Lipschitz constant $1$, then in particular 
					\[\frac{\lvert f(x) - f(x_0) \rvert}{d(x,x_0)} \leq 1,\]
					And so $\lvert T_x(f ) \rvert \leq d(x,x_0)$. 
			\end{itemize}
			Therefore $T_x$ is an element of the continuous dual. Now, let $x, y\in E$ be arbitrary. We may bound the value of $\lVert T_x - T_y \rVert$ by $d(x,y)$: if $f \in E$ is an arbitrary Lipschitz function with $f(x_0) = 0$ and Lipschitz constant $1$, then $\frac{\left \lvert { f(x) - f(y) } \right \lvert }{d(x,y)} \leq 1$, so
			\begin{align*}
				\lvert (T_x - T_y)(f) \rvert &= \lvert (f(x) - f(y)) \rvert\\
				&\leq d(x,y)
			\end{align*}
			Now, we can also show that this bound is attained: let $f$ be the function:
			\[z \mapsto \frac{d(z,y)d(x_0,x)}{d(x_0,x)+d(x_0,y)} - \frac{d(z,x)d(x_0,y)}{d(x_0,x)+d(x_0,y)}.\]
			Then $f$ is Lipschitz, because the distance function has Lipschitz constant $1$, and it is an element of $E$, because $f(x_0) = 0$:
			\[f(x_0) = \frac{d(x_0,y)d(x_0,x)}{d(x_0,x)+d(x_0,y)} - \frac{d(x_0,x)d(x_0,y)}{d(x_0,x)+d(x_0,y)} = 0\]
			Also, $\left \lVert { f } \right \lVert \leq1$, because for any two $w,z \in X$,
			\begin{align*}
				\left \lvert { f(z)-f(w) } \right \lvert &= \left \lvert { \frac{d(z,y)d(x_0,x) - d(w,y)d(x_0,x)}{d(x_0,x)+d(x_0,y)} - \frac{d(z,x)d(x_0,y) - d(w,x)d(x_0, y)}{d(x_0,x)+d(x_0,y)} } \right \lvert \\
				&\leq \left \lvert { \frac{((d(z,x) - d(w,x))d(x_0,y)}{d(x_0,x)+d(x_0,y)}  } \right \lvert  + \left \lvert { \frac{(d(z,y) - d(w,y))d(x_0,x)}{d(x_0,x)+d(x_0,y)} } \right \lvert \\
					&\leq \frac{d(z,w)d(x_0,y)}{d(x_0, x)+d(x_0,y)} + \frac{d(z,w)d(x_0,x)}{d(x_0,x) + d(x_0, y)}\\
					&= d(w,z)
			\end{align*}
			In fact, $\left \lVert { f } \right \lVert  = 1$, because $\left \lvert { f(x) - f(y) } \right \lvert  = d(x,y)$. So, $f$ is an element of the unit ball in $E$.
			\par Now, we can show that $\left \lvert { T_x(f) - T_y(f) } \right \lvert = d(x,y)$. In fact, this follows from the earlier observation that $\left \lvert { f(x) - f(y) } \right \lvert  = d(x,y)$, which we can see by calculation:
			\begin{align*}
				\left \lvert { f(x) - f(y) } \right \lvert &= \left \lvert { \frac{(d(x,y) - d(x,x))d(x_0,x)}{d(x_0, x) + d(x_0, y)} + \frac{(d(x,y) - d(y,y)d(x_0,x))}{d(x_0,x)+d(x_0,y)} } \right \lvert \\
				&= d(x,y)
			\end{align*}
			Therefore, $\left \lVert { T_x - T_y } \right \lVert = d(x,y)$. Therefore the subspace of elements $\left\{ T_x ; x \in X \right\}$ is isometric to the space $X$ itself.
		\item \textit{incomplete}
	\end{enumerate}
\end{proof}
\begin{problem}{4}
	Let $(Y,d)$ be a complete metric space and let $f : B \to Y$ be a contractive mapping with Lipschitz constant $\alpha < 1$, where $B$ is an open ball centered at some $y_0$ with radius $r > 0$. Prove that if $d(f(y_0), y_0) < (1-\alpha)r$ then $f$ has a fixed point.
\end{problem}
\begin{proof}
	Given the above assumptions, let $k = d(f(y_0), y_0)/r$, which we assume to be less than $(1-\alpha)$. We now show that $f$ restricts to a contraction mapping on the closed ball $B'$ with radius $\frac{kr}{1-\alpha}$: Let $x \in B'$. Then we want to show that $f(x) \in B'$, i.e. that $d(f(x),y_0) \leq \frac{kr}{1-\alpha}$. This in fact holds:
	\begin{align*}
		d(f(x), y_0) \leq d(f(x), f(y)) + d(f(y), y)\\
		&\leq \alpha d(x,y) + kr\\
		&\leq \alpha \frac{kr}{1-\alpha} + kr\\
		&= \frac{kr}{1-\alpha}
	\end{align*}
	Therefore $f$ restricts to a function $B' \to B'$, and the restriction of a contraction mapping is a contraction mapping. Because $Y$ is a complete space, the closed subspace $B'$ is complete, and by the contraction mapping theorem $f$ admits a fixed point $x_0 \in B' \subset B$.
\end{proof}
\begin{problem}
	Let $(X,d)$ be a complete metric space and $f : X \to X$ a map such that for some $n > 1$ the composition of the function $f$ with itself $n$ times: $f^{(n)}: X \to X$ is a contraction.
	\begin{enumerate}[label=(\roman*)]
		\item Does $f$ have to be continuous?
		\item Prove that $f$ has a unique fixed point in $X$.
	\end{enumerate}
\end{problem}
\begin{proof}
	\begin{enumerate}[label=(\roman*)]
		\item No, $f$ does not have to be continuous - for example, let $f$ be the function on the unit ball in $\R^2$:
			\[f(x,y) = \begin{cases}
					(\frac{x}{2}, \frac{y}{2}) & y \neq 0\\
					(-\frac{x}{2}, 0) & y = 0
		\end{cases}\]
		Then $f$ is discontinuous on the line $y = 0$, but $f^{(2)}$ is the continuous contraction mapping $(x,y) \mapsto (\frac{x}{4}, \frac{y}{4})$. 
	\item \textit{incomplete}
	\end{enumerate}
\end{proof}
\begin{thebibliography}{}
	\bibitem{se}{Tsemo Aristide, Complement a finite dimensional subspace in a Banach space, URL: https://math.stackexchange.com/q/3224629}
\end{thebibliography}
\end{document}
