% --------------------------------------------------------------
% Andrew Tindall
% --------------------------------------------------------------
 
\documentclass[12pt]{article}
 
\usepackage[margin=1in]{geometry} 
\usepackage{amsmath,amsthm,amssymb,enumitem}

\newcommand{\N}{\mathbb{N}}
\newcommand{\Q}{\mathbb{Q}}
\newcommand{\Z}{\mathbb{Z}}
\newcommand{\R}{\mathbb{R}}
\newcommand{\mc}[1]{\mathcal{#1}}
\newcommand{\e}{\varepsilon}
\newcommand{\bs}{\backslash}
\newcommand{\PGL}{\text{PGL}}
\newcommand{\Sp}{\text{Sp}}
\newcommand{\tr}{\text{tr}}
\newcommand{\Lie}{\text{Lie}}
\newcommand{\rec}[1]{\frac{1}{#1}}
\newcommand{\toinf}{\rightarrow \infty}


\theoremstyle{definition}
\newtheorem{proofpart}{Part}
\newtheorem{theorem}{Theorem}
\makeatletter
\@addtoreset{proofpart}{theorem}
\makeatother


\newenvironment{problem}[2][Problem]{\begin{trivlist}
\item[\hskip \labelsep {\bfseries #1}\hskip \labelsep {\bfseries #2.}]}{\end{trivlist}}
 
\begin{document}
 
%\renewcommand{\qedsymbol}{\filledbox}
 
\title{ Homework 9}
\author{Andrew Tindall\\
Analysis I}
 
\maketitle
\begin{problem}{1}
	Given an open box $A = (a_1, b_1)\times \dots \times (a_n, b_n) \subset \R^n$ of finite, positive volume, construct a sequence $\varphi_n$ of functions in $\mathcal{C}^\infty_c(\R^n, \R)$ such that $\varphi_n \nearrow \mathbf{1}_A$	.
	\begin{proof}
		We use the following step function:
		\[\sigma(x) = \begin{cases}
				\exp\left( -\frac{1}{1 - x^2} \right), & -1 < x < 0\\
				0, & x \leq -1\\
				1, & x \geq 0
	\end{cases}\]
	It is smooth, and it is $0$ for $x \leq -1$ and $1$ for $x \geq 0$. Then, using the two linear transformations
	\begin{align*}
		x \mapsto \frac{x-a}{\varepsilon} - 1, \text{ and}\\
		y \mapsto -\frac{y - b+\varepsilon}{\varepsilon},
	\end{align*}which map $[a, a +\varepsilon]$ and $[b, b- \varepsilon]$ to $[-1, 0]$, respectively, we get the two step functions
		\begin{align*}f_{\varepsilon,a}(x) &= \begin{cases}
				\exp\left( -\frac{1}{1 - ((x-a)/\varepsilon - 1)^2} \right) &, x \in (a, a + \varepsilon)\\
				0, & x \leq a\\
				1, & x \geq a + \varepsilon
		\end{cases}\\
		g_{\varepsilon,b}(x) &=
		\begin{cases}
			\exp\left( -\frac{1}{1-(-(y-b+\varepsilon)/\varepsilon)^2} \right), & b-\varepsilon < x < b\\
			0, & x \geq b\\
			1, & x \leq b-\varepsilon
	\end{cases}\end{align*}
	Then, using these two step functions, and assuming that $a+ \varepsilon < b -\varepsilon$, we have the following bump function:
	\[h_{a,b,\varepsilon}(x) = \begin{cases}
			f_{\varepsilon,a}(x), & x \leq a + \varepsilon\\
			1, & a + \varepsilon < x < b - \varepsilon\\
			g_{\varepsilon, b}(x), & x \geq b - \varepsilon
\end{cases}\]
This is a smooth function with compact support that is $0$ outside of $(a,b)$, and is $1$ inside of $(a + \varepsilon, b - \varepsilon)$.
\par Now, let $m$ be the minimum value of $b_i - a_i$ for $1 \leq i \leq n$, and let $\varepsilon < m/2$. Define the function $H_\varepsilon: \R^n \to \R$ by 
\[H_\varepsilon(x_1, \dots x_n) = (h_{a_1, b_1, \varepsilon}(x_1), \dots h_{a_n, b_n, \varepsilon}(x_n)).\]
Because each component function is smooth, and has compact support, $H$ is $\mathcal{C}^\infty_c$, and is $0$ outside of $A$ and $1$ inside the box $(a_1 + \varepsilon, b_1 - \varepsilon) \times \dots \times (a_n + \epsilon, b_n - \epsilon)$. 
\par Finally, let $\varphi_n = H_{m/2^{n+1}}$. This is a monotone increasing sequence of functions which, for any point in the interior of $A$, is eventually $1$, and is always $0$ for any point in the complement of $A$. Therefore, $\varphi_n \nearrow \mathbf 1_A$.
	\end{proof}
\end{problem}
\begin{problem}{2}
	Prove that if $A \in \mathcal{L}_{n+m}$ has measure $0$, then for almost every $x \in \R^{n}$ the set:
	\[ A_{y}:= \left\{ x \in \R^n; (x, y) \in A \right\}\] 
	is in $\mathcal{L}_n$ and has measure $0$. (I did this problem in $x$ instead of $y$ before noticing my mistake; the argument is symmetric)
	\begin{proof}
		We first prove that the set $A_x$ is in $\mathcal{L}_m$, for all $x$. The following proof follows Rudin in \cite{rudin}:
		\par Let $\Omega$ be the class of all $A \in \mathcal{L}_{n+m}$ such that $A_x \in \mathcal{L}_m$for every $x \in \R^n$. If $A = (a,b) \times (c,d)$ then $A_x = (c,d)$ if $x \ in (a,b)$, and $\emptyset$ otherwise. Therefore, every measurable rectangle $(a,b) \times (c,d)$ belongs to $\Omega$. The following statements show that $\Omega$ is a $\sigma$-algebra, and therefore that it is equal to $\mathcal{L}_{n+m}$:
		\begin{enumerate}[label=(\alph*)]
			\item $\R^{n+m} \in \Omega$
			\item If $A \in \Omega$, then $(A^c)_x = (A_x)^c$, so $(A^c)_x$ is measurable and $A^c \in \Omega$
			\item If $A_i \in \Omega$ for $i \in (1,2,3,\dots)$ and $A = \bigcup A_i$, then $A_x = \bigcup (A_i)_x$, so it $A_x$ is measurable, and $A \in \Omega$
		\end{enumerate}
		So, $\Omega = \mathcal{L}_{n+m}$, and every set $A$ in $\mathcal{L}_{n+m}$ is such that $A_x$ is measurable for all $x$.
		\par Now, assume that $A$ has measure $0$; we show that $A_x$ has measure $0$ for almost every $x \in \R^n$. Let $X_\varepsilon$ be the set of all $x \in \R^n$ such that $m(A_x) \geq \varepsilon$ for all $x \in X_\varepsilon$; by the above argument, $\left( \bigcup_{\varepsilon > 0} X_\varepsilon \right)^c$ is the set of all $x \in \R^n$ such that $A_x$ has measure $0$.
		\par Fix some $\varepsilon > 0$; we show that $X_\varepsilon$ has measure $0$. To do this, fix $\delta > 0$; we show that $m(X_\varepsilon) \leq \delta$. 
		\par Let $A_\varepsilon$ be the set $\left\{ (x,y) \in A; x \in X_\varepsilon \right\}$. Since $A$ has measure $0$, we can cover it with a union of boxes with total size $\leq \delta\varepsilon$. By definition of $A_\varepsilon$, the width of each box (its measure when projected onto its $y$-component) must be $\geq \varepsilon$; therefore the sum of the heights of the boxes must be $\leq \delta$. Because $\delta$ was arbitrary, the projection of $A_\varepsilon$ onto its $x$-components must be measure $0$, meaning that $X_\varepsilon$ has measure $0$.
		\par Let $X$ be the set of all $x$ such that $A_x$ has measure $> 0$. Because $X = \bigcup X_\varepsilon$, a countable union of measure $0$ sets, it must also have measure $0$, which was to be shown.
	\end{proof}
\end{problem}
\begin{problem}{3}
	\textit{incomplete}	
\end{problem}
\begin{problem}{4}
	Let $f_n: [0, \pi] \to \R$ be given by:	
	\[ f_n(x) = n \frac{\sin x}{1 + n^2 \sin^2 x}.\]
	For a given $\varepsilon > 0$, find explicitly the Egoroff set $E_\varepsilon$ on which the sequence $f_n$ converges uniformly, and such that $\mu(E_\varepsilon) > \pi - \varepsilon$.
	\begin{proof}
		We show that $[\varepsilon/3, \pi - \varepsilon/3]$ satisfies the requirements. The measure of this set is $\pi - 2\varepsilon/3$, so it satisfies the measure requirement; we can also show that $f_n \to 0$ uniformly on $E_\varepsilon$ as $n \to \infty$.
		\par Each $f_n$ is the composition of the rational function $g_n : [0,1] \to \R$ given by
		\[g_n(x) = n\frac{x}{1 + n^2x^2}\]
		with the function $\sin: [0,\pi] \to [0,1]$. Because $\sin$ is smooth and monotone on $[0,\pi/2]$ and on $[\pi/2, \pi]$, with bounded first derivative, it suffices to show that $g_n$ converges smoothly to $0$ on $\sin([\varepsilon/3, \pi - \varepsilon/3])$, which is $[\sin(\varepsilon/3), 1]$. 
		\par We wish to show that, for $\delta > 0$, there exists some $N \in \N$ such that, for all $n \geq N$, $\lvert g_n(x)\rvert < \delta$ for all $x \in [\sin(\varepsilon/3), 1]$. 
		\par First, we note that for large enough $n$, $g_n$ is monotone decreasing on $[\sin(\varepsilon/3), 1]$. In particular, this holds for all $n > \sin^{-1}(1/\varepsilon)$, as $g'_n$ has only one $0$ in $[0,1]$, which is at $1/n$. For $n > \sin^{-1}(3/\varepsilon)$, this maximum falls outside of the range $[\sin(\varepsilon/3), 1]$, so $g_n$ is positive and monotone decreasing on this interval, and we need only check that $g_n(\sin(\varepsilon/3)) < \delta$ for large enough $n$:
		\begin{align*}
			 n\frac{\sin(\varepsilon/3)}{1 + n^2\sin^2 (\varepsilon/3)} &< \delta\\
			 n\sin(\varepsilon/3) &< \delta + \delta n^2 \sin^2(\varepsilon/3)\\
			 0 &< \delta n^2 \sin^2(\varepsilon/3) - n\sin(\varepsilon/3) + \delta
		\end{align*}
		Because this is a quadratic polynomial in the $\sin$ terms with positive first coefficient, it suffices to show that, for sufficiently large $n$, both zeroes of the polynomial 
		\[\delta n^2 x^2 - nx + \delta\]
		are less than $\sin(\varepsilon/3)$. By the quadratic formula, the zeroes are
		\begin{align*}\frac{n \pm \sqrt{n^{2} - 4\delta n^2}}{2\delta^2 n^4} &= \frac{1 \pm \sqrt{1 - 4\delta}}{2\delta^2 n^3}.
		\end{align*}
		Assuming $\delta$ is small enough, $\sqrt{1 - 4\delta}$ is close to $1$, so the zeroes are at approximately $0$ and $1/\delta^2 n^3$, with neither one greater than $1/\delta^2n^{3}$. For $n > (\delta^2 \sin(\varepsilon/3))^{-1/3}$, both zeroes are less than $[\sin(\varepsilon/3)]$, and so the function $g_n$ is bounded as desired. Therefore, $f_n$ converge uniformly to $0$ on $E_\varepsilon$.
	\end{proof}
\end{problem}
\begin{problem}{5}
	Prove that for every set $A \subset \R^n$ which is not of Lebesgue measure $0$, there holds:
	\[ \forall c \in (0,1) \quad \exists P \subset \R^n \qquad \mu^{*}(A \cap P) > c \mu^*(P).\]
	\begin{proof}
		We show the contrapositive: if $A$ is a subset of $\R^n$ such that there exists a $c$ for which, for all closed boxes $P \subset \R^n$, $\mu^*(A \cap P) \leq c \mu^*(P)$, then $A$ is of Lebesgue measure $0$.
		\par First, we note that if this property holds for all closed boxes $P$, it holds for all countable unions of these sets as well, by subadditivity of the outer measure.
		\par Now, assume $A$ has the given property, for some $c \in (0,1)$, and let $\varepsilon = 1- c$. Let $A_m$ be the set $A \cap ([-m, m] \times \dots \times [-m, m])$, which has finite outer measure. The same property, with $c\mu^{*}(P)$ bounding the measure of $\mu^{*}(P \cap A_m)$, assuming that $P \subset [-m, m] \times \dots \times [-m, m]$. Take $P = [-m, m] \times [-m, m]$; then $\mu^*(P) = (2m)^n$, which is less than $\infty$. Write $M = (2m)^n$, then we see that
		\[\mu^*(A_m \cap P) \leq (1 - \varepsilon)M \]
		By definition of the Lesbesgue outer measure, there must be some countable set of boxes whose union contains $A_m$, and such that the outer measure of their union is $(1 - \varepsilon/2)M$. Call the union of these boxes $P_1$; then
		\[\mu^*(A_m \cap P_1) \leq (1-\varepsilon)(1-\varepsilon/2)M\]
		Inductively, we can construct a sequence of measurable sets $P_i$, such that for each one,
		\[\mu^{*}(A_m \cap P_i) \leq (1 - \varepsilon)(1 - \varepsilon/2)^i M\]
		Thus $A_m$ can be covered by a measurable set of arbitrarily small measure, and it must have Lebesgue measure $0$. Because $A = \bigcup A_m$, a countable union of measure $0$ sets, it too must have Lebesgue measure $0$.

	\end{proof}
\end{problem}
\begin{thebibliography}{}
	\bibitem{rudin}{Rudin, W. Real and Complex Analysis, 3rd edition.}
\end{thebibliography}
\end{document}
