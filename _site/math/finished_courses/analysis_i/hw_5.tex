% --------------------------------------------------------------
% Andrew Tindall
% --------------------------------------------------------------
 
\documentclass[12pt]{article}
 
\usepackage[margin=1in]{geometry} 
\usepackage{amsmath,amsthm,amssymb,enumitem}

\newcommand{\N}{\mathbb{N}}
\newcommand{\Q}{\mathbb{Q}}
\newcommand{\Z}{\mathbb{Z}}
\newcommand{\R}{\mathbb{R}}
\newcommand{\mc}[1]{\mathcal{#1}}
\newcommand{\e}{\varepsilon}
\newcommand{\bs}{\backslash}
\newcommand{\PGL}{\text{PGL}}
\newcommand{\Sp}{\text{Sp}}
\newcommand{\tr}{\text{tr}}
\newcommand{\Lie}{\text{Lie}}
\newcommand{\rec}[1]{\frac{1}{#1}}
\newcommand{\toinf}{\rightarrow \infty}


\theoremstyle{definition}
\newtheorem{proofpart}{Part}
\newtheorem{theorem}{Theorem}
\makeatletter
\@addtoreset{proofpart}{theorem}
\makeatother


\newenvironment{problem}[2][Problem]{\begin{trivlist}
\item[\hskip \labelsep {\bfseries #1}\hskip \labelsep {\bfseries #2.}]}{\end{trivlist}}
 
\begin{document}
 
%\renewcommand{\qedsymbol}{\filledbox}
 
\title{Homework 5}
\author{Andrew Tindall\\
Analysis I}
 
\maketitle
\begin{problem}{1}
	Let $\Lambda$ be a metric space, $E$ a Banach space, and let $F : \Lambda \times E \to E$ be a function such that 
	\[ \exists \kappa < 1 \forall \lambda \in \Lambda \forall x,y \in E \, \left \lVert { F(\lambda, x) - F(\lambda, y) } \right \lVert \leq \kappa\left \lVert { x-y } \right \lVert. \]
	Prove that:
	\begin{enumerate}[label=(\roman*)]
		\item For every $\lambda \in \Lambda$ there exists a unique $x(\lambda) \in E$ such that $x(\lambda) = F(\lambda, x(\lambda))$,
		\item For every $\lambda \in \Lambda, y \in E$, one has:
			\[\left \lVert { x(\lambda) - F(\lambda, y) } \right \lVert \leq \frac{\kappa}{1-\kappa} \left \lVert { y - F(\lambda, y) } \right \lVert \]
			\[ \left \lVert {  y - x(\lambda) } \right \lVert  \leq \frac{1}{1-\kappa}\left \lVert { y - F(\lambda, y) } \right \lVert \]
	\end{enumerate}
\end{problem}
\begin{proof}
	\begin{enumerate}[label=(\roman*)]
		\item We show that for each $\lambda$, the function $F_\lambda: x \mapsto F(\lambda, x)$ has a unique fixed point $x(\lambda)$. Fix an arbitrary $\lambda$; then for any $x, y \in E$, 
			\[\lVert F_\lambda(x) - F_\lambda(y) \rVert \leq k \left \lVert {  x - y } \right \lVert < \frac{k+1}{2}\left \lVert { x - y } \right \lVert. \]
			Because $\frac{k+1}{2} < 1$, this makes $F_\lambda$ a contraction mapping. $E$ is Banach, so it is in particular complete, so the Brouwer fixed-point theorem gives a unique $x(\lambda)$ such that $F_\lambda(x(\lambda)) = x(\lambda)$. Because $\lambda$ was arbitrary, this holds for every $\lambda \ in \Lambda$.
		\item Using the fixed-point equality $x(\lambda) = F_\lambda(x(\lambda))$, and the existence of the almost-contraction-constant $\kappa$, we see:
			\begin{align*}
				\left \lVert { x(\lambda) - F_\lambda(y) } \right \lVert &= \left \lVert { F_\lambda(x(\lambda)) - F_\lambda(y) } \right \lVert \\
				&\leq \kappa \left \lVert { x(\lambda) - y } \right \lVert \\
				&\leq \kappa (\left \lVert { y - F_\lambda(y) } \right \lVert  + \left \lVert { x(\lambda) - F_\lambda(y) } \right \lVert )\\
				&= \kappa (\left \lVert { y - F_\lambda(y) } \right \lVert + \left \lVert { F_\lambda(x(\lambda)) - F_\lambda(y) } \right \lVert )
			\end{align*}
			Therefore,
			\[\left \lVert { F_\lambda(x(\lambda)) - F_\lambda(y) } \right \lVert \leq \kappa(\left \lVert { y - F_\lambda(y) } \right \lVert + \left \lVert { F_\lambda(x(\lambda)) - F_\lambda(y) } \right \lVert ),\]
			Or, rearranging,
			\[\left \lVert { F_\lambda(x(\lambda)) - F_\lambda(y) } \right \lVert \leq \frac{\kappa}{1 - \kappa}\left \lVert { y - F_\lambda(y) } \right \lVert, \]
			which was to be shown.
			\par Now, we show the second inequality to be true. Again using the constant $\kappa$ and the fixed-point $x(\lambda)$,
			\begin{align*}
				\left \lVert { y - x(\lambda) } \right \lVert &\leq \left \lVert { y - F_\lambda(y) } \right \lVert  + \left \lVert { F_\lambda(y) - x(\lambda) } \right \lVert \\
				&= \left \lVert { y - F_\lambda(y) } \right \lVert + \left \lVert { F_\lambda(y) - F_\lambda(x(\lambda)) } \right \lVert \\
				&\leq \left \lVert { y - F_\lambda(y) } \right \lVert + \kappa\left \lVert { y - x(\lambda) } \right \lVert 
			\end{align*}
			Which, rearranging, gives us
			\[\left \lVert { y - x(\lambda) } \right \lVert \leq \frac{1}{1+\kappa}\left \lVert { y - F_\lambda(y) } \right \lVert, \]
			which was to be shown.
	\end{enumerate}
\end{proof}
\begin{problem}{2}
	Let $f_1: E \to F_1$ and $f_2: E \to F_2$ be two differentiable mappings between Banach spaces $E,F_1, F_2$. Define $f: E \to F_1 \times F_2$ by $f(x) = (f_1(x), f_2(x))$. Prove that $f$ is differentiable and find its derivative. (Here $F_1 \times F_2$ is the Banach space equipped with the norm $\left \lVert { (y_1, y_2) } \right \lVert  = \left \lVert { y_1 } \right \lVert _{F_1} + \left \lVert { y_2 } \right \lVert _{F_2}$).
\end{problem}
\begin{proof}
	We show that, for any $x \in E$, the map $Df: x \mapsto (Df_1(x), Df_2(x))$ is linear, and that it satisfies the definition of the derivative,
	\[\lim_{\left \lVert { h } \right \lVert \to 0} \frac{f(x + h) - f(x) - Df(x)h}{\left \lVert { h } \right \lVert } = 0.\]
	Expanding the functions $f$ and $Df$ by their definitions, the above limit is the same as
	\[\lim_{\left \lVert { h } \right \lVert  \to 0} \frac{(f_1(x+h),f_2(x+h)) - (f_1(x),f_2(x)) + (Df_1(x)h, Df_2(x)h) }{\left \lVert { h } \right \lVert }.\]
	Now, using the componentwise definition of addition in the vector space $F_1 \times F_2$, this limit is equal to the following:
	\[\lim_{\left \lVert { h } \right \lVert \to 0} \left( \frac{f_1(x+h) - f_1(x) + Df_1(x)h}{\left \lVert { h } \right \lVert}, \frac{f_2(x+h) - f_2(x) + Df_2(x)h}{\left \lVert { h } \right \lVert } \right);\]
	The inner two limits exist and are equal to zero by the fact that $f_1$ and $f_2$ are differentiable. We finally need only see that convergence to $0$ in $F_1 \times F_2$ under the given norm is equivalent to componentwise convergence to $0$ in $F_1$ and $F_2$, and that the product of two linear functions into $F_1$ and $F_2$ is itself linear.
	\par First, we show that componentwise convergence implies convergence. Let $x^i$ be a sequence in $F_1 \times F_2$ such that the sequences of components $x^i_1$ and $x^i_2$ independently converge to $0$. Let $\varepsilon > 0$. Then there exist $N_1$, $N_2$ such that for all $i > N_1$, $\left \lVert { x^i_1 } \right \lVert_{F_1} < \varepsilon/2 $, and for all $j > N_2$, $\left \lVert { x^j_2 } \right \lVert _{F_2} < \varepsilon/2$. Then, for all $i > \max(N_1, N_2)$,
	\begin{align*}\left \lVert { x^i } \right \lVert &= \left \lVert { x^i_1 } \right \lVert_{F_1} + \left \lVert { x^i_2 } \right \lVert_{F_2}\\
		&< \varepsilon/2 + \varepsilon/2 = \varepsilon.
	\end{align*}
	So the sequence $x^i$ converges in $F_1 \times F_2$.
	\par Now, we show that $Df(x)$ is a linear map for all $x$ - more generally, that given any two linear maps $L_1: E \to F_1$, and $L_2: E \to F_2$, the product map $L_1 \times L_2 : E \to F_1 \times F_2$ is also linear.
	\begin{itemize}
		\item $(L_1 \times L_2)(\alpha x) = \alpha (L_1 \times L_2)(x)$: 
			\begin{align*}
				(L_1 \times L_2)(\alpha x) &= (L_1(\alpha x), L_2(\alpha x))\\
				&= (\alpha L_1(x), \alpha L_2(x))\\
				&= \alpha (L_1(x), L_2(x))\\
				&= \alpha(L_1 \times L_2)(x)
			\end{align*}
		\item $(L_1 \times L_2)(x+y) = (L_1 \times L_2)(x) + (L_1 \times L_2)(x)$:
			\begin{align*}
				(L_1 \times L_2)(x + y) &= (L_1(x + y), L_2(x + y))\\
				&= (L_1(x) + L_1(y), L_2(x) + L_2(y))\\
				&= (L_1(x), L_2(x)) + (L_1(y), L_2(y))\\
				&= (L_1 \times L_2)(x) + (L_1 \times L_2)(y)
			\end{align*}
	\end{itemize}
	So the product of any two linear maps is linear; in particular, the map $Df(x) = Df_1(x) \times Df_2(x)$ is linear. Since it satisfies the limit definition of the derivative, it is the derivative of $f$ at $x$. 
	\par Because $f$ is differentiable at every point $x$ of $E$, it is a differentiable map. Therefore, the product $f_1 \times f_2$ of any two differentiable maps into any two spaces is itself differentiable.
\end{proof}
\begin{problem}{3}
	Let $E,F,G$ be normed spaces and let $\varphi: E \times F \to G$ be a bilinear map (i.e. such that both maps $\varphi(\cdot, y): E \to G$ and $\varphi(x, \cdot) : E \to F$ are linear, for every $x \in E$ and $y \in F$).
	\begin{enumerate}[label=(\roman*)]
		\item Prove that $\varphi$ is continuous if and only if it is bounded; that is:
			\[\exists C>0 \forall x \in E \forall y \in F \; \left \lVert { \varphi(x,y) } \right \lVert \leq C \cdot \left \lVert { x } \right \lVert \cdot \left \lVert {  y } \right \lVert, \]
		\item Let $\mathcal L(E,F;G)$ be the linear space of all continuous bilinear mappings $\varphi$, as above. Prove that it is a normed space, with the norm defined as:
			\[\left \lVert { \varphi } \right \lVert : \sup\left\{ \left \lVert { \varphi(x,y) } \right \lVert ; \left \lVert { x } \right \lVert \leq 1, \left \lVert { y } \right \lVert \leq 1 \right\}.\]
		\item Prove that if $G$ is Banach then $\mathcal L(E,F;G)$ is also Banach.
	\end{enumerate}
\end{problem}
\begin{proof}
	Some of the following ideas were found in the lecture notes \cite{unimi}.
	\begin{enumerate}[label=(\roman*)]
		\item First, assume that $\varphi$ is continuous. If it were not bounded, there would exist some sequence $(x_n, y_n)$ of points in $E \times F$ whose norm under $\varphi$ was unbounded - that, say, $\left \lVert { \varphi(x_n,y_n) } \right \lVert > n^2 \left \lVert { x_n } \right \lVert \cdot \left \lVert { y_n } \right \lVert $.
			\par Now, we can use these points to construct a convergent sequence in $E\times F$ whose values under $\varphi$ diverge, contradicting continuity. Define the following sequences:
			\[\bar x_n = \frac{x_n}{n\left \lVert { x_n } \right \lVert } \text{  and  }\bar y_n = \frac{y_n}{n\left \lVert { y_n } \right \lVert }\]
			Then $\bar x_n \to 0$ and $\bar y_n \to 0$, but the value of $\left \lVert { \varphi(\bar x_n, \bar y_n) } \right \lVert > n^2 \frac{\left \lVert { x_n } \right \lVert}{n\left \lVert { x_n } \right \lVert }\cdot \\frac{\left \lVert { y_n } \right \lVert }{n\left \lVert { y_n } \right \lVert }$, meaning the value of $\varphi(\bar x_n, \bar y_n)$ cannot converge to the value $0 = \varphi(0,0)$. Thus $\varphi$ cannot be continuous.
			\par Now, We show the reverse implication. Let $\varphi$ be a bounded bilinear mapping $E\times F \to G$ with bounding constant $C$, and let $x_n \to x$ and $y_n \to y$ be convergent sequences in $E$ and $F$. We show that $\varphi(x_n, y_n) \to \varphi(x,y)$. 
			\par By convergence of the sequences $x_n$, $y_n$, there is some upper bound to the norms $\left \lVert { x_n } \right \lVert $ and $\left \lVert {  y_n } \right \lVert $; say $M$. Then
			\begin{align*}
				\left \lVert { \varphi(x_n, y_n) - \varphi(x,y) } \right \lVert &= \left \lVert { \varphi(x_n, y_n) - \varphi(x_n, y) + \varphi(x_n, y) - \varphi(x,y) } \right \lVert \\
				&\leq \left \lVert { \varphi(x_n, y_n) - \varphi(x_n, y) } \right \lVert +\left \lVert { \varphi(x_n, y) - \varphi(x,y) } \right \lVert \\
				&= \left \lVert { \varphi(x_n, y_n - y) } \right \lVert  + \left \lVert { \varphi(x_n - x, y) } \right \lVert \\
				&\leq C \cdot \left \lVert { x_n } \right \lVert \cdot \left \lVert { y_n - y } \right \lVert + C \cdot\left \lVert { x_n - x } \right \lVert \cdot \left \lVert { y } \right \lVert \\
				&\leq CM \cdot \left \lVert { x_n - x } \right \lVert + CM \cdot \left \lVert { y_n - y } \right \lVert,
			\end{align*} which converges to $0$ because $\left \lVert { x_n - x } \right \lVert $ and $\left \lVert { y_n - y } \right \lVert $ do. Therefore $\varphi$ is continuous.
		\item We show that the function
			\[ \left \lVert {  \varphi } \right \lVert = \sup\left\{ \left \lVert { \varphi(x,y) } \right \lVert ; \left \lVert { x } \right \lVert  \leq 1 , \left \lVert { y } \right \lVert \leq 1\right \},\]
			defined on bounded bilinear maps $E\times F \to G$, is a norm. We first note that this function is equivalent to a supremum defined on the whole of $E \times F$:
			\[ \left \lVert { \varphi } \right \lVert  = \sup\left\{ \frac{\left \lVert { \varphi(x,y) } \right \lVert }{\left \lVert { x } \right \lVert \cdot \left \lVert { y } \right \lVert }; x \in E, y \in F \right\}.\]
			\begin{itemize}
				\item $\left \lVert { \varphi + \psi } \right \lVert \leq \left \lVert {  \varphi } \right \lVert  + \left \lVert { \psi } \right \lVert $: This follows from the pointwise inequality
					\[\left \lVert { \varphi(x,y) + \psi(x,y) } \right \lVert \leq \left \lVert { \varphi(x,y) } \right \lVert + \left \lVert { \psi(x,y) } \right \lVert,\]
					which holds at all points $(x,y)$ of $E\times F$ by the triangle inequality in $G$.
				\item $\left \lVert { \alpha \varphi } \right \lVert  = \left \lvert { \alpha } \right \lvert \cdot \left \lVert { \varphi} \right \lVert $: This also follows pointwise, from the norm properties of $G$:
					\begin{align*}
						\left \lVert { (\alpha \varphi)(x,y) } \right \lVert &= \left \lVert { \alpha (\varphi(x,y)) } \right \lVert \\
						&= \left \lvert { \alpha } \right \lvert \cdot \left \lVert { \varphi(x,y) } \right \lVert 
					\end{align*}
				\item If $\left \lVert {  \varphi } \right \lVert = 0$, then $\varphi$ is the $0$ function: if $\left \lVert { \varphi } \right \lVert = 0$, then $\left \lVert { \varphi(x,y) } \right \lVert = 0$ at each point $(x,y)$, which implies that $\varphi(x,y) = 0$ for all $(x,y)$.
			\end{itemize}
		\item Assume that $G$ is Banach, and let $\{\varphi_n\}$ be a Cauchy sequence of functions in $\mathcal L(E,F;G)$, i.e. the norms $\left \lVert { \varphi_n - \varphi_m } \right \lVert $ go to zero. Because the norm in this space is the supremum of pointwise norms, then for any point $(x,y)$, the norms $\left \lVert { \varphi_n(x,y) - \varphi_m(x,y) } \right \lVert$ must go to $0$ as $n, m $ go to $\infty$. Because $G$ is Banach, the values $\varphi_n(x,y)$ must converge to a unique point, which we define as $\varphi(x,y)$. This gives a well-defined set-function $E\times F \to G$, and it is the only possible function that $\varphi_n$ may converge to. We now show that it is a bounded bilinear function.
			\par First, bilinearity. By symmetry in $x$ and $y$, it suffices to show that $\varphi(\cdot, y)$ is linear on $E$ for any fixed $y \in F$. 
			\begin{itemize}
				\item $\varphi(\alpha x, y) = \alpha \varphi(x,y)$: by the definition of $\varphi(x,y)$, we see
					\begin{align*}
						\varphi(\alpha x, y) &= \lim_{n \to \infty} \varphi_n(\alpha x, y)\\
						&= \lim_{n \to \infty } \alpha \varphi(x, y)\\
						&= \alpha \lim_{ n \to \infty} \varphi(x,y)\\
						&= \alpha \varphi(x,y)
					\end{align*}
				\item $\varphi(x_1 + x_2, y) = \varphi(x_1, y) + \varphi(x_2, y)$:
					\begin{align*}
						\varphi(x_1 + x_2, y) &= \lim_{n \to \infty} \varphi_n(x_1 + x_2, y)\\
						&= \lim_{n \to \infty }\left( \varphi_n(x_1, y) + \varphi_n(x_2, y) \right)\\
						&= \left( \lim_{n \to \infty} \varphi_n(x_1, y) \right) + \left( \lim_{n \to \infty}(x_2, y) \right)\\
						&= \varphi(x_1, y) + \varphi(x_2, y)
					\end{align*}
			\end{itemize}
			Now, we show that $\varphi$ is bounded: that there exists some global bound $C$ such that, for any $(x,y) \in E \times F$, 
			\[ \left \lVert { \varphi(x,y) } \right \lVert \leq C \cdot \left \lVert {  x } \right \lVert  \cdot \left \lVert { y } \right \lVert. \]
			Because the functions $\varphi_n$ converge in the norm on $\mathcal L(E,F;G)$, the bounds of each $\varphi_n$ converge to some number $C$:
			\[ \lim_{n \to \infty} \left( \sup\left\{ \left \lVert { \varphi_n(x,y); \left \lVert {  x } \right \lVert \leq 1, \left \lVert { y } \right \lVert \leq 1 } \right \lVert  \right\} \right) = C < \infty. \]
			We show that $C$ is a bound for $\varphi$. Let $(x,y) \in E \times F$. Then
			\begin{align*}
				\left \lVert { \varphi(x,y) } \right \lVert &= \left \lVert { \lim_{n\to \infty} \varphi_n(x,y) } \right \lVert 
			\end{align*}
			By continuity of the norm, we have
			\begin{align*}
				\left \lVert { \lim_{n \to \infty} \varphi_n(x,y) } \right \lVert &= \lim_{n \to \infty} \left \lVert { \varphi_n(x,y) } \right \lVert \\
				&= \frac{\lim_{n \to \infty} \left \lVert { \varphi_n(x,y) } \right \lVert }{\left \lVert { x } \right \lVert \cdot \left \lVert { y } \right \lVert }\cdot \left \lVert { x } \right \lVert \cdot \left \lVert {  y } \right \lVert \\
				&\leq \lim_{n \to \infty} \sup\left\{  { \frac{\left \lVert\varphi_n(x,y)\right\rVert}{\left \lVert { x  } \right \lVert \cdot \left \lVert { y } \right \lVert } } x \in E, y \in F \right\} \cdot \left \lVert {  x } \right \lVert \cdot \left \lVert { y } \right \lVert \\
				&= C \cdot \left \lVert { x } \right \lVert \cdot \left \lVert { y } \right \lVert 
			\end{align*}
			Thus, $\varphi$ is an element of $\mathcal L(E,F;G)$. We have shown that any Cauchy sequence converges to an element of this normed vector space; thus it is Banach.
	\end{enumerate}
\end{proof}
\begin{problem}{4}
	For $p \in (0,1)$ define $l_p$ and $\left \lVert { \cdot } \right \lVert _{l_p}$ by the standard formula. Show that $\left \lVert { \cdot } \right \lVert _{l_p}$ is not a norm.
\end{problem}
\begin{proof}
	The definition we use is that $l_p$ is the space of all sequences of real numbers $\left\{ x_i \right\}$ such that 
	\[\sum_{i = 1}^\infty \lvert x_i\rvert^{p}\]
	is less than $\infty$, with the norm $\left \lVert { x } \right \lVert _{l_p}$ defined as
	\[\left \lVert { x } \right \lVert = \left( \sum_{i=1}^\infty \left \lvert { x_i } \right \lvert ^{p}\right)^{1/p}\]
	This function works well with scalars, and takes $0$ to $0$, but it does not satisfy the triangle inequality, because the function $x \mapsto \left \lvert { x } \right \lvert ^p$ is not convex for $p \in (0,1)$. For example, let $p \in (0,1)$, and let $x$ be the sequence $\left\{ 2^{1/p}, 0, 0,\dots \right\}$, and $y$ the sequence $\left\{ 0, 2^{1/p}, 0,\dots \right\}$. Then $\left \lVert { x } \right \lVert  = \left \lVert { y } \right \lVert  = 2$, but $\left \lVert { x+y } \right \lVert = 4^{1/p}$, which is greater than $2+2$ for any $p \in (0,1)$.
\end{proof}
\begin{problem}{5}
	Give an example of a discontinuous linear map between normed spaces, so that:
	\begin{enumerate}[label=(\roman*)]
		\item Its graph is closed and its target space is Banach,
		\item Its graph is closed and its domain space is Banach.
	\end{enumerate}
\end{problem}
\begin{proof}
	\begin{enumerate}[label=(\roman*)]
		\item Let $A$ be the space of real-valued smooth functions on $[0,1]$ with the norm $\left \lVert { f } \right \lVert  = \sup_{x \in [0,1]}\left \lvert { f(x) } \right \lvert $, let $B$ be the Banach space $\R$, and let $T: A \to B$ be the map $f \mapsto f'(0)$. The derivative map is well-known to be linear, and its target space $\R$ is Banach. Further, the graph of $T$ is closed: if $\left\{ f_i \right\}$ is a sequence of smooth functions on $[0,1]$ which is Cauchy, and the sequence $\left\{ f_i'(0) \right\}$ is also Cauchy, then the sequence $\left\{ (f_i, f_i'(0)) \right\}$ in $A \times B$ converges to the value $(f, f'(0))$, where $f$ is the limit of the $f_i$ in the space $A$. 
			\par However, the derivative-at-0 operator is not continuous. Let $\left\{ f_n \right\}$ be the sequence of functions $f_n(x) = \frac{\sin(n^2x)}{n}$. This sequence converges in the supremum norm to $0$, but the derivative at $0$ grows without bound - $f_n'(0) = n$. 
		\item Incomplete.
	\end{enumerate}
\end{proof}<++>
\begin{thebibliography}{}
	\bibitem{unimi}{Vesely, Libor. Continuity of Bilinear Mappings. Universit\`a degli Studi di Milano, 2017.}
\end{thebibliography}
\end{document}
