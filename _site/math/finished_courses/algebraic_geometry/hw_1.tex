% --------------------------------------------------------------
% Andrew Tindall
% --------------------------------------------------------------
 
\documentclass[12pt]{article}
 
\usepackage[margin=1in]{geometry} 
\usepackage{amsmath,amsthm,amssymb,enumitem,hyperref}

\newcommand{\N}{\mathbb{N}}
\newcommand{\Q}{\mathbb{Q}}
\newcommand{\Z}{\mathbb{Z}}
\newcommand{\C}{\mathbb{C}}
\newcommand{\R}{\mathbb{R}}
\newcommand{\mc}[1]{\mathcal{#1}}
\newcommand{\e}{\varepsilon}
\newcommand{\bs}{\backslash}
\newcommand{\PGL}{\text{PGL}}
\newcommand{\Sp}{\text{Sp}}
\newcommand{\tr}{\text{tr}}
\newcommand{\Lie}{\text{Lie}}
\newcommand{\rec}[1]{\frac{1}{#1}}
\newcommand{\toinf}{\rightarrow \infty}


\theoremstyle{definition}
\newtheorem{proofpart}{Part}
\newtheorem{theorem}{Theorem}
\newtheorem{lemma}{Lemma}
\makeatletter
\@addtoreset{proofpart}{theorem}
\makeatother


\newenvironment{problem}[2][Problem]{\begin{trivlist}
\item[\hskip \labelsep {\bfseries #1}\hskip \labelsep {\bfseries #2.}]}{\end{trivlist}}
 
\begin{document}
 
%\renewcommand{\qedsymbol}{\filledbox}
 
\title{Homework 1}
\author{Andrew Tindall\\
Algebraic Geometry}
 
\maketitle
\begin{problem}{1}
\begin{enumerate}[label=(\alph*)]
    \item Show that the Zariski topology on an algebraic set is quasi-compact, in that every open cover has a finite subcover.
    \begin{proof}
	    Let $X \simeq \mathbb A^n$ be an affine algebraic set, and let 
	    \[ X = \bigcup_{i \in I} X_i\]
	    be a decomposition of $X$ into principal open sets, where each $X_i = (Z( (f_i)))^c$ for some polynomial $f_i$. Since $X = \bigcup X_i$, we see that $\bigcap Z( (f_i)) = \emptyset$. By a theorem of commutative algebra, if a set of polynomials $\left\{ f_i \right\}$ has no point at which they are all $0$, then the ideal generated by those polynomials is $(1)$; i.e. there is some finite linear combination 
	    \[1 = \sum_{j = 1}^k a_jf_{i_j}\]
	    But then the sets $X_{i_j}$ cover $X$
	    \[ \bigcup_{j=1}^{m} (Z( (f_{i_j}))^c) = X.\]
	    \par To see this, it is sufficient to see that 
	    \[\bigcap_{j=1}^m V( (f_{i_j})) = \emptyset.\]
	    By functoriality of $V$, this is equivalent to 
	    \[V\left( \sum_{j=1}^m (f_{i_j}) \right) = \emptyset\]
	    And, because $1 \in \sum_{j=1}^m (f_{i_j})$, it must be true that $V\left( \sum_{j=1}^m (f_{i_j})\right) $ has no points. 
	    \par So, we see that $X \simeq \mathbb A^n$ has the finite open cover $\left\{ X_{i_j} \right\}_{j=1	}^m$. Since any closed subset of a quasi-compact set is itself quasi-compact, we see that every algebraic set is quasi-compact.
    \end{proof}
    \item Show that Zariski topology on an algebraic set is not Hausdorff unless it is a finite set. In fact, show that a Hausdorff space is Noetherian if and only if it is finite.
    \begin{proof}
	    \par Let $X$ be a Hausdorff Noetherian topological space. Assume $X$ is irreducible, and let $U \subset X$ be open. We know that an open set in an irreducible Noetherian topological space is dense, so $U$ must be dense in $X$. However, it is impossible for an open set to be dense in an Hausdorff space, unless the set is the space itself. We can see that this is true by taking some $y \notin U$. Because $X$ is Hausdorff, there must be some open neighborhood $V$ of $y$ such that $V \cap U = \emptyset$. But then $V^c \supset \overline U$, and so $y \notin \overline U$, contradicting the fact that $U$ is dense. So, every open set in an irreducible Noetherian Hausdorff topologicla space must be equal to the space itself.
	    \par However, this condition implies that the space has only one point. This follows from Hausdorffness: let $x, y \in X$. If $x \neq y$, then there exist two disjoint open sets $U$ and $V$ containing $x$ and $Y$. But it must be true that $U = V = X$, so they cannot be disjoint. So, every point of $X$ is equal, showing that it contains (at most) one point.
	    \par Finally, we extend from the irreducible to the general case. Let $X$ be a Noetherian Hausdorff topological space. We see that $X$ must contain finitely many irreducible components, for if $X_1, X_2, \dots$ is an infinite number of irreducible components of $X$, then we have an infinite ascending chain
	    \[X_1 \subset X_1 \cup X_2 \subset \dots\]
	    which never stabilizes. So the set of irreducible components of $X$ must be finite. Since each component is itself finite, we see that $X$ has a finite number of points.
    \end{proof}
\end{enumerate}
\end{problem}
\begin{problem}{2}
\begin{enumerate}
\item Show that the Zariski topology on $\mathbb A^2$ is not the same as the product topology on $\mathbb A^1 \times \mathbb A^1$.
\begin{proof}
We show that there is a Zariski closed set in $\mathbb A^2$ which is not closed in the product topology on $\mathbb A^1 \times \mathbb A^1$. Let $D = Z(x - y)$ be the diagonal in $\mathbb A^2$, the set $\{(x,y) \in \mathbb A^2 \mid x = y\}$. This is the zero set of the ideal $(x-y)$, so it is Zariski closed.
\par However, $D$ is not closed in the product topology. If it were, there would be an open neighborhood $U_z$ around every point $z \in \mathbb A^1 \times \mathbb A^1 \backslash D$ such that $U_z \cap D = \emptyset$. Assume that this holds: let $z = (z_1, z_2)$ be an arbitrary point such that $z_1 \neq z_2$, so that it does not lie on the diagonal, and let $U_z$ be any open neighborhood around $z$ such that $U_z \cap D = \emptyset$. Because $U_z$ is a union of sets $X \times Y$, where both $X$ and $Y$ are open in $\mathbb A^1$, and if $U_z \cap D = \emptyset$, then $X \times Y \cap D = \emptyset$ as well, we can assume that $U_z = X \times Y$.
\par Since open sets are complements of Zariski closed sets in $\mathbb A^1$, we can assume that $X^c = Z(I_1)$ and $Y^c = Z(I_2)$, for some ideals $I_1, I_2 \subset \mathbf k[x]$. Since $ \mathbf{k}[x]$ is a PID, we can also assume that $I_1 = (f_1)$ and $I_2 = (f_2)$ for some polynomials $f_1, f_2$.
\par Because $X \times Y \cap D = \emptyset$, we see that every point $(x,y)$ of $D$ cannot lie in both $X$ and $Y$; therefore either $x \in X^c = Z((f_1))$ or $y \in Y^c = Z((f_2))$. This is equivalent to saying that either $f_1(x) = 0$ or $f_2(y) = 0$. 
\par Since every element $w \in \mathbf{k}$ can be mapped to an element $(w,w) \in D$, it must be true that \[\{w \in \mathbf{k} \mid f_1(w) = 0 \text{ or } f_2(w) = 0\} = \mathbf{k}\]
\par Because $\mathbf k$ is algebraically closed, it must be infinite; therefore either $Z((f_1))$ or $Z((f_2))$ is infinite. We can infer that either $f_1$ or $f_2$ is the zero polynomial; assume WLOG that it is $f_1$. Then $Z((f_1)) = \mathbb A^1$, and so $ X = Z((f_1))^c = \emptyset$. Thus $U_z = X \times Y = \emptyset$, contradicting the assumption that it contains $z$. Thus, $D$ cannot be closed in the product topology on $\mathbb A^1 \times \mathbb A^1$.
\end{proof}
\item Show that every nonempty Zariski open set is dense in $\mathbb A^1$.
\begin{proof}
Let $U \subset \mathbb A^1$ be a nonempty Zariski open set. We wish to show that $\overline U = \mathbb A^1$. By definition, $U$ is the complement of some closed set $Z(I_1)$, where $I_1$ is an ideal in $\mathbf k[x]$, and because $\mathbf k[x]$ is a principal ideal domain, we can assume that $I_1 = (f_1)$, for some $f_1 \in \mathbf k[x]$. Since $U$ is nonempty, there must be some points in $U$ which are not zeroes of $f_1$, and because $\mathbf k$ is algebraically closed and therefore infinite, any polynomial which is not identically zero has an infinite number of points where it is nonzero. Therefore, $U$ contains an infinite number of points.
\par Let $x \in \mathbb A^1$; we show that $x \in \overline U$. It suffices to show that, for any closed set $V$, if $U \subset V$, then $x \in V$. So, let $V \subset \mathbb A^1$ be a Zariski closed set which contains $U$. By definition, $V = Z(I_2)$ for some ideal $I_2 \subset \mathbf k[x]$, and again we can assume that $I_2 = (f_2)$ for some $f_2 \in \mathbf k [x]$. 
\par Since $U \subset Z((f_2))$, it must be true that every point of $U$ is a zero of $f_2$. Because $U$ is infinite, $f_2$ therefore has an infinite number of zeroes, which implies that it is identically zero. So, $V = Z((0)) = \mathbb A^1$. Since every closed set containing $U$ is $\mathbb A^1$, the closure of $U$ must be $\mathbb A^1$ as well, so it is dense.
\end{proof}
\end{enumerate}
\end{problem}
\begin{problem}{3}
Let $X$ be an algebraic set. Show that there is a one-to-one correspondence between points of $X$ and maximal ideals in $\mathbf{k}[X]$. Moreover, show that there is a one-to-one correspondence between irreducible closed subsets of $X$ and prime ideals in $\mathbf{k}[X]$.
\begin{proof}
Let $X = Z(I)$, for some $I \subset \mathbf k [x_1, \dots, x_n]$. The definition of $\mathbf k [X]$ is
\[
    \mathbf k[X] = \frac{\mathbf k[x_1, \dots, x_n]}{I}
\]
By a theorem of commutative algebra, the maximal ideals of the quotient ring $\mathbf k[X]$ correspond one-to-one to maximal ideals $\mathfrak m \subset k[x_1, \dots, x_n]$ such that $\mathfrak m$ contains $I$. By the Nullstellensatz, such a maximal ideal is determined uniquely by a point $p = (p_1, \dots, p_n) \in \mathbb A^1$.
\textit{Incomplete}
\end{proof}
\end{problem}
\begin{problem}{4}
Let $\mathbf{k}$ be a field of characteristic $\neq 2$. Decompose the algebraic set $X \subset \mathbb A^3$ defined by the equations $x^2 + y^2 +z^2 = 0$ and $x^2 - y^2 - z^2 + 1 = 0$, into irreducible components.
\begin{proof}
We show that this variety is the disjoint union of two separate circles:
\begin{align*}
x = \frac{i}{\sqrt{2}}, y^2 + z^2 &= \frac{1}{2}\\
x = -\frac{i}{\sqrt{2}}, y^2 + z^2 &= \frac{1}{2}
\end{align*}
This can be seen algebraically, by first substituting $x^2 = -y^2 - z^2$ into the second equation (where we can divide by $2$ because the characteristic of the field is $\neq 2$:
\begin{align*}
-2y^2 - 2z^2 &= -1\\
y^2 + z^2 &= \frac{1}{2}
\end{align*}
And then substituting this into the first:
\begin{align*}
    x^2 + \frac{1}{2} &= 0\\
    x &= \pm \frac{i}{\sqrt{2}}
\end{align*}
So, $X$ is the product of the reducible variety $\{i/\sqrt{2}, -i/\sqrt{2}\}$ with the circle $y^2 + z^2 = \frac{1}{2}$.
\end{proof}
\end{problem}
\begin{problem}{5}
	Let $X = \{(t^2, t^3) \mid t \in \mathbf k\}$ be the cuspidal cubic curve. Show that $X$ is an irreducible algebraic variety which is not isomorphic to the affine line $\mathbb A^1$. On the other hand, let $\mathbf{k}(X)$ be the quotient field of the coordinate ring $\mathbf k[X]$ (the \textit{field of rational functions} on $X$). Show that the field $\mathbf k$ is isomorphic (as a $\mathbf k$-algebra) to the field of rational polynomials $\mathbf k(t)$ in one variable.
\begin{proof}
	\par An equivalent definition of $X$ is as the variety corresponding to the ideal $(x^3 - y^2)$. It is clear that for each point $(t^2, t^3)$, the polynomial $x^3 - y^2$ is zero; in the opposite direction, any nonzero point $(x, y)$ in the zero set of this polynomial is equal to $(t^2, t^3)$, where $t := y/x$. Also, the point $(0,0)$ is the image of $t = 0$. So we can write $X = Z( (x^3- y^2))$.
	\par We show first that $I(X)$ is prime, implying that $X$ is irreducible. Since $I(X)$ is generated by one polynomial, it is sufficient to see that $x^3 - y^2$ is irreducible in $\mathbf k[x,y]$. Let $f, g$ be two polynomials in $\mathbf k [x,y]$ such that $f \cdot g = x^3 - y^2$. One of these polynomials must have a term with nonzero exponent in $x$; assume WLOG that it is $f$. Also, one must have a term with nonzero exponent in $y$. If this holds for $g$, then their product would have a term $x^ky^j$ with nonzero exponent in both $x$ and $y$, which does not hold. So, $f$ must have a term with nonzero exponent in $y$, and $g$ cannot have such a term. 
	\par But then if $g$ had a term with nonzero exponent in $x$, it would also be true that their product $fg$ had a term $x^ky^j$, with nonzero exponent in both $x$ and $y$, which cannot hold. So, all terms in $g$ must have zero exponent in both $x$ and $y$, meaning $g$ is a constant, and $f$ is a scalar multiple of $x^3 - y^2$. Since the polynomial is irreducible, the ideal $(x^3 - y^2)$ is prime, and the variety $X = I(x^3 - y^2)$ is irreducible.
	\par Now we will show that $X$ is not isomorphic to $\mathbb A^1$. If there were such an isomorphism $\varphi : X \simeq \mathbb A^1$, functoriality of the coordinate ring would give us an isomorphism $\varphi^* : \mathbf k[x] \simeq \mathbf k[X]$. However, the coordinate ring $\mathbf k[X]$ of $X$ is not isomorphic to $k[x]$. To see this, it is easiest to show that one is integrally closed in its field of fra ctions, while the other is not.
	\par In fact, the fields of fractions of both are the same, $\mathbf k(t)$. This is true by definition for $ \mathbf{k}[x]$. In the case of $\mathbf k[X]$, we see that a generic element of the fraction field can be written in terms of the element $t = \overline y / \overline x$, where $\overline y$ and $\overline x$ are the images of $x$ and $y$ in the coordinate ring of $X$, and we also see that every ratio of polynomials in $t$ is an element of the fraction field of $\mathbf k[X]$.
	\par First, let $\frac{f(\overline x,\overline y)}{g(\overline x,\overline y)}$ be an element of the fraction field of $\mathbf k[X]$. Since $t^3 = (\overline y^3) / (\overline x^3) = \overline y$, and $t^2 = (\overline y^2 )/(\overline x^2) = \overline x$, we can write both $f$ and $g$ as polynomials in $t$. 
	\par On the other hand, let $f(t)/g(t)$ be an element of $\mathbf{k}(t)$. Multiplying both the top and bottom by $t^2$, we have a ratio of polynomials $f'(t)$ and $g'(t)$, where every term in $f'$ and $g'$ has exponent $\geq 2$ in $t$. Because every number $\geq 2$ can be written as a sum $k \cdot 2 + j \cdot 3$, where $k, j \geq 0$, every monomial in both $f $ and $g$ can be rewritten in terms of $x$ and $y$:
	\[t^n = (t^{k \cdot 3})(t^{j \cdot 2}) = \overline y^{2k} \cdot \overline x^{3j}\]
	So, the fraction field of $\mathbf k[X]$ is equal to $\mathbf k(t)$. This is the same as the field of fractions of $\mathbf k[x]$, but $\mathbf k[X]$ is not integrally closed in $\mathbf k(t)$; for instance, $x \neq f(t)$ for any monic polynomial $f$. On the other hand, $ \mathbf k[x]$ is integrally closed in $\mathbf k(x)$, since it is a UFD. Therefore, the two rings cannot be isomorphic. 
	\par This descends to the algebraic sets $X$ and $\mathbb A^1$. Because their coordinate rings are not isomorphic, the two varieties cannot be isomorphic. This is despite the fact that the fraction fields of their coordinate rings are isomorphic.
\end{proof}
\end{problem}
\begin{problem}{6}
Let $X \subset \mathbb A^{n+1}$ be a hypersurface defined by a polynomial $f \in \mathbf k[t_1, \dots , t_n, x]$. Let 
\[
f = a_mx^m + \cdots + a_0 
\]
where $a_i \in \mathbf{k} [t_1, \dots, t_n]$. Consider the projection map $\varphi : X \to \mathbb A^n$ given by $(t_1, \dots, t_n, x) \mapsto (t_1, \dots, t_n)$. Prove that $\varphi$ is a finite map if and only if $a_m$ is a nonzero constant polynomial.
\begin{proof}
	\textit{incomplete}
\end{proof}
\end{problem}
\begin{problem}{7}
Give an example of a morphism $\varphi : \mathbb A^2 \to  \mathbb A^2$ whose image is neither closed nor open.
\begin{proof}
(found on wikipedia): We show that the morphism $\varphi: \mathbb A^2 \to \mathbb A^2$ defined by $\varphi(x,y) = (x, xy)$ has an image that is neither open nor closed.
\par The image of $\varphi$ is $\mathbb A^2 \backslash \{(0,y) \mid y \neq 0\}$. Any point $(x,z)$ with $x \neq 0$ is the image of $(x, z/x)$ under $\varphi$, and $(0,0)$ is the image of any point $(0,y)$. On the other hand, any point $(0,z)$ with $z \neq 0$ cannot be the image of any point $(0,y)$ under $\varphi$, since $0\cdot y = 0$ for all $y$.
\par This set is neither open nor closed. It cannot be closed, as its intersection with the closed set $\{(x, y) \mid y = 1\}$, which is the variety corresponding to the ideal $(y-1)$, is the non-closed set $\{(x,y) \mid y = 1, x \neq 0\}$. It is also not open, as its intersection with the open set $\{(x,y) \mid x \neq 0\}$ is the non-open set $\{(0,0\}$.
\end{proof}
\end{problem}
\end{document}
