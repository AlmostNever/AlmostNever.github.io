% --------------------------------------------------------------
% Andrew Tindall
% --------------------------------------------------------------
 
\documentclass[12pt]{article}
 
\usepackage[margin=1in]{geometry} 
\usepackage{amsmath,amsthm,amssymb,enumitem,hyperref}

\newcommand{\N}{\mathbb{N}}
\newcommand{\Q}{\mathbb{Q}}
\newcommand{\Z}{\mathbb{Z}}
\newcommand{\C}{\mathbb{C}}
\newcommand{\R}{\mathbb{R}}
\newcommand{\mc}[1]{\mathcal{#1}}
\newcommand{\e}{\varepsilon}
\newcommand{\bs}{\backslash}
\newcommand{\PGL}{\text{PGL}}
\newcommand{\Sp}{\text{Sp}}
\newcommand{\tr}{\text{tr}}
\newcommand{\Lie}{\text{Lie}}
\newcommand{\rec}[1]{\frac{1}{#1}}
\newcommand{\toinf}{\rightarrow \infty}


\theoremstyle{definition}
\newtheorem{proofpart}{Part}
\newtheorem{theorem}{Theorem}
\makeatletter
\@addtoreset{proofpart}{theorem}
\makeatother


\newenvironment{problem}[2][Problem]{\begin{trivlist}
\item[\hskip \labelsep {\bfseries #1}\hskip \labelsep {\bfseries #2.}]}{\end{trivlist}}
 
\begin{document}
 
%\renewcommand{\qedsymbol}{\filledbox}
 
\title{Homework 3}
\author{Andrew Tindall\\
Algebra 1}
 
\maketitle
\begin{section}{Homework 1}
\begin{problem}{1}
	Show that a finite group generated by two involutions is dihedral.
	\begin{proof}
		We recall the definition of the dihedral group $D_n$ as
		\[D_n = \langle r, s \mid r^2 = s^n = e, rs = sr^{-1}\rangle.\]
		Let $G$ be a finite group generated by $a$, $b$, where $a$ and $b$ are involutions: 
		\[a^2 = b^2 = e\]
		Now, since $a = a^{-1}$ and $b = b^{-1}$, every element in $G$ can be written as a finite-length word in $a$ and $b$. Also, since $a^2 = b^2 = e$, we only need to consider words with no two identical adjacent letters. Therefore, the following two lists exhaust the group:
		\begin{align*}
			a&, ab, aba, abab, \dots\\
			b&, ba, bab, baba, \dots
		\end{align*}
		Since $G$ is finite, neither of these lists can be infinite. Let $e = \underbrace{aba\dots}_{\text{$m$ terms}} $ be the first occurrence of an identity element in the first list $a, ab, \dots$.
		\par It must be true that $e$ is of the form $(ab)^n$ for some $n$, because otherwise we could cancel on the left and right - say $e = aba\dots aba = (ab)^na$ for some $n$. Then 
		\begin{align*}
			e &= aea\\
			&= a(ab)^n a^2\\
			&= (ba)^{n-1}b
		\end{align*}
		But then we would have 
		\begin{align*}
			e &= beb\\
			&= b(ba)^{n-1}b^2\\
			&= (ab)^{n-2}a
		\end{align*}
		Which contradicts our choice of $\underbrace{aba\dots}_{\text{ $m$ terms}}$ as the shortest such word. So, $(ab)^{n} = e$ for some $e$.
		\par We see that this gives $(ba)^{n}$ as the shortest word in the second list which is equal to the identity, as well. First, we see that if we have $(ab)^n = e$, then it must also be true that $(ba)^n = e$:
		\begin{align*}
			(ba)^n &= a^2 (ba)^n\\
			&= a(ab)^n a\\
			&= a^2\\
			&= e
		\end{align*}
		Now, since $(ba)^n = e$ also implies that $(ab)^n =e$, it must be true that the smallest such $n$ are equal. So, there are at most $2n$ elements in $G$.
		\par In fact, there are exactly $2n$ elements in $G$. The two 
	\end{proof}<++>
\end{problem}
\begin{problem}{2}
	What is the order of the largest cyclic subgroup of $S_n$?
	\begin{proof}
		We show that the order of the largest cyclic subgroup of $n$, which we denote $o_n$, is equal to the maximum Least Common Multiple of any of the sets of nonzero numbers which partition $n$ (Landau's Function on $n$).
		\par First, note that since $\left \lvert {  \langle x\rangle } \right \lvert  = \text{ord}(x)$, we are equivalently looking for the greatest order of any element $x \in S_n$. Now, let $x \in S_n$. The element $x$ has a unique cycle decomposition, up to reordering of the cycles and cyclic reordering of the elements in the cycles. The multiset of lengths of the cycles of an element is its \textit{cycle type}: for example, the cycle type of $x = (123)(4567)(89)(10,11)$ is $\left\{ 3,4,2,2 \right\}$. We also see that $x^k = 0$ if and only if $k \equiv 0$ modulo $3,4,2$, and $2$, because if $k \equiv 0$ modulo $3$, then $k = 3j$ for some $j$, and
		\begin{align*}
		    x^k &= (123)^k\cdot (4567)^k \cdot (89)^k \cdot (10,11)^k\\
		    &= (123)^{3j} \cdot (4567)^k \cdot (89)^k \cdot (10,11)^k\\
		    &= ((123)^3)^j\cdot (4567)^k \cdot (89)^k \cdot (10,11)^k\\
		    &= e \cdot (4567)^k \cdot (89)^k \cdot (10,11)^k
		\end{align*}
		And similarly, if the orders of $(4567)$, of $(89)$, and of $(10,11)$ divide $k$, then $x^k = e$. The lowest exponent $k$ such that $x^k = e$ is exactly the least common multiple of the set $\left\{ 3,4,2 \right\}$.
		\par Since the cyclic decomposition of an element is unique up to order of the cycles, and cyclic reordering of the elements in the cycles, we see that the multiset of \textit{lengths} of cycles is unique for every element of $S_n$. Since the lengths of the cycles always adds to $n$, if we count trivial cycles, every element has a unique multiset partitioning $n$ which determines its order in $S_n$. 
		\par We will also see that every multiset $S$ partitioning $n$ corresponds to at least one element of $S_n$: for instance, if $n=6$ and $S = \left\{ 1,3,2 \right\}$, then one such element with this multiset as its cycle class is $(23)(456) = (1)(23)(456)$, and the order of this element is $\text{Lcm}(\left\{ 1,2,3 \right\}) = 6$. 
		Since the order of each element is equal to the least common multiple of the lengths of the cycles in its decomposition, and the possible lengths of the cycles of the elements of $S_n$ correspond exactly to the partitions of $n$ into positive integers, we reach the following conclusion:
		\par For any $n$, the largest cyclic subgroup has order equal to the maximum possible $\text{Lcm}$ over all positive-integer partitions $S$ of $n$. Letting $P(n)$ be the set of all positive partitions of $n$, we have a formula for the maximum order $o_n$ of any cyclic subgroup of $S_n$:
		\[o_n = \text{max}_{S \in P(n)} (\text{Lcm}(S)).\]
	\end{proof}
\end{problem}
\begin{problem}{4}
	If $G$ is a non-Abelian $p$-group of order $p^3$, then $Z(G) = [G,G]$.
	\begin{proof}
		
	\end{proof}
\end{problem}

\begin{problem}{7}
	Prove that, if $p$ and $q$ are primes, a group of order $p^2 q$ cannot be simple.
	\begin{proof}
		<++>
	\end{proof}
\end{problem}
\begin{problem}{15}
	If $m$ divides the order of a nilpotent group $G$, then $G$ contains a subgroup of order $m$.
	\begin{proof}
		<++>
	\end{proof}
\end{problem}
\begin{problem}{19}
	Let $P$ be a $p$-group, and $\Phi(P)$ be the intersection of all maximal subgroups of $P$. Prove that, for $Q \trianglelefteq P$, the group $P/Q$ is elementary Abelian if and only if $\Phi(G) \leq Q$.
\end{problem}
\end{section}
\begin{section}{Homework 4}
\begin{problem}{6}
    Show that a maximal subgroup of a solvable group is of prime power index in the group.
    \begin{proof}
    <++>
    \end{proof}
\end{problem}
\begin{problem}{7}
    A minimal normal subgroup of a nilpotent group is isomorphic to $\Z_p$, with $p$ a prime.
    \begin{proof}
        <++>
    \end{proof}
\end{problem}
\begin{problem}{8}
    Find all composition series of $S_3 \times \Z_2$. Verify the Jordan-H\"older theroem directly in this case.
    \begin{proof}
        <++>
    \end{proof}
\end{problem}
\end{section}
\begin{section}{Homework 5}
\begin{problem}{7}
    Show that $\text{PSL}(2,7) \simeq \text{GL}(3,2)$.
    \begin{proof}
    <++>
    \end{proof}
\end{problem}
\end{section}
\end{document}
