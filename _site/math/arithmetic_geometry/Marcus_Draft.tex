% --------------------------------------------------------------
% Andrew Tindall
% --------------------------------------------------------------
 
\documentclass[12pt]{article}
 
\usepackage[margin=1in]{geometry} 
\usepackage{amsmath,amsthm,amssymb,enumitem,hyperref,tikz-cd}

\newcommand{\N}{\mathbb{N}}
\newcommand{\Q}{\mathbb{Q}}
\newcommand{\Z}{\mathbb{Z}}
\newcommand{\C}{\mathbb{C}}
\newcommand{\R}{\mathbb{R}}
\newcommand{\mc}[1]{\mathcal{#1}}
\newcommand{\e}{\varepsilon}
\newcommand{\bs}{\backslash}
\newcommand{\PGL}{\text{PGL}}
\newcommand{\Sp}{\text{Sp}}
\newcommand{\tr}{\text{tr}}
\newcommand{\Lie}{\text{Lie}}
\newcommand{\rec}[1]{\frac{1}{#1}}
\newcommand{\toinf}{\rightarrow \infty}


\theoremstyle{definition}
\newtheorem{proofpart}{Part}
\newtheorem{theorem}{Theorem}
\makeatletter
\@addtoreset{proofpart}{theorem}
\makeatother


\newenvironment{problem}[2][Problem]{\begin{trivlist}
\item[\hskip \labelsep {\bfseries #1}\hskip \labelsep {\bfseries #2.}]}{\end{trivlist}}
 
\begin{document}
 
%\renewcommand{\qedsymbol}{\filledbox}
 
\title{Marcus, Ch. 2\\
Selected Problems}
\author{Andrew Tindall}
 
\maketitle
\begin{problem}{8}
	\begin{enumerate}[label= (\alph*)]
		\item 
	Let $\omega = e^{2\pi i p}$, $p$ an odd prime. Show that $\Q[\omega]$ contains $\sqrt{p}$ if $p \equiv 1$ (mod $4$), and $\sqrt{-p}$ if $p \equiv -1$ (mod $4$). Express $\sqrt{-3}$ and $\sqrt{-5}$ as polynomials in the appropriate $\omega$.
	\begin{proof}
		It is hinted for the first half of this problem that we want to use the fact, proven in Marcus, ch. 2, that $\text{disc}(\omega) = \pm p^{p-2}$, with $+$ holding iff $p \equiv 1$ (mod $4$). Another useful fact is that 
		\[ \text{disc}(\omega) = \prod_{1 \leq r < s \leq n} (\omega^r - \omega^s)^2.\]
		\par We also note that $p$ is assumed to be an odd prime: therefore, $p - 3$ is even and nonnegative: let $k = (p - 3)  / 2$, so that $p^{p - 2} = p (p^k)^2$. Putting all of these facts together, we have
		\[\left( \prod_{1 \leq r < s \leq n} (\omega^r - \omega^s)\right )^2 = \pm p (p^k)^2.\]
		So, it must be true that
		\[\left( \frac{\prod_{ 1 \leq r < s \leq n} (\omega^r - \omega^s)}{p^k} \right)^2 = \pm p,\]
		So indeed the field $\Q[\omega]$ must contain $\sqrt{ \pm p}$, with $+$ holding if and only if $p \equiv 1$ (mod $4$).
		\par Since this proof is constructive, we can use it to get a formula for $\sqrt{\pm p}$ in any given cyclotomic field. However, the term $\prod_{1 \leq r < s \leq n} (\omega^r - \omega^s)$ grows quickly with the degree of the given cyclotomic field. For example, for $\omega_3$ a primitive $3$rd root of unity, it is a polynomial of degree $8$ in $\omega_3$, before reducing to a quadratic polynomial. However, using $\omega_3 = -\frac{1}{2} + \frac{\sqrt{3}}{2} i$, it is not too hard to find
		\begin{align*}\omega_3 - \omega_3^2 &= \left( -\frac{1}{2} + \frac{\sqrt{3}}{2}i \right) - \left( -\frac{1}{2} - \frac{\sqrt{3}}{2} i \right)\\
			&= \sqrt{3} i\\
			&= \sqrt{-3}
		\end{align*}
		\par In the case of $\omega_5$ a primitive root of unity, it is not as easy to find a polynomial formula for $\sqrt{5}$ in terms of $\omega_5$. For example, one primitive $5$th root of unity is
		\[ \omega_5 = \frac{\sqrt{5} - 1}{4} + \frac{\sqrt{10 + 2 \sqrt 5}}{4} i\]
		Instead, we will use the formula derived above. First, we can simplify $\prod_{1 \leq r < s \leq 5} (\omega_5^r - \omega_5^s)$:
		\[ \prod_{1 \leq r < s \leq 5} (\omega_5^r - \omega_5^s) = \prod_{1 \leq r < s \leq 5} \omega_5^r(1 - \omega_5^{s - r}).\]
		There are $4$ terms where $r$ is equal to $1$, $3$ where it is equal to $2$, $2$ where it is $3$, and $1$ where it is $4$. So, we can factor out $\omega_5^{4 + 3 \cdot 2 + 2 \cdot 3 + 4} = \omega_5^{20}$, which is equal to $1$, leaving us with 
		\[\prod_{1 \leq r < s \leq 5} (1 - \omega_5^{s - r}).\]
		There are $4$ ways to choose $1 \leq r <  s \leq 5$ such that $r - s = 1$, $3$ ways to choose them such that $r - s = 2$, and so on. So, we can rewrite this as
		\[\prod_{1 \leq r < s \leq 5} (1 - \omega_5^{s - r}) = (1 - \omega_5)^4(1 - \omega_5^2)^3(1-\omega_5^3)^2(1-\omega_5^4).\]
		Here, we can rewrite \[(1 - \omega_5^4) = (1 - \omega_5^2)(1 + \omega_5^2),\]
		\[(1 - \omega_5^3) = (1 - \omega_5)(1 + \omega_5 + \omega_5^2), \text{ and}\] \[(1 - \omega_5^2) = (1 - \omega_5)(1 + \omega_5).\]
		So, we end up with
		\[ (1 - \omega_5)^4(1 - \omega_5^2)^3(1-\omega_5^3)^2(1-\omega_5^4) = (1 - \omega_5)^7 (1 + \omega_5)^2(1 + \omega_5 + \omega_5^2)(1 + \omega_5^2)\]
	\end{proof}
\item Show that the $8$th cyclotomic field contains $\sqrt{2}$.
	\par This is not hard to see if we take the primitive $8$th root of unity $\omega_8$ to be
	\[\omega_8 = \frac{\sqrt2}{2} + \frac{\sqrt 2}{2} i.\]
	From this, we see that 
	\begin{align*}\omega_8 + \omega_8^7 &=  \frac{\sqrt 2}{2} + \frac{\sqrt 2}{2} i + \frac{\sqrt 2}{2} - \frac{\sqrt 2}{2} i \\
		&= \frac{\sqrt 2}{2} + \frac{\sqrt 2}{2}\\
		&= \sqrt{2}.
	\end{align*}
\item Show that every quadratic field is contained in a cyclotomic field: in fact, $\Q[\sqrt m]$ is contained in the $d$th cyclotomic field, where $d = \text{disc}(\mathbb A \cap \Q[\sqrt{m}])$.
	\begin{proof}
		\textit{incomplete}
	\end{proof}
\end{enumerate}	
\end{problem}
\begin{problem}{11}
	\begin{enumerate}[label=(\alph*)]
		\item Suppose all roots of a monic polynomial $f \in \Q[x]$ have absolute value $1$. Show that the coefficient of $x^r$ has absolute value $\leq \binom{n}{r}$, where $n$ is the degree of $f$ and $\binom{n}{r}$ is the binomial coefficient.
			\begin{proof}
				Since all roots of $f$ must exist in $\C$, in $\C[x]$ we can write $f$ as
				\[f(x) = \prod_{1 \leq i \leq n} (x - \alpha_i),\]
				Where $\alpha_i$ are the roots of $f$ in $\C$. The coefficients of $f$ can be calculated from the $\alpha_i$s: by Vieta's formulas, the coefficient $a_r$ of $x^r$ in a monic polynomial with roots $\alpha_1, \dots , \alpha_n$ is 
				\[a_r = (-1)^{r}\sum_{1 \leq i_1 < i_2 < \dots < i_r < n} \left( \prod_{j = 1}^r \alpha_{i_j} \right)\]
				Since all terms $-1, \alpha_1, \dots , \alpha_i$ in this expression have absolute value $1$ in $\C$, we see that
				\begin{align*}
					\left \lvert { a_r } \right \lvert &= \left \lvert { (-1)^r \sum_{1 \leq i_1 < i_2 < \dots < i_r < n} \left( \prod_{j = 1}^r \alpha_{i_j} \right) } \right \lvert \\
					&\leq \sum_{1 \leq i_1 < i_2 < \dots < i_r} \left \lvert {  \prod_{j = 1}^r \alpha_{i_j}} \right \lvert \\
					&= \sum_{1 \leq i_1 < i_2 < \dots < i_r} 1\\
					&= \binom{n}{r}
				\end{align*}
			\end{proof}
		\item Show that there are only finitely many algebraic integers $\alpha$ of fixed degree $n$, all of whose conjugates (including $\alpha$) have absolute value $1$.
			\begin{proof}
			An algebraic integer $\alpha$ of degree $n$, all of whose conjugates have absolute value $1$, has an irreducible polynomial of the kind discussed in part (a); since all the roots of $f$ in $\C$ have absolute value $1$, the coefficients of $f$ must have absolute value $\leq \binom{n}{r}$. However, since the coefficients of $f$ all lie in $\Z$, there are only $2\binom{n}{r} + 1$ possibilities for each coefficient $a_r$ of $f$:
			\[ - \binom{n}{r}, - \binom{n}{r} + 1, \dots , -1, 0, 1, \dots , \binom{n}{r} - 1, \binom{n}{r}\]
			Therefore, there are only 
			\[\prod_{1 \leq r \leq n} \left( 2 \binom{n}{r} + 1 \right)\]
			possible minimal polynomials. Since only $n$ algebraic integers $\alpha_1, \dots , \alpha_n$ can share the same minimal polynomial $f$, there can be no more than 
			\[n \prod_{1 \leq r \leq n} \left( 2 \binom{n}{r} + 1 \right)\]
			algebraic integers of degree $n$, all of whose conjugates have absolute value $1$. 
			\end{proof}
		\item Show that $\alpha$ (as in (b)) must be a root of $1$. (Show that its powers are restricted to a finite set.)
			\begin{proof}
				Let $\alpha$ be an algebraic integer of degree $n$, all of whose conjugates have absolute value $1$. If $\alpha_i$ is a conjugate of $\alpha$, then $\alpha_i^k$ is a conjugate of $\alpha^k$, for any $k \geq 1$ - this shows that each algebraic integer in the sequence $\alpha, \alpha^2, \alpha^3, \dots$ has absolute value $1$, and also each of its conjugates has absolute value $1$, since
				\[\left \lvert { \alpha^k } \right \lvert  = \left \lvert { \alpha } \right \lvert ^k = 1, \text{ and}\]
				\[ \left \lvert { \alpha_i^k } \right \lvert = \left \lvert { \alpha_i } \right \lvert ^k = 1.\]
				But as we have seen, the set of all algebraic integers whose conjugates all have absolute value $1$ is finite. As a subset of this set, the set $\left\{ \alpha, \alpha^2, \alpha^3, \dots \right\}$ is also finite: $\alpha^j = \alpha^k$ for some $j \neq k$ - assume WLOG that $j < k$. Then $\alpha^{j}(1 - \alpha^{k-j}) = 0$, showing that $\alpha$ is a $(k - j)$th root of unity. 
			\end{proof}
	\end{enumerate}
\end{problem}
\begin{problem}{12}
	Now we can prove Kummer's lemma on units in the $p$th cyclotomic field, as stated before exercise $26$, chapter $1$: Let $\omega = e^{2\pi i / p}$, $p$ an odd prime, and suppose $u$ is a unit in $\Z[\omega]$.
	\begin{enumerate}[label=(\alph*)]
		\item Show that $u / \overline u$ is a root of $1$. (Use $11$(c)) above and observe that complex conjugation is a member of the Galois group of $\Q[\omega]$ over $\Q$.) Conclude that $u / \overline u = \pm \omega^k$  for some $k$.
		\begin{proof}
			\par Because $u$ is a unit in $\Z[\omega]$, so is $\overline u$, and so $u / \overline u$ is a well-defined member of $\Z[\omega]$. We know already that $\left \lvert { u / \overline u } \right \lvert  = 1$, as this holds for every number. What we want to show is that this holds for each conjugate $\sigma_i(u / \overline u)$ of $u / \overline u$, for each embedding of $\Q[\omega]$ in $\C$. 
			\par Since complex conjugation is a member of the Galois group of $\Q[\omega]$ over $\Q$, we know it also corresponds to an embedding $\sigma_j$ of $\Q[\omega]$ in $\C$. It is shown in Marcus, Ch. 2, that the Galois group of $\Q[\omega]$ over $\Q$ is isomorphic to $\Z_p^*$. In particular, it is commutative, so $\sigma_i \circ \sigma_j = \sigma_j \circ \sigma_i$. So, we have
			\begin{align*}
				\sigma_i(u / \overline u) &= \sigma_i (u / \sigma_j(u))\\
				&= \sigma_i(u) / \sigma_i(\sigma_j(u))\\
				&= \sigma_i(u) / \sigma_j(\sigma_i(u))\\
				&= \sigma_i(u)/\overline{\sigma_i(u)}
		\end{align*}
		So, every conjugate of $u / \overline u$ also has absolute value $1$, and it fulfills the hypothesis of problem $11$. As we showed there, this implies that it is a root of unity. The only roots of unity in $\Q[\omega]$ are the $p$th roots of unity, and these are exactly $\pm \omega^k$ for $1 \leq k \leq p$.
		\end{proof}
	\item Show that the $+$ sign holds: Assuming $u / \overline u = -\omega^k$, we have $u^p = - \overline{u^p}$; whow that this implies that $u^p$ is divisible by $p$ in $\Z[\omega]$. But this is impossible since $u^p$ is a unit.
		\begin{proof}
			Assume we did have $u / \overline u = - \omega^k$: then $u = -\overline u \omega^k$, and
			\begin{align*}
				u^p &= (- \overline u \omega^k)^p\\
			&= (-1)^p {\overline u}^p \omega^{pk}\\
			&= -({\overline u}^p)
			\end{align*}
			\textit{Incomplete - why does this imply that $p \mid u^p$?}
		\end{proof}
	\end{enumerate}
\end{problem}
\begin{problem}{13}
	Show that $1$ and $-1$ are the only units in the ring $\mathbb A \cap \Q[\sqrt m]$, $m$ squarefree, $m < 0$, $m \neq -1, -3$. What if $m = -1$ or $-3$?
	\begin{proof} 
		Let us first split into cases for the value of $m$ modulo $4$. If $m \equiv 0$ or $3$ (mod $4$), then $\mathbb A \cap \Q[\sqrt m]$ is equal to 
	\end{proof}
\end{problem}
\begin{problem}{14}
	Show that $1 + \sqrt 2$ is a unit in $\Z[\sqrt 2]$, but not a root of $1$. Use the powers of $1 + \sqrt 2$ to generate infinitely many solutions to the diophantine equation $a^2 - 2b^2 = \pm 1$.
	\begin{proof}
		We see that the element $-\overline{(1 + \sqrt{2})} = -1 + \sqrt 2 \in \Z[\sqrt 2]$ is an inverse to $1 + \sqrt 2$:
		\begin{align*}
			(1 + \sqrt 2)(-1 + \sqrt 2) &= -1 + (\sqrt{2})^2\\
			&= -1 + 2\\
			&= 1
		\end{align*}
		However, it is not a root of unity, as its absolute value, $1 + \sqrt 2$, is greater than $1$ - any root of unity must be of absolute value $1$ in $\C$.
		\par Therefore, we see that the numbers $\alpha_1, \alpha_2, \dots $, where $\alpha_i = (1 + \sqrt 2)^i$, form an infinite set, and also that they are all units. Also, because $\alpha_1^{-1} = - \overline{\alpha_1}$, we have
	\[\alpha_i^{-1} = (-1)^i\overline{\alpha_i}.\]
	If we write $\alpha_i = a_i + b_i\sqrt{2}$ for some $a_i, b_i \in \Z$, we have
	\begin{align*}
		a_i^2 - 2b_i^2 &= (a_i + b_i \sqrt{2})(a_i - b_i \sqrt{2})\\
		&= (-1)^n \alpha_i \alpha_i^{-1}
		&= \pm 1,
	\end{align*}
	With $+$ holding iff $i = 0$ (mod $2$). So, there are an infinite number of solutions to both Diophantine equations $a - 2b^2 = 1$ and $a - 2b^2 = -1$.
	\end{proof}
\end{problem}
\begin{problem}{15}
	\begin{enumerate}[label=(\alph*)]
		\item Show that $\Z[\sqrt {-5}]$ contains no element whose norm is $2$ or $3$.
			\begin{proof}
				Let $\alpha = a + b\sqrt{-5}$ be an arbitrary element of $\Z[\sqrt{-5}]$, and write $N(\alpha)$ for $N^{\Q[\sqrt{-5}]}$. Then $N(\alpha)$ is equal to 
				\begin{align*}
					\alpha \cdot \overline \alpha &= (a + b \sqrt{-5})(a - b \sqrt{-5})\\
					&= (a^2 + 5b^2).
				\end{align*}
				Therefore $N(\alpha)$ (mod $5$) is equal to $a^2$. The quadratic residues modulo $5$ are $0$, $1$ and $4$, so there is no way that $N(\alpha) \equiv 2$ or $3$ (mod $5$). So, $N(\alpha) \neq 2$ or $3$.
			\end{proof}
		\item Verify that $2 \cdot 3 = (1 + \sqrt {-5})(1 - \sqrt{-5})$ is an example of non-unique factorization in the number ring $\Z[\sqrt{-5}]$.
			\begin{proof}
				\begin{align*}
					(1 + \sqrt{-5})(1 - \sqrt{-5}) &= 1 + 5\\
					&= 6\\
					&= 2 \cdot 3
				\end{align*}
				However, the elements $1 + \sqrt{-5}$, $1 - \sqrt{-5}$, $2$, and $3$ are all irreducible in $\Z[\sqrt{-5}]$: 
				\par The norm of $1 + \sqrt{-5}$ is $6$, so if it could be factored as two nonunits $a\cdot b = 1 + \sqrt{5}$, then we would have $N(a) \cdot N(b) = N(\alpha) = 6$. Assuming $N(a) \leq N(b)$, we would have either $N(a) = 1$ and $N(b) = 6$, or $N(a) = 2$ and $N(b) = 3$. We have seen that the second is impossible, and we also know that $N(a) = 1$ only if $a = \pm 1$, and we have assumed it is not a unit. So, $1 + \sqrt{-5}$ is irreducible, as is $1 - \sqrt{-5}$ by the same argument.
				\par Similarly, $2$ must be irreducible because its norm is $4$; if it could be factored into two nonunits $a$ and $b$, with $N(a) \leq N(b)$, then either $N(a) = N(b) = 2$, which is impossible, or $N(a) = 1$, so it is a unit. Finally, $3$ is irreducible, since its norm is $9$, and the norms of its nonunit factors would have to be $3$ and $3$ or $1$ and $9$, which is also impossible. So, $6$ has non-unique factorization into irreducibles in $\Z[\sqrt{-5}]$.
			\end{proof}
	\end{enumerate}
\end{problem}
\begin{problem}{21}
	Let $\alpha$ be an algebraic integer and let $f$ be a monic polynomial over $\Z$ (not necessarily irreducible) such that $f(\alpha) = 0$. Show that $\text{disc}(\alpha)$ divides $N^{\Q[\alpha]}f'(\alpha)$. 
	\begin{proof}
		Let $g$ be the minimal polynomial of $\alpha$. It is a theorem in Marcus, Ch. 2, that $\text{disc}(\alpha) = N^{\Q[\alpha]}(g'(\alpha))$. Because $f(\alpha) = 0$, it must have $g$ as a factor: say $f = gh$, for some polynomial $h \in \Z[x]$. Then $f' = g'h + gh'$, by the product rule. So, calculating:
		\begin{align*}
			N^{\Q[\alpha]}(f'(\alpha)) &= N^{\Q[\alpha]}(g'(\alpha)h(\alpha) + g(\alpha)h'(\alpha))\\
			&= N^{\Q[\alpha]}(g'(\alpha)h(\alpha) + 0)\\
			&= N^{\Q[\alpha]}(g'(\alpha))N^{\Q[\alpha]}(h(\alpha))\\
			&= \text{disc}(\alpha) \cdot N^{\Q[\alpha]}(h(\alpha))
		\end{align*}
		So, we do see that $\text{disc}(\alpha)$ divides $N^{\Q[\alpha]}(f'(\alpha))$.
	\end{proof}
\end{problem}
\begin{problem}{22}
	Let $K$ be a number field of degree $n$ over $\Q$ and fix algebraic integers $\alpha_1, \dots , \alpha_n \in K$. We know that $d = \text{disc}(\alpha_1, \dots , \alpha_n)$ is in $\Z$; we will show that $d \equiv 0$ or $1$ (mod $4$). Letting $\sigma_1, \dots , \sigma_n$ denote the embeddings of $K$ in $\C$, we know that $d$ is the square of the determinant $\left \lvert { \sigma_i(\alpha_j) } \right \lvert $. This determinant is a sum of $n!$ terms, one for each permutation of $\left\{ 1, \dots , n \right\}$. Let $P$ denote the sum of the terms corresponding to even permutations, and let $N$ denote the sum of the terms (without negative signs) corresponding to odd permutations. Thus $d = (P - N)^2 = (P + N)^2 - 4PN$. Complete the proof by showing that $P + N$ and $PN$ are in $\Z$.
	\par In particular we have $\text{disc}(\mathbb A \cap K) \equiv O$ or $1$ (mod $4$). This is known as \textit{Stickelberger's criterion.}
\end{problem}
\begin{problem}{23}
	Just as with the trace and norm, we can define the relative discriminant $\text{disc}_K^L$ of an $n$-tuple, for any pair of number fields $K \subset L$, $[L:K]=n$.
	\begin{enumerate}[label=(\alph*)]
		\item Generalize Theorems $6-8$ and the corollary to Theorem $6$.
			\begin{proof}
				<++>
			\end{proof}
		\item Let $K \subset L \subset M$ be number fields, $[L : K] = n$, $[M:L] = m$ and let $[\alpha_1, \dots , \alpha_n]$ and $[\beta_1, \dots , \beta_m]$ be bases for $L$ over $K$ adn $M$ over $L$, respectively. Establish the formula 
		\[\text{disc}_K^M(\alpha_1\beta_1, \dots , \alpha_n\beta_m) = (\text{disc}_K^L(\alpha_1, \dots , \alpha_n))^mN_K^L \text{disc}_L^M(\beta_1, \dots , \beta_m).\]
			\begin{proof}
				<++>
			\end{proof}
		\item Let $K$ and $L$ be number fields satisfying the conditions of Corollary $1$, Theorem $12$. Show that $(\text{disc} T) = (\text{ disc} R)^{[L:Q]}(\text{disc }S)^{[K:\Q]}$. (This can be used to obtain a formula for $\text{disc}(\omega)$, $\omega = e^{2 \pi i / m}$)
	\end{enumerate}
\end{problem}
\end{document}
