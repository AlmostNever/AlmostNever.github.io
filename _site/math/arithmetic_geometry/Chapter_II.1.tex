% --------------------------------------------------------------
% Andrew Tindall
% --------------------------------------------------------------
 
\documentclass[12pt]{article}
 
\usepackage[margin=1in]{geometry} 
\usepackage{amsmath,amsthm,amssymb,enumitem,hyperref,tikz-cd}
\setlist{  
  listparindent=\parindent,
  parsep=0pt,
}
\newcommand{\N}{\mathbb{N}}
\newcommand{\Q}{\mathbb{Q}}
\newcommand{\Z}{\mathbb{Z}}
\newcommand{\C}{\mathbb{C}}
\newcommand{\R}{\mathbb{R}}
\newcommand{\mc}[1]{\mathcal{#1}}
\newcommand{\e}{\varepsilon}
\newcommand{\bs}{\backslash}
\newcommand{\PGL}{\text{PGL}}
\newcommand{\Sp}{\text{Sp}}
\newcommand{\tr}{\text{tr}}
\newcommand{\Lie}{\text{Lie}}
\newcommand{\rec}[1]{\frac{1}{#1}}
\newcommand{\toinf}{\rightarrow \infty}


\theoremstyle{definition}
\newtheorem{proofpart}{Part}
\newtheorem{theorem}{Theorem}
\makeatletter
\@addtoreset{proofpart}{theorem}
\makeatother


\newenvironment{problem}[2][Problem]{\begin{trivlist}
\item[\hskip \labelsep {\bfseries #1}\hskip \labelsep {\bfseries #2.}]}{\end{trivlist}}
 
\begin{document}
 
%\renewcommand{\qedsymbol}{\filledbox}
 
\title{Hartshorne Exercises}
\author{Andrew Tindall \\
Ch II.1}
\maketitle
\begin{problem}{1}
Let $ A$ be an abelian group, and define the \textit{constant presheaf} on $X$ associated to $A$ to be the presheaf $U \mapsto A$, for all $U \neq \emptyset$, with restriction maps the identity. Show that the constant sheaf $\mathcal A$ defined in the text is the sheaf associated to this presheaf.
\begin{proof}
Denote the constant presheaf by $\mathcal B$. Then, the sheaf associated to $\mathcal B$, as defined in Hartshorne (p. 64), is the sheaf $\mathcal B^+$ defined as follows:
\par For any open set $U \subset X$, $\mathcal B^+$ is the set of functions $s: U \to \bigcup_{P \in U} \mathcal B_P$, to the union of stalks of $\mathcal B$ over points of $U$, with the following two properties:
\begin{enumerate}
    \item For each $P \in U$, $s(P) \in \mathcal B(P)$
    \item For each $P \in U$, there is a neighborhood $V$ of $P$, contained in $U$, and an element $t \in \mathcal B(V)$, such that, for all $Q \in V$, the germ $t_Q$ of $t$ at $Q$ is equal to $s(Q)$.
\end{enumerate}
\par We also recall that the \textit{constant sheaf} $\mathcal A$ takes an open set $U$ to the set of continuous maps $s: U \to A$, where $A$ is given the discrete topology (where every set is both open and closed). The addition of two functions $s$ and $t$ in $\mathcal A(U)$ can be done componentwise, so this sheaf takes each set $U$ to the product of $n$ copies of $A$, where $n$ is the number of connected components of $U$.
\par To show that these two sheaves are isomorphic, we need to construct an isomorphism $\varphi_U$ between $\mathcal A(U)$ and $\mathcal B^+(U)$, for each open $U \subset X$, and we also have to show that these isomorphisms are compatible with the restriction maps $\rho_{UV}$ and $\rho'_{UV}$ of $\mathcal A$ and $\mathcal B^+$, for any open $V \subset U$, \textit{i.e.} that the following square commutes:
\[
\begin{tikzcd}
\mathcal{A}(U) \arrow[r, "\varphi_U"] \arrow[d, "\rho_{UV}"] & \mathcal B^+(U) \arrow[d, "\rho'_{UV}"]\\
\mathcal A(V) \arrow[r, "\varphi_V"] &\mathcal B^+(V)
\end{tikzcd}
\]
\par First, we can show that the stalk $\mathcal B_P$, for each point $P \in X$, is isomorphic to $A$. By definition of the stalk, any element $f \in \mathcal B_P$ can be represented by a pair $(g,U)$, where $g \in \mathcal{B}(V) = A$. Any two pairs $(g,U)$ and $(h,V)$ represent the same element of the stalk $\mathcal B_P$ whenever there is some open set $W \subset U \cap V$, such that $g \lvert_W = h\lvert_V$. In this case, because the restriction maps are the identity, any element $f$ which is represented by $(g, U)$ can also be represented by $(g, X)$. On the other hand, any element $g$ of $A$ defines an element $(g,X)$ of the stalk $\mathcal B_P$. The addition on the stalks is defined by $(g,X) + (h, X) = (g + h, X)$, so indeed the stalk is isomorphic to $A$.
\par Now, let $U \subset X$ be some open set of $X$, and let $s \in \mathcal B^+(U)$ be some section of $\mathcal B^+$ over $U$. Then $s$ is a function which takes each $P \in U$ to an element of $\mathcal B_P$. We have seen that $\mathcal B_P$ is isomorphic to $A$, so we can view $s$ as a function 
\[U \to \bigcup_{P \in U} A = A.\]
\par So, $\mathcal B^+(U)$ consists of certain set-functions $U \to A$. The added restriction of the sheaf $\mathcal B^+$ is that, for any $P$, there must be some open neighborhood $V$ on which $s$ is represented by a single value of $\mathcal B(V)$: on this open set, there is some $g \in A$, so that $s(Q) = g$ for all $Q \in V$. We will see that in fact this makes the $\mathcal B^+(U)$ into exactly $\mathcal A(U)$.
\par First, let $s \in \mathcal B^+(U)$ be some section of $\mathcal B^+$ over $U$. We want to see that $s$ is continuous when considered as a map $U \to A$, where $A$ is given the discrete topology. The open sets of the discrete topology are generated by the singletons $\{a\}$, $a \in A$, so it suffices to show that the inverse image $s^{-1}(\{a\})$ is an open set of $U$.
\par Let $P \in s^{-1}(\{a\})$. Then, by the definition of $\mathcal B^+(U)$, there is some open neighborhood $V$ of $P$, on which $s$ is identically equal to $s(P) = a$. Therefore, $V \subset s^{-1}(\{a\})$. Since the set contains an open neighborhood around each of its points, it must be open. So, the map $s$ is continuous, and is a valid element of $\mathcal A(U)$.
\par Next, let $t$ be an element of $\mathcal A(U)$, \textit{i.e.} a continuous map $U \to A$. We want to see that every element $P \in U$ is contained in an open neighborhood, on which $t$ is constant. Because $t$ is continuous and $A$ has been given the discrete topology, the inverse image $t^{-1}(t(P))$ is an open set which contains $P$, and on which $t$ is constant. So, $t$ is a valid element of $\mathcal B^+(U)$. So, we see that the values of the sheaves are the same on every open set $U \subset X$.
\par It only remains to show that the restriction maps in each sheaf are identical. This follows quickly from the definitions of both sheaves: in both cases, the restriction $\rho_{UV}$ is simply the restriction of functions on $U$ to the smaller domain $V \subset U$. So, the two sheaves are in fact the same.
\end{proof}
\end{problem}
\begin{problem}{2}
    \begin{enumerate}[label=(\alph*)]
        \item For any morphism of sheaves $\varphi: \mathcal F \to \mathcal G$, show that for each point $P$, $(\text{ker }\varphi)_P = \text{ker}(\varphi_P)$ and $(\text{im }\varphi)_P = \text{im}(\varphi_P)$.
        \begin{proof}
            We recall that an element of the localization of a sheaf or presheaf $\mathcal F$ at a point is represented by a pair $(f,U)$, for some open $U \subset X$ which contains $P$, and some value $f \in \mathcal F(U)$, with two representatives $(f,U)$ and $(g,V)$ being considered equal if there is some neighborhood $W \subset U \cap V$ of $P$ where $f\lvert_W = g\lvert_W$. The localization of a morphism of presheaves $\varphi : \mathcal F \to \mathcal G$ takes an element $(f,U) \in \mathcal F_P$ to $(\varphi(f), U) \in \mathcal G_P$. A morphism of sheaves is simply a morphism of the underlying presheaves.
            \par The kernel of a morphism of sheaves is the presheaf defined by taking $U$ to $\text{ker}(\varphi_U)$, the kernel of the abelian group homomorphism $\varphi_U : \mathcal F(U) \to \mathcal G(U)$. This presheaf is in fact a sheaf.
            \par What we wish to prove here about kernels of sheaves is that the operation of taking the kernel of a morphism commutes with the operation of localization at a point. So, let $P \in X$ be a point of the space $X$, and let $\varphi: \mathcal F \to \mathcal G$ be a morphism of sheaves on $X$. 
            \par Let $(f, U) \in (\text{ker }\varphi)_P$. Then $f \in (\text{ker } \varphi)(U)$ - i.e. $f$ is an element of $\mathcal F(U)$ such that $\varphi_U(f) = 0$. But then $(f,U)$ can also be seen as an element of $\mathcal F_P$. The localization of $\varphi$ at $P$ takes this point to 
            \begin{align*}
            \varphi_P((f,U)) &= (\varphi_U(f) ,U)\\
            &= (0,U)\\ 
            &= 0
            \end{align*}
            So indeed, $(f,U)$ is an element of $\text{ker}(\varphi_P)$.
            \par Now, let $(g,V)$ be an element of $\text{ker}(\varphi_P)$. This means that $\varphi_P(g,V)$ is equal to $0$ in the stalk $\mathcal G_P$, i.e. that $\varphi_U(g)\lvert_W = 0$, for some neighborhood $W$ of $P$. Because $\varphi$ is a morphism of sheaves, it commutes with localization, so $\varphi_V(g)\lvert_W = \varphi_W(g\lvert_W)$. But since $g \lvert_W$ gets taken to $0$ by $\varphi_W$, we see that
            \begin{align*}
                (\varphi_V(g),V) &= ((\varphi_V(g))\lvert_W, W)\\
                &=(\varphi_W(g\lvert_W),W)\\
                &=(0, W)
            \end{align*}
            So $\varphi_V(g)$ localizes to $0$ near $P$, implying that $(g,V)$ is an element of $(\text{ker}(\varphi))_P$. So, the two stalks are equal.
            \par Now, we want to show that the same property holds for the image sheaf of $\varphi$. The image of a morphism of sheaves $\varphi$ is the sheaf associated to the presheaf $U \mapsto \text{im}(\varphi_U)$. 
            \par Let $(f, U)$ be an element of $\text{im}(\varphi_P)$. So, there is some $(g, V) \in \mathcal G_P$, such that $\varphi_P( (g,V)) = (\varphi_V(g),V) = (f,U)$. By definition of equality in this stalk, there is some open $W \subset V \cap U$ such that $f \lvert_W = (\varphi_V(g))\lvert_W$. Since $\varphi$ commutes with restriction, we have $(\varphi_V(g))\lvert_W = \varphi_W(g\lvert_W)$. Finally, since an element of a stalk of a presheaf is naturally also in the corresponding stalk of the associated sheaf, we have
            \begin{align*}
                (f,U) &= (f\lvert_W, W)\\
                &= ((\varphi_V(g))\lvert_W, W)\\
                &= (\varphi_W(g\lvert_W), W)\\
                &\in \text{im}(\varphi)_P
            \end{align*}
            So, $f$ is an element of $\text{im}(\varphi)_P$.
            \par Now, let $(g, V)$ be an element of $(\text{im}(\varphi))_P$. There must be some neighborhood $U$ of $P$ on which $g$ is equal to some section $f$ of the presheaf $U \mapsto \text{im}(varphi_U)$. There must exist some $h \in \mathcal F(U)$ such that $f = \varphi_U(h)$. Letting $V' = U \cap V$ and $g' = f\lvert_{V'}$, we have
            \begin{align*}
                (g,V) &= (g',V')\\
                &= (f\lvert_{V'}, V')\\
                &= ((\varphi_U(f))\lvert_{V'}, V')\\
                &= (\varphi_{V'}(f\lvert{V'}), V')\\
                &= \varphi_P(f\lvert_{V'}, V')
            \end{align*}
            So, the element $(g,V)$ of $\text{im}(\varphi)_P$ is also in $\text{im}(\varphi_P)$.
        \end{proof}
        \item Show that $\varphi$ is injective (respectively, surjective) if and only if the induced ma[ on the stalks $varphi_P$ is injective (respectively, surjective) for all $P$.
        \begin{proof}
        We first deal with injectivity. We recall that the definition of an injective morphism $\varphi$ of sheaves is one where $\text{ker}(\varphi) = 0$, the sheaf which is constantly $0$. Let $\mathcal F$ and $\mathcal G$ be sheaves on a space $X$.
        \par First, let $\varphi: \mathcal F \to \mathcal G$ be an injective morphism of sheaves, and let $P$ be a point of $X$. We want to show that $\varphi_P$ is injective; \textit{i.e.} that $\text{ker}(\varphi_P)$ is injective. But we have seen above that $\text{ker}(\varphi_P) = (\text{ker}(\varphi))_P$, which is $0$ by assumption on $\varphi$. So $\varphi_P$ must be injective.  
        \par Now, let $\varphi: \mathcal F \to \mathcal G$ be an arbitrary morphism, and assume that $\varphi_P$ is injective for every $P \in X$. Let $U$ be some open set of $X$, and let $f \in \text{ker}(\varphi)(U)$ be a section of the kernel of $\varphi$ over $U$.
        \par For any $P \in X$, the pair $(f, U)$ is an element of $\text{ker}(\varphi)_P$. As seen above, this means that it is also an element of $\text{ker}(\varphi_P)$. But because $\varphi_P$ is injective, this means that $(f, U)$ is $0$ in $\mathcal F_P$. There must exist some neighborhood $V_P \subset U$ of $P$ such that $f\lvert_{V_P} = 0$. But because $\mathcal F$ is a sheaf, and $f$ is a section of $\mathcal F(X)$ which is $0$ when restricted to each set of the cover $\{V_P\}_{P \in X}$ of $X$, the properties of a sheaf imply that $f$ is zero as well. So $\text{ker}(\varphi)$ is zero, and $\varphi$ is injective.
        \par We now want to show the same thing for surjectivity. First, let $\varphi: \mathcal F \to \mathcal G$ be a surjective morphism of sheaves on $X$, and let $P$ be a point of $X$. As we saw above, $\text{im}(\varphi_P) = (\text{im}(\varphi))_P$. Since $\text{im}(\varphi) = \mathcal G$, we see that $\text{im}(\varphi_P) = \mathcal G_P$, and so $\varphi_P$ is surjective.
        \par Now, let $\varphi: \mathcal F \to \mathcal G$ be a morphism of sheaves on $X$, such that $\varphi_P$ is surjective for every $P \in X$. Let $U \subset X$ be an open set. We want to show that $\text{im}(\varphi)(U) = \mathcal G(U)$. Since the inclusion $\text{im}(\varphi)(U) \subset \mathcal G(U)$ is known, it suffices to show that every section of $\mathcal G(U)$ is an element of $\text{im}(\varphi)(U)$.
        \par Let $f \in \mathcal G(U)$. Then, for each $P \in U$, the stalk $f_P$ of $f$ is an element of $\mathcal G_P$. Since $\varphi_P$ is surjective, we know that $f_P \in \text{im}(\varphi_P)$. As shown above, this implies that $f_P \in (\text{im}(\varphi))_P$. So, there is some $W_P \subset U$ containing $P$, and $g_{W_P} \in \text{im}(\varphi)(W_P)$, such that $f\lvert_W = g_{W_P}$. 
        \par The sets $W_P$ cover $U$, and any two $G_{W_P}$ and $G_{W_Q}$ are both equal to $f\lvert{W_P \cap W_Q}$ on the intersection $W_P \cap W_Q$. Since $\text{im}(\varphi)$ is a sheaf, there must be some element $h$ of $\text{im}(\varphi)(U)$, such that $h\lvert_{W_P} = g_{W_P}$ for each $W_P$. Because $\text{im}(\varphi)$ is a subsheaf of $\mathcal G$, $h$ is also an element of $\mathcal G(U)$. But $f$ is also an element of $\mathcal G(U)$ which is equal to $h$ when restricted to each set of the cover $\{W_P\}_{P\in U}$ of $U$. By the fact that $\mathcal G$ is a sheaf, this implies that $f$ and $h$ are equal, and so $f$ is an element of $\text{im}(\varphi)$ as well. So, $\text{im}(\varphi)(U) = \mathcal G(U)$ for each open set $U$ of $X$, and $\varphi$ is surjective.
        \end{proof}
        \item Show that a sequence 
        \[
        \cdots \to \mathcal F^{i-1} \rightarrow{\varphi_{i-1}} \mathcal F^i \rightarrow{\varphi_i} \mathcal F^{i+1} \to \cdots
        \]
        of sheaves and morphisms is exact if and only if for each $P \in X$ the corresponding sequence of abelian groups is exact.
        \begin{proof}
        We recall that the definition of an exact sequence of sheaves is that $\text{im}(\varphi_{i}) = \text{ker}(\varphi_{i+1})$, for each $i$. 
        \par First, assume that the sequence is exact. Then, localizing, we have $\text{im}(\varphi_{i})_P = \text{ker}(\varphi_{i+1})_P$ for each $P \in X$. As shown in part (a), this implies that $\text{im}((\varphi_i)_P) = \text{ker}((\varphi_{i+1})_P)$ for each $P$, and so each sequence
        
        \[
        \cdots \to \mathcal F_P^{i-1} \rightarrow{(\varphi_{i-1})_P} \mathcal F_P^i \rightarrow{(\varphi_i)_P} \mathcal F_P^{i+1} \to \cdots
        \]
        is indeed exact.
        \par Now, assume that for each $P \in X$, the sequence obtained by localizing is exact, and let $i$ be arbitrary. 
        \par First, let $U$ be an open set of $X$, and let $g \in \text{im}(\varphi_i)(U)$. Then, for each $P \in U$, $g_P \in \text{im}(\varphi_i)_P$, which we have seen means $g_P \in \text{im}((\varphi_i)_P)$. By assumption, this means that $g_P \in \text{ker}((\varphi_{i+1})_P)$, so finally we see that $g_P \in \text{ker}(\varphi_{i+1})_P$. So, there is some neighorhood $W_P$ of $P$, and some element $g_{W_P}$ of $\text{ker}(\varphi_{i+1})(W_P)$, such that $g\lvert_{W_P}$ = $g_{W_P}$.
        \par The open sets $W_P$ cover $U$, and any two elements $g_{W_P}$ and $g_{W_Q}$ are both equal to $g\lvert_{W_P \cap W_Q}$ when restricted to the intersection $W_P \cap W_Q$, so there is some section $h$ of the sheaf $\text{ker}(\varphi)(U)$ such that $h\lvert_{W_P} = g_{W_P}$, for each $W_P$. But both $g$ and $h$ have this property, and they are both elements of the sheaf $\mathcal F^{i+1}$, so they must be equal. So, $g$ is an element of $\text{ker}(\varphi_{i+1})(U)$.
        \par Now, assume that $h$ is an element of $\text{ker}(\varphi_{i+1})$. Then, for each $P \in U$, the localization $h_P$ is an element of $\text{ker}(\varphi)_P$. Following the chain of equalities
        \begin{align*}
            \text{ker}(\varphi_{i+1})_P &=  \text{ker}((\varphi_{i+1})_P)\\
            &= \text{im}((\varphi_i)_P)\\
            &= \text{im}(\varphi_i)_P
        \end{align*}
        We see that $h_P$ is an element of $\text{im}(\varphi_i)_P$, for each $P \in X$. So, for each $P$, there is some neighborhood $W_P$ of $P$, and en element $h_{W_P}$ of $\text{im}(\varphi_i)(W_P)$. These elements are equal on the intersections of their domains, and the union of the $W_P$s is $U$ so there must be an element $g$ of $\text{im}(\varphi_i)(U)$ such that $g\lvert_{W_P} = h_{W_P}$. But then the elements $g$ and $h$ of the sheaf $\mathcal F^{i+1}$ are equal when restricted to each set $W_P$ of a cover of $U$, and so $g=h$. So, $h \in \text{im}(\varphi_i)$. 
        \par By the two arguments above, we must have $\text{im}(\varphi_i) = \text{ker}(\varphi_{i+1})$ for each $i$. So, the sequence of sheaves and morphisms is exact.
        \end{proof}
    \end{enumerate}
\end{problem}
\begin{problem}{3}
\begin{enumerate}[label=(\alph*)]
    \item Let $\varphi: \mathcal F \to \mathcal G$ be a morphism of sheaves of $X$. Show that $\varphi$ is surjective if and only if the following condition holds: for every open set $U \subset X$, and for every $s \in \mathcal G(U)$, there is a covering $\{U_i\}$ of $U$, and there are elements $t_i \in \mathcal F(U_i)$, such that $\varphi(t_i) = s\lvert_{t_i}$, for all $i$.
    \begin{proof}
    First, assume that $\varphi$ is surjective, i.e. that $\text{im}(\varphi) = \mathcal G$. Let $U$ be an open set of $X$ and let $s \in \mathcal G(U)$. Then $s \in \text{im}(\varphi)$, which is the sheaf associated to the presheaf $U \mapsto \text{im}(\varphi_U)$. By definition, this means that, for each $P \in U$, there is a neighborhood $V_P$ of $P$, on which $s$ is represented by an element $s_{V_P}$ of $\text{im}(\varphi_{V_P})$. So, $s_{V_P} = \varphi_{V_P}(t_{V_P})$, for some $t_{V_P} \in \mathcal G(V_P)$, and indeed the open sets $V_P$ form an open cover of $U$. And, we see the following:
    \begin{align*}
        \varphi(t_{V_P}) &= s_{V_P}\\
        &= s\lvert_{V_P}
    \end{align*}
    Together, these are exactly the requirements of the theorem.
    \par Now, assume that, for each open $U \subset X$, and for each $s \in \mathcal G(U)$, there exists some open cover $U_i$ of $U$, with elements $t_i \in \mathcal F(U_i)$ such that $\varphi_{U_i}(t_i) = s\lvert_{U_i}$. We want to show that $\text{im}(\varphi) = \mathcal G$. 
    \par Let $U$ be an arbitrary open set of $X$, and let $s$ be an arbitrary section of $\mathcal G$ over $U$. We want to show that $s$ is an element of $\text{im}(\varphi)$. Let $U_i$ be the cover guaranteed by the hypothesis, and let $t_i$ be the element of $\mathcal F(U_i)$ such that $\varphi_{U_i}(t_i) = s_i$. Then any two elements $\varphi_{U_i}(t_i) \in \text{im}(\varphi)(U_i)$ and $\varphi_{U_j}(t_j) \in \text{im}(\varphi)(U_j)$ are both equal to $s\lvert{U_i \cap U_j}$ when restricted to $U_i \cap U_j$, and by the fact that $\mathcal F$ is a sheaf, there is an element $s'$ of $\text{im}(\varphi)(U)$ such that $s'\lvert_{U_i} = \varphi_{U_i}(t_i)$ for each $U_i$. But $s$ already fulfils this property, and $\varphi G$ is a sheaf, so $s = s'$. Therefore, $s$ is an element of $\text{im}(\varphi)(U)$, and we see that $\varphi$ is surjective.
    \end{proof}
    \item Give an example of a surjective morphism of sheaves $\varphi: \mathcal F \to \mathcal G$ and an open set $U$ such that $\varphi_U: \mathcal F(U) \to \mathcal G(U)$ is not surjective.
    \begin{proof}
        \textit{TODO: After defining skyscraper sheaves}
    \end{proof}
\end{enumerate}
\end{problem}
\begin{problem}{4}
\begin{enumerate}[label=(\alph*)]
\item Let $\varphi: \mathcal F \to \mathcal G$ be a morphism of presheaves such that for each open $U \subset X$, the associated map $\varphi_U: \mathcal F \to \mathcal G$ is injective. Show that the induced map $\varphi^+: \mathcal F^+ \to \mathcal G^+$ of associated sheaves is injective.
    \begin{proof}
        We recall that the definition of the sheaf associated to a presheaf gives functions $\theta_{\mathcal F}: \mathcal F \to \mathcal F^+$ and $\theta_{\mathcal G}: \mathcal G \to \mathcal G^+$, and the universal property of sheafification implies that $(\theta_{\mathcal G} \circ \varphi): \mathcal F \to \mathcal G^+$ can be lifted to a function $\psi: \mathcal F^+ \to \mathcal G^+$ such that $\psi \circ \theta_{\mathcal F} = \theta_{\mathcal G} \circ \varphi$. We take this $\psi$ to be the definition of $\varphi^+$.
        \par We want to show that $\text{ker}(\varphi^+) = 0$. So, let $U$ be an open set of $X$, and let $s \in \text{ker}(\varphi^+)(U)$. By definition of the kernel, this means that $s$ is an element of $\mathcal F^+(U)$ such that $\varphi^+_U(s) = 0$. 
        \par Since $s$ is a section of the sheaf associated to the presheaf $\mathcal F$, it is a function
        \[
        U \to \bigcup_{P \in U} \mathcal F_P
        \]
	such that $s(P) \in \mathcal F_P$, and so that for each $P$ there is some open set $V_P$ containing $P$ such that $s\lvert_{V_P}$ is identically equal to an element $s_{V_P} \in \mathcal F(V_P)$. This is how the map $\theta_{\mathcal{F}}$ is defined - \textit{i.e.}, $s\lvert_{V_P} = \theta(s_{V_P})$. Since $\varphi^+$ is a morphism of sheaves, it commutes with restrictions, so we have $\varphi^+_U(s) = \varphi^+_{V_P}(s\lvert_{V_P})$. And, by the construction of $\varphi^+$, we have $\varphi^+ \circ \theta_{\mathcal{F}} = \theta_{\mathcal{G}} \circ \varphi$. So, for every $Q \in V_P $,
    \begin{align*}
	    0 &= \varphi^+_U(s)(Q)\\ &= \varphi^+_{V_P}(s\lvert_{V_P})(Q)\\
	    &= \varphi^+_{V_P}(\theta_{\mathcal{F}}(s_{V_P}))(Q)\\
	&= (\theta_{\mathcal{G}} \circ \varphi_{V_P}(s_{V_P}))(Q)\\
	&= \varphi_{V_P}(s_{V_P})(Q)
    \end{align*}
    So, the element $\varphi_{V_P}(s_{V_P})$ takes every $Q \in V_P$ to $0$, and $s_{V_P}$ is in the kernel of $\varphi_{V_P}$. By assumption, the induced map $\varphi_{V_P}$ is injective, so this means that $s_{V_P}$ must be $0$ as well, and so must be $\theta_{\mathcal{F}}(s_{V_P})$. Since the element $s$ restricts to the element $\theta_{\mathcal{F}}(s_{V_P}) = 0$ on each set of the open cover $V_P$, it must be equal to $0$ as well. So, every element of the kernel of $\varphi^{+}$ is $0$, and it is injective.
    \par 
    \end{proof}
    \begin{item}
	    Use part (a) to show that if $\varphi: \mathcal{F} \to \mathcal{G}$ is a morphism of sheaves, then $\text{im} \varphi$ can be naturally identified with a subsheaf of $\mathcal{G}$, as mentioned in the text.
	    \begin{proof}
		    We first note than any injective morphism of sheaves $\psi: \mathcal{A}\to \mathcal{B}$ naturally identifies $\mathcal{A}$ as a subsheaf of $\mathcal{B}$, because every $\psi_U: \mathcal{A}(U) \to \mathcal{B}(U)$ is also an injective morphism of abelian groups, which identifies $\mathcal{A}(U)$ as a subgroup of $\mathcal{B}(U)$. 
		    \par \textit{incomplete}
	    \end{proof}
    \end{item}
\end{enumerate}
\end{problem}
\begin{problem}{5}
	Show that a morphism of sheaves is an isomorphism if and only if it is both injective and surjective.
	\begin{proof}
		One direction is clear, so we will show that injectivity and surjectivity together give an isomorphism.
		\par Let $\varphi: \mathcal{F}\to \mathcal{G}$ be a morphism of sheaves which is both injective and surjective. We wish to construct a morphism $\psi: \mathcal{G}\to \mathcal{F}$ which is a two-sided inverse to $\varphi$. 
		\par Let $U \subset X$ be an open set, and let $s \in \mathcal{G}(U)$. Because $\varphi$ is surjective, there is some open cover $V_i$ of $U$, and elements $s_i \in \mathcal{G}(V_i)$, such that $s_i = \varphi(t_i)$ for some $t_i \in \mathcal{F}(V_i)$. Now, take two $V_i$ and $V_j$, and let $W = V_i \cap V_j$. The element $(t_i\lvert_{W}) - (t_j \lvert_{W})$ maps under $\varphi$ to 
		\begin{align*} \varphi_W((t_i\lvert_W) - (t_j\lvert_W))&= \varphi_W(t_i\lvert_W) - \varphi_W(t_j\lvert_W)\\
			&= (\varphi_{V_i}(t_i))\lvert_W - (\varphi_{V_j}(t_j))\lvert_W\\
			&= (s_i)\lvert_W - (s_j)\lvert_W\\
			&= (s\lvert_{V_i})\lvert_W - (s\lvert_{V_j})\lvert_W\\
			&= s\lvert_W - s\lvert_W\\
			&= 0
		\end{align}
		So, we see that $t_i\lvert_W - t_j \lvert_W = 0$, and the two elements $t_i$ and $t_j$ are equal on the intersection of their domains. This holds for any pair $t_i$, $t_j$, so by the fact that $\mathcal{F}$ is a sheaf, there is some element $t \in \mathcal{F}(U)$ such that $t\lvert_{V_i} = t_i$ for any $i$. 
		\par In fact, we see that $\varphi_U(t) = s$, i.e. that $\varphi_U(t) - s = 0$. Restricting to any set of the open cover $V_i$ of $U$, we have
		\begin{align*}
			(\varphi_U(t) - s)\lvert_{V_i} &= \varphi_U(t)\lvert_{V_i} - s\lvert_{V_i}\\
			&= \varphi_{V_i}(t\lvert_{V_i}) - s_i\\
			&= \varphi_{V_i}(t_i) - \varphi_{V_i}(t_i)\\
			&= 0
		\end{align*}
		So, $\varphi_U(t) - s$ is zero on every set $V_i$. Because $\mathcal{F}$ is a sheaf, it is therefore zero on the set $U$, and the two elements $\varphi_U(t)$ and $s$ must be equal. Finally, we see that $t$ must be unique, for if there are two elements $t_1, t_2$ such that $\varphi_U(t_1) = \varphi_U(t_2) = s$, then $t_1 - t_2$ would be in the kernel of $\varphi_U$, and so would be zero. So, for every $s \in \mathcal{G}(U)$, there is a unique $t\in \mathcal{F}(U)$ such that $\varphi_U(t) = s$.
		\par Let $\psi_U$ be the function which takes every element $s$ to the corresponding element $t$.
	\end{proof}
\end{problem}
\end{document}
