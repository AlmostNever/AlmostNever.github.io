% --------------------------------------------------------------
% Andrew Tindall
% --------------------------------------------------------------
 
\documentclass[12pt]{article}
 
\usepackage[margin=1in]{geometry} 
\usepackage{amsmath,amsthm,amssymb,enumitem}

\newcommand{\N}{\mathbb{N}}
\newcommand{\Q}{\mathbb{Q}}
\newcommand{\Z}{\mathbb{Z}}
\newcommand{\C}{\mathbb{C}}
\newcommand{\R}{\mathbb{R}}
\newcommand{\mc}[1]{\mathcal{#1}}
\newcommand{\e}{\varepsilon}
\newcommand{\bs}{\backslash}
\newcommand{\PGL}{\text{PGL}}
\newcommand{\Sp}{\text{Sp}}
\newcommand{\tr}{\text{tr}}
\newcommand{\Lie}{\text{Lie}}
\newcommand{\rec}[1]{\frac{1}{#1}}
\newcommand{\toinf}{\rightarrow \infty}


\theoremstyle{definition}
\newtheorem{proofpart}{Part}
\newtheorem{theorem}{Theorem}
\makeatletter
\@addtoreset{proofpart}{theorem}
\makeatother


\newenvironment{problem}[2][Problem]{\begin{trivlist}
\item[\hskip \labelsep {\bfseries #1}\hskip \labelsep {\bfseries #2.}]}{\end{trivlist}}
 
\begin{document}
 
%\renewcommand{\qedsymbol}{\filledbox}
 
\title{Homework 2}
\author{Andrew Tindall\\
Topology II}
 
\maketitle
\begin{problem}{1}
Write down all of the data producing a finite $CW$ complex structure $X = \bigcup_n X_n$ for $\Sigma_2$. 
\begin{proof}
    We start with one $0$-cell $e^0$, four $1$-cells $e_1^1, e_2^1, e_3^1, e_4^1$, and one $2$-cell $e^2$.
    \par Our $0$-Skeleton $X^0$ is the one-point space $e^0 = *$.
    \par Next, the $1$-skeleton $X_1$ is, setwise, the union 
    \[ X_1 = \left ( e_1^1 \amalg e_2^1 \amalg e_3^1 \amalg e_4^1\right ) \amalg X_0,\]
    \par Combined with gluing maps $\Phi^1_i: \partial D^1 \to X_0$ for each $e^1_i$. The boundary $\partial D^1$ can be identified with $S^0 = \left\{ -1, 1 \right\}$, the $0$-sphere, which consists of only $2$ points. The $0$-skeleton has only one point, which we call $*$, so each boundary map $\Phi^1_i$ here is the trivial map which takes both $-1$ and $1$ to $*$.
    \par Now, the $2$-skeleton $X_2$ is the union
    \[X_2 = e^2 \amalg X_1 \]
    Combined with a gluing map $\Phi^2: \partial D^2 \to X_1$. The boundary $\partial D^2$ of the $2$-cell can be identified with the $1$-sphere $S^1$: the set of points $\{ (x,y) ; x^2 + y^2 = 1\}$. Reparameterizing this set as 
    \[
    \partial D^2 = \{ (\cos(\theta), \sin(\theta); 0 \leq \theta < 2\pi\}
    \]
    We can then construct a map from this set to the $1$-skeleton $X_1 = \coprod_i e^1_i \amalg e^0$. 
    \par Let $\pi_i(x): I^\circ \to e^1_i$ be the function taking $0 < x < 1$ to its image in the $1$-cell $e_i^1$, and let $*(x)$ be the constant function $x \mapsto * \in e^0$.
    \begin{align*}
	    \Phi( (\cos(\theta), \sin(\theta))) = \begin{cases}
		    \pi_1(4\theta / \pi) & 0 < \theta < \pi / 4\\
		    \pi_2(4(\theta - \pi / 4) / \pi) & \pi / 4 < \theta < \pi / 2\\
		    \pi_1(1 - 4(\theta - \pi / 2) / \pi) & \pi / 2 < \theta < 3 \pi / 4\\
		    \pi_2(1 - 4(\theta - 3 \pi / 4) / \pi) & 3 \pi / 4 < \theta < \pi\\
		    \pi_3(4(\theta - \pi) / \pi) & \pi < \theta < 5\pi / 4\\
		    \pi_4(4(\theta - 5 \pi / 4) \pi) & 5 \pi / 4 < \theta < 3 \pi  / 2\\
		    \pi_3(1 - 4(\theta - 3 \pi / 2)/ \pi) & 3 \pi / 2 < \theta < 7 \pi /4\\
		    \pi_4(1 - 4(\theta - 7 \pi / 4) / \pi) & 7 \pi / 4 < \theta < 2\pi\\
		    *(x) & x \equiv 0 \text{ mod } (\pi / 4)
	    \end{cases}
    \end{align*}
    This is a continuous map: it is a linear function in the parameter $\theta$ to the interval $e_i^1 \cong I$ on each domain $n\pi / 4 < \theta < (n+1)\pi / 4 $, and its value at the boundaries of these intervals is $* \in e^0$, which has been identified with the boundaries of each $e^1_i$ via the gluing maps $\Phi^1_i$.
\end{proof}
\end{problem}
\begin{problem}{2}
	Using the presentation of the M\"obius band $M$ as $I \times I / \sim$, where $\sim$ is the equivalence relation generated by 
	\[ (x, 0) \sim (1 - x, 1), \text{ for $x \in I$},\]
	write down a deformation retract from $M$ to the circle $S^1$.
	\begin{proof}
		Let $r : M \times I \to I$ be the map taking $(x, y, a)$ to $(a (x - 1/2) + 1/2, y)$ - that is, leaving $y$ fixed, and taking $x $ to $a \cdot (x - 1/2) + 1/2$. For $a = 0$, this is the identity map. We also see that it is well-defined for other values of $a$, as it is constant on equivalence classes of $\sim$ - the only thing we need to check is that $r(x,0, a) = r(1-x, 1, a)$:
		\begin{align*}
			r(x,0,a) = (a(x - 1/2) + 1/2, 0)\\
			&= (ax - a/2 + 1/2, 0)\\
			&= (1 - (ax - a/2 + 1/2), 1)\\
			&= (a (-x + 1/2) + 1/2, 1)\\
			&= (a ( (1 - x) - 1/2 ) + 1/2, 1)\\
			&= r(1-x,1,a)
		\end{align*}
		Therefore, $r$ is a well-defined function $M \times I \to I$. And, for $a = 1$, we see that $r$ is a retraction of $M$ to the subspace $\left\{ (1/2, y) ; y \in I \right\} / \sim*$, where $\sim*$ is the restriction of $\sim$ to the space $x = 1/2$ - the only nontrivial equivalence being $(1/2, 0) \sim* (1/2,1)$. This space is homeomorphic to $S^1$, and $r(\cdot, \cdot, 1)$ is constant on it, making it a retract of $M$ to $S^1$. 
		\par Finally, because $r$ is defined simply by multiplication in $x, y$, and $a$, it is continuous in all of these variables, except possibly at the glued edge of the mobius strip, $y = 0, 1$. However, it is continuous here as well - the inverse image of an open neighborhood of an element $[(x, 1)]$ of this edge is a stretched (by a factor of $1/a$) open neighborhood of $[(x/a, 1)]$.
		\par As a continuous function $M \times I \to M$, which is equal to $\text{Id}_M$ on $M \times \left\{ 0 \right\}$, and to a retract to $S^1$ on $M \times \left\{ 1 \right\}$, this is a deformation retract of the m\"obius band to the circle.
	\end{proof}
\end{problem}
\end{document}
