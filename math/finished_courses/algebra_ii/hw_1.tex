% --------------------------------------------------------------
% Andrew Tindall
% --------------------------------------------------------------
 
\documentclass[12pt]{article}
 
\usepackage[margin=1in]{geometry} 
\usepackage{amsmath,amsthm,amssymb,enumitem,hyperref}

\newcommand{\N}{\mathbb{N}}
\newcommand{\Q}{\mathbb{Q}}
\newcommand{\Z}{\mathbb{Z}}
\newcommand{\C}{\mathbb{C}}
\newcommand{\R}{\mathbb{R}}
\newcommand{\mc}[1]{\mathcal{#1}}
\newcommand{\e}{\varepsilon}
\newcommand{\bs}{\backslash}
\newcommand{\PGL}{\text{PGL}}
\newcommand{\Sp}{\text{Sp}}
\newcommand{\tr}{\text{tr}}
\newcommand{\Lie}{\text{Lie}}
\newcommand{\rec}[1]{\frac{1}{#1}}
\newcommand{\toinf}{\rightarrow \infty}


\theoremstyle{definition}
\newtheorem{proofpart}{Part}
\newtheorem{theorem}{Theorem}
\makeatletter
\@addtoreset{proofpart}{theorem}
\makeatother


\newenvironment{problem}[2][Problem]{\begin{trivlist}
\item[\hskip \labelsep {\bfseries #1}\hskip \labelsep {\bfseries #2.}]}{\end{trivlist}}
 
\begin{document}
 
%\renewcommand{\qedsymbol}{\filledbox}
 
\title{Homework1}
\author{Andrew Tindall\\
Algebra II}
 
\maketitle
\begin{section}{Problems}

	\begin{problem}{1}
		Let $R$ be a commutative ring, and let $V$ and $W$ be $R$-modules. Consider the abelian group $\text{Hom}_R(V,W)$. \\
		(a) Show that the $R$-action on $V$ endows $\text{Hom}_R(V,W)$ with the structure of an $R$-module.\\
		(b) Show that the $R$-action on $W$ endows $\text{Hom}_R(V,W)$ with the structure of an $R$-module.\\
		(c) Show that these two $R$-module structures are identical.
	\end{problem}
	\begin{proof}
		$(a)$ and $(b)$ are very similar. First, let $\rho(r,v)$ be the left action of $R$ on $V$, and denote this by $r \cdot b$, and let the action $\sigma(r, w)$ on $W$ also (by abuse of notation) be denoted by $r \cdot w$.
		\par  We first form the action $\theta(r, f) = r \diamond f$ of $R$ on $\text{Hom}_R(V,W)$. Let $f$ be a member of the hom set, and define $r \diamond f$ as the function taking $v \in V$ to the value $f(r \cdot v)$. We first verify that this is a legitimate homomorphism:
		\begin{itemize}
			\item $(r \diamond f)(v + u) = (r \diamond f)(v) + (r \diamond f)(u)$:
				\begin{align*}
					(r \diamond f)(v + u) &= f(r \cdot (v + u))\\
					&= f(r \cdot v + r \cdot u)\\
					&= f(r \cdot v) + f(r \cdot u)\\
					&= (r \diamond f)(v) + (r \diamond f)(u)
				\end{align*}
			\item $(r \diamond f)(s \cdot v) = s \cdot (r \diamond f)(v)$:
				\begin{align*}
					(r \diamond f)(s \cdot v) &= f(r \cdot (s \cdot v))\\
					&= f( (rs) \cdot v)\\
					&= f( (sr) \cdot v)\\
					&= f( s \cdot (r \cdot v))\\
					&= s \cdot (f (r \cdot v))\\
					&= s \cdot (r \diamond f)(v)
				\end{align*}
		\end{itemize}
		Thus $\diamond$ is a legitimate function from $R \times \text{Hom}(V,W)$ to $\text{Hom}(V,W)$. Now we verify that $\diamond$ satisfies the axioms of a left $R$-action. Let $v$ be an arbitrary element of $V$. Here and in the rest of the solutions, we are implicitly using extensionality to show that two functions are equal; if they take an arbitrary element $v$ to equal elements, then they are equal as functions.
		\begin{itemize}
			\item $r \diamond (f + g) = r \diamond f + r \diamond g$ :     
				\begin{align*}
					(r \diamond (f + g))(v) &= (f+ g)(r \cdot v)\\
					&= f(r \cdot v) + g(r \cdot v)\\
					&= (r \diamond f)(v) + (r \diamond g)(v)\\
					&= (r \diamond f + r \diamond g)(v)
				\end{align*}
			\item $(r + s)\diamond f = r \diamond f + s \diamond f$
				\begin{align*}
					((r+ s)\diamond f)(v) &= f(( r + s) \cdot v)\\
					&= f( r \cdot v + s \cdot v) \\
					&= f( r \cdot v) + f( s \cdot v)\\
					&= (r \diamond f)(v) + (s \diamond f)(v)\\
					&= (r \diamond f + s \diamond f)(v)
				\end{align*}
			\item $(rs) \diamond f = r \diamond (s \diamond f)$ 
				\begin{align*}
					((rs) \diamond f )(v) &= f( (rs) \cdot v)\\
					&= f( (sr) \cdot v)\\
					&= f ( s \cdot (r \cdot v))\\
					&= (s \diamond f)(r \cdot v)\\
					&= (r \diamond (s \diamond f))(v)
				\end{align*}
			\item $1\diamond f = f$:
				\begin{align*}
					(1 \diamond f)(v) &= f(1 \cdot  v)\\
					&= f(v)
				\end{align*}
		\end{itemize}
		Thus $\text{Hom}_R(V,W)$ is a left $R$-module.
		\par Now, let the left action $\phi(r, g) = r \star g$ be defined as $(r \star g)(v) = r \cdot (g(v))$. We verify the same axioms for this function to be a homomorphism:
		\begin{itemize}
			\item $(r \star f)(v + u) = (r \star f)(v) + (r \star f)(u)$:
				\begin{align*}
					(r \star f)(v + u) &= r \cdot (f(v + u))\\
					&= r \cdot (f(v) + (f(u)))\\
					&= r \cdot f(v) + r \cdot f(u)\\
					&= (r \star f)(v) + (r \star f)(u)
				\end{align*}
			\item $(r \star f)(s \cdot v) = s \cdot (r \star f)(v)$:
				\begin{align*}
					(r \star f)(s \cdot v) &= f(r \cdot (s \cdot v))\\
					&= f( (rs) \cdot v)\\
					&= f( (sr) \cdot v)\\
					&= f(s \cdot (r \cdot v))\\
					&= s \cdot f(r \cdot v)\\
					&= s \cdot (r \cdot f(v))\\
				\end{align*}
		\end{itemize}
		We also verify the $R$-action axioms. Let $v$ be an arbitrary element of $V$:
		\begin{itemize}
			\item $r  \star (f+g) = r \star f + r\star g$:
				\begin{align*}
					(r \star (f+g))(v) &= r \cdot ((f+g)(v))\\
					&= r \cdot (f(v) + g(v))\\
					&= r\cdot f(v) + r\cdot g(v)\\
					&= (r \star f)(v) + (r \star g)(v)\\
					&= (r \star f + r \star g)(v)
				\end{align*}
			\item $(r + s) \star f = r \star f + s \star f$:
				\begin{align*}
					((r + s) \star f)(v) &= (r + s) \cdot (f(v))\\
					&= r \cdot f(v) + s \cdot f(v)\\
					&= (r \star f)(v) + (s \star f)(v)\\
					&= (r \star f + s \star f)(v)
				\end{align*}
			\item $(rs) \star f = r \star (s \star f)$:
				\begin{align*}
					((rs) \star f)(v) &= (rs)\cdot (f(v))\\
					&= r \cdot (s \cdot (f(v)))\\
					&= r \cdot ((s \star f)(v))\\
					&= (r \star (s \star f))(v)
				\end{align*}
			\item $1 \star f = f$:
				\begin{align*}
					(1 \star f)(v) &= 1 \cdot (f(v))\\
					&= f(v)
				\end{align*}
		\end{itemize}
		Thus this action makes $\text{Hom}_R(V,W)$ into a left $R$-module.
		\par Now we show that $r \star f = f \diamond r$. 
		\begin{align*}
			(r \star f)(v) &= r \cdot (f(v))\\
			&= f(r \cdot v) \\
			&= (r \diamond f)(v)
		\end{align*}
		So, for a commutative ring $R$, there is a natural $R$-module structure on the hom-sets $\text{Hom}(V,W)$ of the category $R$-mod, where the $R$-module structure may come from either $V$ or $W$. This gives us an internal exponential object $W^V$ in $R$-mod. In fact, $R$-mod is Cartesian closed, since we also have a terminal object (the $0$ object) and all finite products.
	\end{proof}
	\begin{problem}{2}
		Let $R$ be a commutative ring, and let $V$ be an $R$-module. Prove that $\text{End}_R(V)$ is an $R$-algebra, where the multiplication operation is composition.
	\end{problem}
	\begin{proof}
		We show first that $\text{End}_R(V)$ is a ring. It is immediate that it is an abelian group, as it is the hom-object $\text{Hom}_R(V,V)$ in an abelian category. Taking composition as the multiplication operation, associativity and identity are also immediate, as they are implied by the fact that $R$-mod is a category. So we need only check the two distributivity properties. Let $f$, $g$, and $h$ be arbitrary endomoprhisms of $V$, and let $v$ be an arbitrary element of $v$.
		\begin{itemize}
			\item $f \circ (g + h) = f \circ g + f \circ h$:
				\begin{align*}
					(f \circ (g + h))(v) &= f( (g + h)(v))\\
					&= f(g(v) + h(v))\\
					&= (f \circ g)(v) + (f \circ h)(v)\\
					&= (f \circ g + f \circ h)(v)
				\end{align*}
			\item $(f + g) \circ h = f \circ h + g \circ h$:
				\begin{align*}
					( (f + g) \circ h)(v) &= (f+g)(h(v))\\
					&= f( h(v)) + g(h(v))\\
					&= (f \circ h)(v) + (g \circ h)(v)
				\end{align*}
		\end{itemize}
		So $\text{End}_R(V)$ is a valid ring. For it to be an $R$-algebra, we also need to exhibit a ring homomorphism $R \to \text{End}_R(V) $, and show that this homomorphism maps $R$ into the center of $\text{End}_R(V)$. 
		\par Let the function $\phi$ map $r \in R$ to the function $\phi(r)$ taking $v \in V$ to $r \cdot V$. We need to show that this function $\phi(r)$ is a homomorphism, so that the image of $\phi$ is in $\text{End}_R(V)$:
		\begin{itemize}
			\item $\phi(r)(v + u) = \phi(r)(v) + \phi(r)(u)$:
				\begin{align*}
					\phi(r)(v+u) &= r \cdot (v + u)\\
					&= r \cdot v + r \cdot u\\
					&= \phi(r)(v) + \phi(r)(u)
				\end{align*}
			\item $\phi(r)(s \cdot v) = s \cdot \phi(r)(v)$:
				\begin{align*}
					\phi(r)(s \cdot v) &= r \cdot (s \cdot v)\\
					&= (rs) \cdot v\\
					&= (sr) \cdot v\\
					&= s \cdot (r \cdot v)\\
					&= s \cdot \phi(r)(v)
				\end{align*}
		\end{itemize}
		Therefore our map $\phi$ is a valid set-map from $R$ to $\text{End}_R(V)$. We now need to show that it is a homomorphism of rings:
		\begin{itemize}
			\item $\phi(r + s) = \phi(r) + \phi(s)$: 
				\begin{align*}
					\phi(r+s)(v) &= (r + s) \cdot v\\
					&= r \cdot v + s \cdot v\\
					&= \phi(r)(v) + \phi(s)(v)\\
					&= (\phi(r) + \phi(s))(v)
				\end{align*}
			\item $\phi(rs) = \phi(r) \circ \phi(s)$:
				\begin{align*}
					\phi(rs)(v) &= (rs) \cdot v\\
					&= r \cdot (s \cdot v)\\
					&= \phi(r)(s \cdot v)\\
					&= \phi(r)(\phi(s)(v))\\
					&= (\phi(r) \circ \phi(s))(v)
				\end{align*}
			\item $\phi(1) = \text{Id}$:
				\begin{align*}
					\phi(1)(v) &= 1 \cdot v\\
					&= v\\
					&= \text{Id}(v)
				\end{align*}
		\end{itemize}
		Finally, we need to prove that the image of the map $\phi$ is contained within the center of $R$, i.e. that given any element $r \in R$ and any endomorphism $f \in \text{End}_R(V)$, $\phi(r)$ commutes with $f$. 
		\begin{itemize}
			\item $\phi(r) \circ f = f \circ \phi(r)$:
				\begin{align*}
					(\phi(r) \circ f)(v) &= \phi(r)(f(v))\\
					&= r \cdot f(v)\\
					&= f(r \cdot v)\\
					&= f(\phi(r)(v))\\
					&= (f \circ \phi(r))(v)
				\end{align*}
		\end{itemize}
		So, indeed, $\text{End}_R(V)$ is an $R$-algebra. It has some interesting sub-algebras; for instance (I know this holds for a vector space, but I think that the scalars can come from any commutative ring), the sub-algebra $R[A]$ generated by a single endomorphism $A$ is ismorphic to the quotient $R[x]/(p(x))$ of the free $R$-algebra over one variable by the ideal generated by the minimal polynomial of $A$. 
	\end{proof}
\end{section}
\begin{section}{Optional/Practice Problems}
	\begin{problem}{1}
		Prove that $\Z$ is the initial object in the category of rings, based on the definition of $\Z$ as the unique ordered ring where the positive elements are well-ordered.
	\end{problem}
	\begin{proof}
		I won't write a complete proof of this, but I've thought a bit about how to prove it. I would start by ``reinventing'' the integers within an arbitrary ring, as the subring generated by the identity ($1$, $1 + 1$, $1 + 1 + 1$, etc.) and show that given the defining properties of $\Z$,
		\begin{itemize}
			\item Every element of $\Z$ is contained within this subring, and
			\item Within $\Z$, any two elements in this ring are equal only if they must be ($1 = 1, 1 + 1 = 1 + 1$, etc.)
		\end{itemize}
		This is easier to do with some abuse of notation, where we might write $1 + 1 + 1$ as $3 \cdot 1$, even though we don't ``know'' what $3$ is. The proof depends on the fact that $0 < 1$, and that no other element of $\Z$ may lie between $0 $ and $1$, as if some element $a$ did, the set $\{a, a^2, a^3, \dots\}$ would be infinite and decreasing, without containing its own lower bound, contradicting well-orderedness.
		\par This totally determines the elements of $\Z$, and we can quickly recover the status of $\Z$ as an initial object, since every ring contains the subring generated by the identity, and any map out of $\Z$ is totally determined by its value on the identity, which is set in stone. 
	\end{proof}
	\begin{problem}{2a}
		Prove that $n$ divides the characteristic of a ring $R$ if there exists a ring homomorphism $\Z/n\Z \to R$, and that the converse is true if $n \in \Z_{\geq 1}$.
	\end{problem}
	\begin{proof}
		Let $\phi$ be some ring homomorphism $\Z/n\Z \to R$. Then \[ 0 = \phi(0) = \phi(\underbrace{1 + 1 + \dots + 1}_{\text{n times}}) = \underbrace{\phi(1) + \phi(1) + \dots + \phi(1)}_{\text{n times}}.\]. This shows that $n \cdot 1 = 0$; to finish the first half of the proof we need to show that if $n \cdot 1 = 0$ in any ring, than $n$ is a multiple of the characteristic, which is a short proof - the characteristic $k$ must be less than or equal to $n$, so we see that \[0 = \underbrace{1 + 1 + \dots + 1}_{k\text{ times}} + \underbrace{1 + 1 + \dots + 1}_{n - k\text{ times}} = 0 + \underbrace{1 + 1 + \dots + 1}_{n - k \text{times}}.\]
		Inductively, we get a series of decreasing natural numbers $n, n-k, n-2k$, which must terminate somewhere between $0$ and $k$. It cannot go lower than $k$ and higher than 0 without contradicting minimality of $k$, so we must have that $n - j \cdot k = 0$ for some for some $j$, and thus that $n$ is a multiple of the characteristic of $R$.
		\par The reason that the opposite is true is that we may lower the unique map $\phi$ from $\Z$ into $R$ to a map from $\Z/n\Z$ to $R$, by the universal property of the quotient construction - the only necessary lemma is that $\phi$ is zero on the subring $n\Z$, which is relatively simple.
	\end{proof}
	\begin{problem}{2b}
		Show that the characteristic of a field $k$ is either $0$ or a prime number.
		
	\end{problem}
	\begin{proof}
		It is immediate from the field axiom $1 \neq 0$ that the characteristic is not $1$, so assume for contradiction that the characteristic of the field is a composite number, $m \cdot n$ for $m, n \geq 2$. Then by minimality of the characteristic, $m \cdot 1 \neq 0$ and $n \cdot 1 \neq 0$, but $(m \cdot 1) \cdot (n \cdot 1) = 0$, which is impossible in a field.
	\end{proof}
	\begin{problem}{3}
		Show that any homomorphism of a field into a ring is injective.
	\end{problem}
	\begin{proof}
		A homomorphism of rings may be decomposed into a quotient by a kernel and an injection into the target. A kernel is an ideal, and fields have very few ideals, so the quotient must be by the trivial ideal $0$, making the map injective. (It cannot be the quotient by the field itself, because $1 \neq 0$ in the rings we are working with? Otherwise the zero homomorphism would be valid.)
	\end{proof}
	\begin{problem}{4}
		Show that if $V$ is an $(R,S)$-bimodule and $W$ is an $(R,T)$-bimodule, then $\text{Hom}_R(V,W)$ is an $(S,T)$-bimodule.	
	\end{problem}
	\begin{proof}
		This proof involves checking a few axioms, much like problem 1, so I won't run through the verification. We note that $V$ is a right $S$-module, but $\text{Hom}_R(V,W)$ is a left $S$-module. This is an interesting result in noncommutative algebra, but it is also useful in the commutative case. One of the important uses for this bimodule-Hom structure is that it allows us to define a tensor-hom adjunction between arbitrary rings: let $S$ and $R$ be commutative rings, and let $M$ an $R$-module, $N$ be an $(R,S)$-bimodule, and $P$ an $S$-module. (In the commutative case, an $(R,S)$-bimodule is the same as an $(S,R)$-bimodule.) Then $\text{Hom}_S(N,P)$ is both an $S $- and $R$-module, and we have a canonical isomorphism\[\text{Hom}_R(M, \text{Hom}_S(N,P)) \cong \text{Hom}_S(M \otimes_R N, P).\] This is an adjunction $F \vdash G$, where $F: R\text{-Mod} \Rightarrow S\text{-Mod}$ is the functor $- \otimes_R N$, and $G: S\text{-Mod} \Rightarrow R\text{-Mod}$ is the functor $\text{Hom}_S(N,-)$ This of course reduces to the tensor-hom adjunction $-\otimes_R N \vdash \text{Hom}_R(N,-)$ between endofunctors of $R$-mod in the case that $R = S$. 
	\end{proof}
\end{section}
\end{document}
