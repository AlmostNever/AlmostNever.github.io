% --------------------------------------------------------------
% Andrew Tindall
% --------------------------------------------------------------
 
\documentclass[12pt]{article}
 
\usepackage[margin=1in]{geometry} 
\usepackage{amsmath,amsthm,amssymb,enumitem,hyperref,tikz-cd}

\newcommand{\N}{\mathbb{N}}
\newcommand{\Q}{\mathbb{Q}}
\newcommand{\Z}{\mathbb{Z}}
\newcommand{\C}{\mathbb{C}}
\newcommand{\R}{\mathbb{R}}
\newcommand{\mc}[1]{\mathcal{#1}}
\newcommand{\e}{\varepsilon}
\newcommand{\bs}{\backslash}
\newcommand{\PGL}{\text{PGL}}
\newcommand{\Sp}{\text{Sp}}
\newcommand{\tr}{\text{tr}}
\newcommand{\Lie}{\text{Lie}}
\newcommand{\rec}[1]{\frac{1}{#1}}
\newcommand{\toinf}{\rightarrow \infty}


\theoremstyle{definition}
\newtheorem{proofpart}{Part}
\newtheorem{theorem}{Theorem}
\makeatletter
\@addtoreset{proofpart}{theorem}
\makeatother


\newenvironment{problem}[2][Problem]{\begin{trivlist}
\item[\hskip \labelsep {\bfseries #1}\hskip \labelsep {\bfseries #2.}]}{\end{trivlist}}
 
\begin{document}
 
%\renewcommand{\qedsymbol}{\filledbox}
 
\title{Homework 3}
\author{Andrew Tindall\\
Algebraic Geometry}
 
\maketitle
\begin{problem}{1}
    Let $X = V(f)\subset \mathbb A^n$ be an affine variety defined by a single polynomial $f$. Prove that $\text{dim}(X) = n-1$.
    \begin{proof}
        We may assume that $f$ is nonconstant, so that $(f)$ is a nontrivial and nonzero ideal - otherwise the variety $X$ would also be trivial. 
        We wish to show that $\text{dim}(X) = n-1$. 
        Because $X$ is an affine variety, this is equivalent to showing that $\text{dim}_{\text{Kr}}(k(X))= n-1$: the Krull dimension of the coordinate ring of $X$ is equal to $n-1$.
        First, assume that $X$ is irreducible as a variety; equivalently, $(f)$ is an irreducible ideal. <++>. 

    \end{proof}
\end{problem}
\begin{problem}(2)
    Let $X \subset \mathbb A^n$ be an algebraic set with $\overline X$ its closure in $\mathbb P^n$. Show that $X$ is irreducible if and only if $\overline X$ is irreducible. 
    \begin{proof}
        <++>
    \end{proof}
\end{problem}
\begin{problem}(3)
    Let $A = \bigoplus_{i \geq 0} A_i$ be a graded ring. Recall that an ideal $I\subset A$ is homogeneous if it is generated by homogeneous elements. 
    \begin{enumerate}[label=(\alph*)]
        \item Show that if $I$ is homogeneous then the following holds: let $f=\sum_i f_i \in A$, where $f_i$ denotes the homogenous degree $i$ component of $f$. Then $f \in I$ if and only if $f_i \in I$, for all $i$.
            \begin{proof}
                One direction is clear; for if every component of $f$ is in $I$, then so is $f$, because $I$ is closed under addition. 
                \par In the opposite direction, assume $f \in I$. Since $I$ is homogeneous, it is generated by some set $\left\{ g_j \right\}$ of homogenous polynomials, and $f$ may be written as a finite sum of terms in the $g_j$:
                \[f = \sum_{j \in J} a_j g_j\]
                Denote by $d_j$ the degree of the homogenous term $g_j$. Now, we can write each term $a_j$ as a sum of homogeneous terms; say
                \[a_j = \sum_{k \geq 0} a_{jk}\]
                \par Using these decompositions, we may write each homogenous component of $f$ as\[ f_i = \sum_{j \in J} a_{j(i -d_k)} g_j,\]
                \par So indeed we may write each homogenous component of $f$ in terms of the homogenous generators of $I$, and so each homogenous component of $f$ is in $I$.
            \end{proof}
        \item Show that if $f$ is homogenous then its radical $\sqrt(I)$ is also homogenous.
             \begin{proof}
                 <++>
             \end{proof}
    \end{enumerate}
\end{problem}
\begin{problem}{4}
    Consider the $n$-tuple morphism $\varphi: \mathbb P^1 \to \mathbb P^n$ given by $(t:s) \mapsto (t^n : st^{n-1}: \cdots : s^n)$ Prove that the image of $\varphi$ is a projective subvariety $C_n$ of $\mathbb P^n$.
    \begin{proof}
        Let $C_n = V(\{f_{ij}\}_{0 <= i < j <= n})$, where $f_{ij} = x^ix^j - x^{i+1}x^{j-1}$. We first see that the image of $\varphi$ lies within $C_n$, since for any point $(t^n :st^{n-1}: \cdots : s^n)$, and any $i,j$,
        \begin{align*}
            f_{ij}(t^n:st^{n-1}:\cdots:s^n)&= (s^it^{n-i})(s^jt^tn-j) - (s^{i+1}t^{n - (i+1)})(s^{j-1}t^{n - (j - 1)})\\
            &= s^{i+j}t^{2n - (i + j)} - s^{i+j}t^{2n - (i + j)}\\
            &= 0
        \end{align*}
        It remains to show that, for any point $x \in C_n$, there exists some $(s:t)$ such that $\varphi(s:t) = x$. Let $x$ be an arbitrary point of $\mathbb P^n$ such that each $f_{ij}(x) = 0$. 
        \par Assume first that $x_0 \neq 0$, and rescale so that
        \[x = (1: x_1 : \cdots : x_n).\]
        \par We will show that the point $(1:x_1)$ maps to $x$ under $\varphi$. We see first that 
        \[\varphi(1:x_1) = (1: x_1 :x_1^2 : \cdots : x_1^n)\]
        \par Now, using the relation $f_{03}(x) = 0$, we see that $x_2 = x_1^2$; using $f_{04}(x)=0$, we see that $x_3 = x_1^3$, and so on. So indeed 
        \begin{align*}
            x &= (1:x_1:x_1^2:\cdots :x_1^n)\\
            &= \varphi(1:x_1)
        \end{align*}
        Next, assume that $x_0 = 0$; we will see that this forces $x_1, \dots , x_{n-1} = 0$, and so that $x_n \neq 0$. If $x_0 = 0$, then the relation $f_{02}(x)=0$ forces $x_1^2 = 0$, so $x_1=0$; that $f_{03}(x)=0$ forces $x_2= 0$, and so on through $x_{n-1}$. Since $x$ is a point of projective space, we know that not all of its components are $0$; therefore $x_n = 1$ (up to scaling), and
        \begin{align*}
            x &= (0:0 : \cdots : 1)\\
            &= \varphi(0:1)
        \end{align*}
        So in each of the cases $x_0 = 0$ and $x_0 \neq 0$, we have $x \in \text{Im}(\varphi)$, and so $\text{Im}(\varphi)$ is indeed the algebraic set $C_n$.
    \end{proof}
\end{problem}
\begin{problem}{5}
    Consider the hyperbola $X = V(xy - 1)$ and the parabola $Y=V(x^2 - y)$ as affine varieties in $\mathbb A^2$. Let $\mathbb P^2$ be the projective plane with coordinates $(x:y:z)$, and $\mathbb A^2 = \left\{ (x:y:z) \mid z \neq 0 \right\}$. Let $\overline X$ and $\overline Y$ denote the closure of $X$ and $Y$ in $\mathbb P^2$, respectively.
    \begin{enumerate}[label=(\alph*)]
        \item Find homogeneous ideals defining $\overline X$ and $\overline Y$ in $\mathbb P^2$.
            \begin{proof}
                The ideals are simply generated by the homogenization of the polynomials generating the ideals for $X$ and $Y$:
                \begin{align*}
                    \overline X &= V(xy - z^2)\\
                    \overline Y &= V(x^2 - yz)
                \end{align*}
            \end{proof}
        \item How many points of intersection do $\overline X$ and $\overline Y$ have?
            \begin{proof}
                One way to calculate this is to first solve the equations over $\mathbb A^2$, i.e. assuming that $z = 1$ and solving the original equations, and then checking for extra points ``at infinity'', where $z=0$.
                \par First, we have the thre points of intersection between the two curves in $\mathbb A^2$: setting $x = 1/y$ and solving $x^2 - y = 0$, we see that 
                \[ y^3 = 1,\]
                which has $3$ solutions over an algebraically closed field $k$, which we denote $1, \zeta_3, \zeta_3^2$. Solving for $x = 1/y$, we get the three points
                \[(1:1:1), (\zeta_3, \zeta_3^2, 1), (\zeta_3^2, \zeta_3, 1)\]
                \par Finally, we check for interesting solutions when $z = 0$:
                \[xy - 0 = x^2 - 0 = 0\]
                \par Of course we have $x = y= z = 0$, but this does not correspond to a point in projective space. The second equation forces $x = z =0$, which is a single point in projective space that does indeed satisfy both equations:
                \[(0:1:0)\]
                Therefore there are $4$ points of intersection between the two curves.
            \end{proof}
        \item Find an automorphism $\varphi$ of the projective plane $\mathbb P^2$ which maps $\overline X$ to $\overline Y$ and vice versa.
            \begin{proof}
                We show that the automorphism
                \[\varphi(x:y:z) = (z:y:x)\]
                maps the two sets to each other. 
                \par First. we see that the morphism is an automorphism. because it is its own inverse. Next, we see that if  satisfies the an element $a \in \mathbb P^2$ satisfies the 
            \end{proof}
    \end{enumerate}
\end{problem}
\begin{problem}{6}
    Let $X = V(y-x^2, z - x^3) \subset \mathbb A^3$. Prove the following:
    \begin{enumerate}[label=(\alph*)]
        \item $I = I(X)= \langle y - x^2, z -  x^3\rangle $.
            \begin{proof}
                By the fact that $I(V(J)) = \sqrt{J}$ for any ideal $J$, we wish to show that 
                \[\langle  y - x^2, z - x^3\rangle = \sqrt{\left \langle y - x^2, z - x^3 \right\rangle }.\]
                \par So, we want to show that $I$ is radical. It will suffice to show the stronger condition that this ideal is prime.<++>
            \end{proof}
        \item <++>
    \end{enumerate}
\end{problem}
\begin{problem}{7}
    <++>
\end{problem}<++>
\end{document}
