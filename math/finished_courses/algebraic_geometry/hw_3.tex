% --------------------------------------------------------------
% Andrew Tindall
% --------------------------------------------------------------
 
\documentclass[12pt]{article}
 
\usepackage[margin=1in]{geometry} 
\usepackage{amsmath,amsthm,amssymb,enumitem,hyperref,tikz-cd}

\newcommand{\N}{\mathbb{N}}
\newcommand{\Q}{\mathbb{Q}}
\newcommand{\Z}{\mathbb{Z}}
\newcommand{\C}{\mathbb{C}}
\newcommand{\R}{\mathbb{R}}
\newcommand{\mc}[1]{\mathcal{#1}}
\newcommand{\e}{\varepsilon}
\newcommand{\bs}{\backslash}
\newcommand{\PGL}{\text{PGL}}
\newcommand{\Sp}{\text{Sp}}
\newcommand{\tr}{\text{tr}}
\newcommand{\Lie}{\text{Lie}}
\newcommand{\rec}[1]{\frac{1}{#1}}
\newcommand{\toinf}{\rightarrow \infty}


\theoremstyle{definition}
\newtheorem{proofpart}{Part}
\newtheorem{theorem}{Theorem}
\makeatletter
\@addtoreset{proofpart}{theorem}
\makeatother


\newenvironment{problem}[2][Problem]{\begin{trivlist}
\item[\hskip \labelsep {\bfseries #1}\hskip \labelsep {\bfseries #2.}]}{\end{trivlist}}
 
\begin{document}
 
%\renewcommand{\qedsymbol}{\filledbox}
 
\title{Homework 3}
\author{Andrew Tindall
\\ Algebraic Geometry}
 
\maketitle
\begin{problem}{1}
	Let $X$ be a (quasi-projective) algenraic variety and let $p \in X$. Show that $\mathcal{O}_{p,X}$ is an integral domain if and only if $p$ belongs to a unique irreducible component of $X$. (hint: first reduce the problem to the case where $X$ is an affine variety).
	\begin{proof}
		First, assume that $X = \text{Spec}(A)$ is an affine variety, so that $\mathcal O_{X}= A$, and $\mathcal O_{p, X} = A_p$ for any prime ideal $p \in \text{Spec}(A)$. Irreducible components of $\text{Spec}(A)$ are the sets $V(q)$, for $q$ a minimal prime.\\
		\par First, assume that $p$ belongs to a unique irreducible component of $\text{Spec}(A)$: that is, there is a unique minimal prime $q$ such that $q \subset p$. We want to show that $A_p$ is an integral domain. Let $x = a/b$, $y = f/g$ be two elements of $A_p$ such that $xy = 0$. By definition, this is equivalent to the existence of some $h \in A \backslash p$ such that $h(ga - bf) = 0$.
	\end{proof}
\end{problem}
\begin{problem}{2}
	Recall that the field of rational functions $k(X)$ of an irreducible, quasi-projective, algebraic variety $X$, is the collection of all rational functions on $X$.A rational function is a regular function $f$ on some non-empty open subset $U \subset X$ (up to the equivalence that $(f,U) \simeq (g, V)$ if $f = g$ on $U \cap V$).
	\begin{enumerate}[label=(\alph*)]
		\item Verify that any two open subsets in $X$ intersect and hence the field operations on $k(X)$ are well-defined. Then show that $k(X)$ is in fact a field.
			\begin{proof}
				<++>
			\end{proof}
		\item Show that if $X$ is an affine variety, then its field of rational functions coincides (is naturally isomorphic to) the field of fractions of its coordinate ring $k[X]$.
			\begin{proof}
				<++>
			\end{proof}
	\end{enumerate}
\end{problem}
\begin{problem}{3}
	Show that any two smooth quadrics in $\mathbb P^n$ are isomorphic. Recall that a quadric (in $\mathbb P^n$) is a subvariety defined by a (homogeneous) quadratic polynomial.
	\begin{proof}
		<++>
	\end{proof}
\end{problem}
\begin{problem}{4}
	Consider the affine curves $C$ in $\mathbb A^2$ below. Find the points at infinity on these curves (that is, points in the closure of $C$ in $\mathbb P^2$ that are not in $\mathbb A^2$). Decide for each point at infinity if it is singular or non-singular and find its tangent space and tangent cone.
	\begin{enumerate}
		\item $y^2 = x^3 + ax + b$
			\begin{proof}
				The homogenization of this polynomial is $y^2z = x^3 + axz^2 + bz^3$. Points at infinity correspond to tuples $(x,y,0)$, where $x$ and $y$ are not both $0$, which satisfy the equation
				\begin{align*}
					y^2 \cdot 0 &= x^3 + ax \cdot 0 + b \cdot 0\\
					0 &= x^3
				\end{align*}
				So, the point $(0,1,0)$ is the unique point at infinity on any elliptic curve.
				\par To find the tangent space at this point, we find the projective closure of the tangent space in $\mathbb A^2$ of the intersection of $X$ with a chart of $\mathbb P^2$ containing $(0,1,0)$. The only such chart is $E_1: \mathbb A \to \mathbb P^2$, defined by taking $(x,z) \mapsto (x:1:z)$, and the restriction of $X$ to this chart is the variety defined by the equation
				\[z = x^3 + axz^2 + bz^3\]
				The tangent space at $(0,0)$ is defined by the linearization of this equation, which is $z = 0$. This polynomial is homogenous, so its projective closure has no extra points: the tangent space of $X$ at $(0,1,0)$ is the algebraic variety $V(z)$, the zero set of the polynomial $z$.
			\end{proof}
		\item $y = x^3$
			\begin{proof}
				<++>
			\end{proof}
		\item $x^3 + x^2y - y = 0$
			\begin{proof}
				<++>
			\end{proof}
	\end{enumerate}
\end{problem}
\end{document}
