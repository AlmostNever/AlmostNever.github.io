% --------------------------------------------------------------
% Andrew Tindall
% --------------------------------------------------------------
 
\documentclass[12pt]{article}
 
\usepackage[margin=1in]{geometry} 
\usepackage{amsmath,amsthm,amssymb,enumitem,tikz-cd}

\newcommand{\N}{\mathbb{N}}
\newcommand{\Q}{\mathbb{Q}}
\newcommand{\Z}{\mathbb{Z}}
\newcommand{\R}{\mathbb{R}}
\newcommand{\C}{\mathbb{C}}
\newcommand{\mc}[1]{\mathcal{#1}}
\newcommand{\e}{\varepsilon}
\newcommand{\bs}{\backslash}
\newcommand{\PGL}{\text{PGL}}
\newcommand{\Sp}{\text{Sp}}
\newcommand{\tr}{\text{tr}}
\newcommand{\Lie}{\text{Lie}}
\newcommand{\rec}[1]{\frac{1}{#1}}
\newcommand{\toinf}{\rightarrow \infty}


\theoremstyle{definition}
\newtheorem{proofpart}{Part}
\newtheorem{theorem}{Theorem}[section]
\newtheorem{lemma}[theorem]{Lemma}
\newtheorem{definition}[theorem]{Definition}
\newtheorem{statement}[theorem]{Statement}
\makeatletter
\@addtoreset{proofpart}{theorem}
\makeatother


\newenvironment{problem}[2][Problem]{\begin{trivlist}
\item[\hskip \labelsep {\bfseries #1}\hskip \labelsep {\bfseries #2.}]}{\end{trivlist}}
 
\begin{document}
 
%\renewcommand{\qedsymbol}{\filledbox}
 
\title{Connected Complex Abelian Lie Groups: Tori and Number Theory}
\author{Andrew Tindall\\
Final Project\\
Lie Theory}
 
\maketitle
In this paper we will first show that all compact connected complex Lie groups are complex tori, then show that every complex abelian variety has the structure of a complex torus, and finally discuss the issue of whether a given complex torus is an abelian variety, solved through the Riemann conditions. 
\begin{section}{Tori}
	The following treatment of compact complex connected Lie groups follows section $1.1$ in \cite{mumford}, expanded where necessary. 
	\begin{definition} A \textbf{Complex Lie Group} $X$ of dimension $n$ is a complex manifold of dimension $n$ with a gorup structure on the underlying set such that the multiplication maps $X \times X \to X$ and inverse map $X \to X$ are both holomorphic. 
		
		\label{clg}
	\end{definition}
	\begin{lemma}
		For any homomorphism $T: X_1 \to X_2$ of complex Lie groups,
		\[ T(\exp_{X_1}y) = \exp_{X_2}( (dT)_ey).\] 
		\par As shown in B\"ocker \& tom Dieck, the following diagram commutes:
	\[\begin{tikzcd}&LX_1 \arrow[r, "(dT)_e"] \arrow[d, "\exp_{X_1}"] &LX_2 \arrow[d, "\exp_{X_2}"]\\&X_1 \arrow[r, "T"] & X_2\end{tikzcd}\]
		\label{exp}
	\end{lemma}
	\begin{theorem}
		Every compact connected complex Lie group is abelian.
		\label{abelian}
		\begin{proof}
			Let $X$ be such a group, with $x \in X$ arbitrary, and let $C_x$ be the conjugation map $X \to X$, taking $y$ to $xyx^{-1}$. We wish to show that $C_x$ is the identity map for all $x$. 
			\par We have seen that the differential of $C_x$ at the origin, $(dC_x)_e$, is always an automorphism of the Lie algebra $LX$. In this case, the association $x \to (dC_x)_e$ is a holomorphic map of $X$ into $\text{Aut}(LX)$, where we are viewing $\text{Aut}(LX)$ as a submanifold of the complex vector space $\text{End}(LX)$.  
			\par Because $x \to (dC_x)_e$ is a holomorphic map from a compact connected complex manifold, and the only holomorphic maps on a compact complex manifold are locally constant (By Liouville's Theorem), the value of $(dC_x)_e$ must be independent of $x$. The map $(dC_e)_e$ is the identity, so $(dC_x)_x$ must also be the identity for all $x$. 
			\par By the lemma above, we have
			\[C_x(\exp y) = \exp( (dC_x)_e y).\]
			Since $(dC_x)_e = 1$, we see that $C_x(\exp y) = \exp y$, so $\exp LX$ is in the center of $X$. Because $X$ is connected, it is generated by a neighborhood of the identity, and $\exp$ is a local diffeomorphism near the identity,  $X$ is generated by an abelian neighborhood of the identity, and so it is itself abelian.
		\end{proof}
	\end{theorem}
	As shown in Br\"ocker \& tom Dieck, every compact connected abelian Lie group is a torus: the exponential map $LX \to X$ is a surjective homomorphism, with kernel $U$ a lattice in the vector space $LX$.
\end{section}
\begin{section}{Abelian Varieties}
	We now discuss abelian varieties, and show their relation to complex Lie groups. Because full proofs of some of these theorems would take us far afield into algebraic geometry, we offer some theorems as unproved statements.
	\begin{definition}
		A \textbf{Projective Variety} $X$ over a field $k$ is a subset of projective $n$-space over $k$, which can be defined as the set of zeroes of a set of homogeneous polynomials with coefficients in $k$. 
		\label{projvar}
	\end{definition}
	\begin{definition}
		A \textbf{Meromorphic function} $f$ on an open subset $D$ of $\C^n$ is a function $ D \bs I \to \C$ which is holomorphic on all of $D$ except for a set $I$ of codimension $n$, which is the set of\textbf{poles} of the function. Equivalently, a meromorphic function is a function which is locally equal to a quotient of holomorphic functions.
		\par The \textbf{Polar Divisor} of a meromorphic function $f: D \bs I \to \C$ is a function $D \to \N$ which takes the value $0$ everywhere except the zeros and poles of $f$, and which takes the value $k$ at every zaero of order $k$, and $-j$ at every pole of order $-j$. For instance, the divisor of the function $\frac{z - \alpha}{(z - \beta)^2}$ takes value $1$ at $\alpha$, $-2$ at $\beta$, and $0$ everywhere else.
		\label{mero}
	\end{definition}
	\begin{definition}
		An \textbf{Abelian Variety} over $\C$ is a projective variety $X$ over $\C$, which is also a group, and for which the group law can be written as a homogeneous polynomial in the coordinates of $X$. Further, $X$ must be nonsingular - the Jacobian of the polynomials defining the variety has constant rank, and $X$ must be irreducible - it cannot be written as the disjoint union of two strictly smaller projective varieties.
		\label{abvar}
	\end{definition}
	The definition does not explicitly state that the group must be abelian, but in fact it always is:
	\begin{theorem}
		As an abstract group, an abelian variety $X$ over $\C$ is commutative and divisible ($mX = X$, for all $m \in \N_+$).
		\begin{proof}
			By definition, $X$ is a subset of the compact complex manifold $\mathbb{P}^n$ defined as the zero-set of a set of functions whose Jacobian has constant rank, and it is therefore a compact complex manifold itself. The coordinate functions are complex-analytic on $X$, and the group law is defined by polynomials in the coordinates, so it is itself complex-analytic. Further, $X$ is irreducible, so it is a connected manifold. As a compact connected complex Lie group, we have seen that $X$ must be abelian and divisible - in fact, it must be a torus.
		\end{proof}
	\end{theorem}
	So we see that every complex abelian varity  is a torus. One might expect that every complex torus is an abelian variety, but this is not true in general - although it is true in the simplest case.
	\begin{theorem}
		Every complex torus of complex dimension $1$ is an abelian variety.
		\begin{proof}
			Let $T$ be a complex torus of complex dimension $1$: $T = \C / U$, where $U$ is the lattice in $\C$ generated by two complex numbers $\omega_1 = a + bi$ and $\omega_2 = c + di$, where $ad -bc \neq 0$.
			\par Define the Weierstrass function $\wp(z; U)$ by
			\[\wp = \frac{1}{z^{2}} + \sum_{n^2 + m^2 \neq 0} \left( \frac{1}{(z + n\omega_1 + m\omega_2)^2} - \frac{1}{(n\omega_1 + m\omega_2)^2} \right).\]
			This is a meromorphic function with a pole of order $2$ at every point of $U$. Further, $\wp(z;U)$ is equal to $\wp(z + u; U)$ for any $u \in U$, so it descends to a meromorphic function on the torus $\C/U$. We use the following statement without proof; its proof can be found in a book on elliptic functions, such as \cite{elliptic}.
			\par The function $\wp(z)$ satisfies the differential equation
			\[(\wp')^2 - 4\wp^3(z) + g_2\wp(z) + g_3,\]
			Where $g_2$ and $g_3$ are constants depending on $\omega_1$ and $\omega_2$:
			\begin{align*}
				g_2 = 30 \sum_{(m, n)\neq (0,0)} \left( n\omega_1 + m \omega_2 \right)^{-4}\\
				g_3 = 5 \sum_{(m,n)\neq (0,0)} \left( n\omega_1 + m\omega_2 \right)^{-6}
			\end{align*}
			Let $E$ be the projective variety in three variables $[w:x:y]$, given by 
		\[wy^2 = 4x^3 + g_2 w^2x + g_3w^3.\]
			By the differential equation which $\wp$ satisfies, we see that the function
			\begin{align*}z &\mapsto [1:\wp(z):\wp'(z)]\\
			0 &\mapsto [1:0:0]\end{align*}
			maps the torus $\C/U$ to the projective variety $E$. 
			\par Therefore, the torus is homeomorphic to the set of complex points of a projective variety. This homeomorphism is also a group isomorphism: as shown in \cite{elliptic}, for any points $z_1, z_2$ on the torus $\C/U$, where $z_1 \neq z_2$, we have
			\begin{align*}\wp(z_1 + z_2) &= \frac{1}{4}\left( \frac{\wp'(z_1) - \wp'(z_2)}{\wp(z_1) - \wp(z_2)} \right)^2 - \wp(z_1) - \wp(z_2)\\
			\wp'(z_1 + z_2) &= \frac{1}{2}\left( \frac{\wp'(z_1) - \wp'(z_2)}{\wp(z_1) - \wp(z_2)} \right)(\wp(z_2) - \wp(z_1 + z_2)) - \wp'(z_2)\end{align*}
			When $z_1 = z_2$, we have
			\begin{align*}
				\wp(2z_1) &= \left( \frac{3\wp(z_1)^2 + g_3}{\wp'(z_1)} \right)^2 - 2\wp(z_1)\\
				\wp'(2z_1) &= -\wp'(z_1)
			\end{align*}
			This group law defines an abelian variety structure on the projective variety $E$; so we have an isomorphism between $\C/U$ and an abelian variety $E$ over $\C$.
		\end{proof}
		\label{ellcurve}
	\end{theorem}
	Every torus $\C/U$ is therefore an abelian variety. However, in higher dimensions this is no longer true: most tori $\C^n / U$ are not complex abelian varieties, even though we have seen that every complex abelian variety is a torus. However, there is a well-defined way to determine whether or not a given torus is an abelian variety.
	\begin{statement}
		Let $X = \C^n / U$ be an $n$-dimensional complex torus. Then $X$ is an abelian variety if and only if there exists a positive definite hermitian form $H$ on $\C^n / U$ such that $\text{Im}(H)$, the imaginary part of $H$, takes integral values on $U \times U$.	
		\par This theorem is proved in \cite{mumford}, and relies on the identification of a line bundle on $X$ with a unique choice of a hermitian form $H$, and a map $\alpha: U \to \left\{ z \in \C^* ; \left \lvert { z } \right \lvert  = 1 \right\}$ such that 
		\[\alpha(u_1 + u_2) = e^{i\pi \text{Im}(H)(z,u)} \cdot \alpha(u_1), \alpha(u_2), u_i \in U. \]
	\end{statement}
\end{section}
\begin{section}{Complex Multiplication}
	Now, we look at an application of the identification of Complex Tori with elliptic curves. <++>
\end{section}<++>
\begin{thebibliography}{}
	\bibitem{mumford}{Mumford, David. Abelian Varieties, revised 2nd edition. TIFR, 2008}
	\bibitem{elliptic}{Chandrasekharan, K. Eliptic Functions. Springer-Verlag, 1984}
	\bibitem{hochschild}{Hochschild, G. The Structure of Lie Groups. Holden-Day, 1965}
	\bibitem{brocker}{Br\"ocker, T. \& tom Dieck, T. Representations of Compact Lie Groups. Springer-Verlag, 1985}
	\bibitem{hartshorne}{Hartshorne, R. Algebraic Geometry. Springer-Verlag, 1977}
\end{thebibliography}

\end{document}
