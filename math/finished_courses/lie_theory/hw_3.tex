% --------------------------------------------------------------
% Andrew Tindall
% --------------------------------------------------------------
 
\documentclass[12pt]{article}
 
\usepackage[margin=1in]{geometry} 
\usepackage{amsmath,amsthm,amssymb,enumitem}

\newcommand{\N}{\mathbb{N}}
\newcommand{\Q}{\mathbb{Q}}
\newcommand{\Z}{\mathbb{Z}}
\newcommand{\R}{\mathbb{R}}
\newcommand{\mc}[1]{\mathcal{#1}}
\newcommand{\e}{\varepsilon}
\newcommand{\bs}{\backslash}
\newcommand{\PGL}{\text{PGL}}
\newcommand{\Sp}{\text{Sp}}
\newcommand{\tr}{\text{tr}}
\newcommand{\Lie}{\text{Lie}}
\newcommand{\rec}[1]{\frac{1}{#1}}
\newcommand{\toinf}{\rightarrow \infty}


\theoremstyle{definition}
\newtheorem{proofpart}{Part}
\newtheorem{theorem}{Theorem}
\makeatletter
\@addtoreset{proofpart}{theorem}
\makeatother


\newenvironment{problem}[2][Problem]{\begin{trivlist}
\item[\hskip \labelsep {\bfseries #1}\hskip \labelsep {\bfseries #2.}]}{\end{trivlist}}
 
\begin{document}
 
%\renewcommand{\qedsymbol}{\filledbox}
 
\title{Homework 3}
\author{Andrew Tindall\\
Lie Theory}
 
\maketitle
\begin{section}{Book Problems}
	\begin{problem}{1}
		Br\"ocker \& tom Dieck, IV.1.12.5: Let $f: G \to H$ be a surjective homomorphism of Lie groups and suppose $H$ is abelian. Show that $f\lvert_T: T \to H$ is also surjective. (where $G$ is a compact connected Lie group, and $T$ is a maximal torus of $G$).
		\begin{proof}
			By the Main Theorem on Tori in Br\"ocker \& tom Dieck, every element of $G$ is conjugate to some element of the maximal torus $T$ (because every element is contained in a maximal torus, and any two maximal tori are conjugate.) Therefore, if $h \in H$ is the image of some $g \in G$, and we have $g = xtx^{-1}$ for some $x \in G$ and $ t \in T$,
			\begin{align*}
				h &= f(g)\\
				&= f(xtx^{-1})\\
				&= f(x)f(t)(f(x))^{-1}\\
				&= f(t),
			\end{align*}
			because $H$ is abelian. Therefore, every $h \in H$ is in the image of $f\lvert_T$.
		\end{proof}
	\end{problem}
	\begin{problem}{2}
		Br\"ocker \& tom Dieck, IV.1.12.6: Show that every abelian normal subgroup of a compact connected Lie group $G$ lies in the center of $G$.
		\begin{proof}
			We use the lemma, shown in Br\"ocker \& tom Dieck, that the center of $G$ is the intersection of all maximal tori in $G$. It is clear that every maximal torus contains the center of $G$, since otherwise one could construct a strictly larger torus; in the other direction, if $x$ is in every maximal torus, and $g$ is an arbitrary element of $G$, then $g$ is contained in some maximal torus, which must also contain $x$, and so $x$ and $g$ commute. Therefore, it suffices to show that any abelian normal subgroup $N$ of a compact Lie group $G$ is contained in every maximal torus of $G$. 
			\par Let $N$ be an abelian normal subgroup of the compact connected Lie group $G$, and let $T$ be a maximal torus of $G$. By a theorem in Br\"ocker \& tom Dieck, the map
			\[q: G/T \times T \to G, \qquad (g, t)\mapsto gtg^-1\]
			is surjective. In particular, the group $N$ lies in the image of $q$. We now show that $q^{-1}(N) \subset e \times T$.
			\par The projection of $q^{-1}(N)$ into $G/T$ is an abelian normal subgroup of $G/T$. This subgroup must be totally disconnected, because $G/T$ is semisimple - if $\pi(q^{-1}(N)) = H \times T'$, where $T'$ is a nontrivial torus, then $T \times T'$ would be a strictly larger torus contianing $T$, contradicting maximality. 
		\end{proof}
	\end{problem}
	\begin{problem}{3}
		Br\"ocker \& tom Dieck, IV.1.12.8: Let $S \subset G$ be a closed subgroup such that $G = \bigcup_{g \in G}gSg^{-1}$. Show that $S$ contains a maximal torus.
		\begin{proof}
			If $S$ contains no maximal torus, then neither does $gSg^{-1}$ for any $g \in G$. 
		\end{proof}
	\end{problem}
	\begin{problem}{4}
		Br\"ocker \& tom Dieck, IV.2.12.4: Show that every compact Lie group contains a finitely generated dense subgroup. Give an upper bound for the number of generators needed.
		\begin{proof}
			We show that the number of generators needed to generate a compact Lie Group $G$ is at most $D - R + 1$, where $D$ is the dimension of the Lie group $G$, and $R$ is the dimension of the maximal torus of $G$.
			\par We have seen already that any torus can be generated by one element $(x_1, \dots x_n)$, where $1, x_1, \dots x_n$ are linearly independent over $\mathbb{Q}$. If $T = G$, we are done.
			\par If $T \neq G$, then let $\mathfrak t$ be the Lie algebra of $T$ and $\mathfrak g$ the Lie algebra of $G$. By a lemma shown in \cite{devito}, there is a subgroup $S^1$ of $G$ such that $S^1 \cap T$ is finite. If $y$ is a generator of $S^1$ and $x$ is a generator of $T$, then the closure of the subgroup generated by $x$ and $y$ must contain both $S^1$ and $T$; its dimension is therefore at least $1$ greater than that of $G$. We can proceed in this way until we have generated $G$ with finitely many elements, and we see that we can have at most $D - R$ steps, giving us the desired bound on the number of generators.
		\end{proof}
	\end{problem}
	\begin{problem}{5}
		Br\"ocker \& tom Dieck, IV.2.12.6: Let $G$ be a compact connected Lie group and $g \in G$. Show that $Z(g)_0$ is the union of the maximal tori in $G$ containing $g$. Find a $g \in \text{SO}(3)$ such that $Z(g)$ is not connected.
		\begin{proof}
			It is clear that $Z(g)_0$ contains each maximal torus in $G$ containing $g$, because any element in a maximal torus containing $g$ must commute with $g$. To see the converse, that every element in $Z(g)_0$ is in a maximal torus containing $g$, let $T$ be some maximal torus in $G$ which contains $g$ (which must exist by the fact that $G$ is compact and connected), and let $h \in Z(g)_0$ be an arbitrary element of the connected component of the centralizer of $g$ which contains the identity.
			\par Because $Z(g)_0$ is a compact connected Lie group, $h$ must be contained in a maximal torus $T$ of $Z(g)_0$. Because $g$ commuted with every element of $Z(g)_0$, it is in $Z(Z(g)_0)$, which we have seen is the intersection of all maximal tori in $Z(g)_0$, so $T$ must also contain $g$. Finally, $T$ must be contained in some maximal torus $T'$ of $G$. Therefore every element of $Z(g)_0$ is contained in a maximal torus of $G$ which contains $g$.
		\end{proof}
	\end{problem}
	\begin{problem}{6}
		V.4.15.5: Show that the Weyl group of a root system $R$ contains no reflections other than the $s_\alpha$, $\alpha \in R$.
		\begin{proof}
			\textit{incomplete}
		\end{proof}
	\end{problem}
	\begin{problem}{7}
		V.5.14.3: Show that a root system splits uniquely into a sum of irreducible root systems. The subscpaces $V_\nu \subset V$ and the subsets $R_\nu \subset R$ of such a splitting are uniquely determined by $R \subset V$. Show that the Weyl group of $R$ is the direct product of the Weyl groups of the irreducible components of $R$.
		\begin{proof}
			\textit{incomplete}
		\end{proof}
	\end{problem}
	\begin{problem}{8}
		V.5.14.11: Compute the determinant of the Cartan matrix for  every type of the Dynkin diagrams corresponding to irreducible root systems. (The correct answers are given in the text)
						
		\par The definition of the Cartan matrix $\left\{ a_{ij} \right\}$ for a given root system is that 
						\[a_{ij} = \frac{\langle a_i, a_j\rangle}{\langle a_j, a_j\rangle},\]where $\langle -, -\rangle$ is an invariant inner product on the representation. This matrix can be easily read off from the Dynkin diagram corresponding to the root system, and in each case the determinant is relatively easily calculated. 
			\begin{itemize}
				\item $A_n$: Determinant is $n+1$. \begin{proof}
						The form of the Cartan matrix of $A_n$ is as follows:
						\[\begin{bmatrix}
								2 & -1 & 0 & \dots & 0 & 0\\
								-1 & 2 & -1 & \dots & 0 & 0\\
								0 & -1 & 2 & \dots & 0 & 0\\
							\vdots & \vdots & \vdots & \ddots & \vdots & \vdots\\
							0 & 0 & 0 & \dots & 2 & -1\\
							0 & 0 & 0 & \dots & -1 & 2
					\end{bmatrix}\]
					\par For $n \geq 2$, the $1,1$th minor of this matrix is clearly the Cartan matrix of $A_{n-1}$, which we can denote by $\alpha_{n-1}$, and the $1,2$th minor is the Cartan matrix of $A_{n-1}$ with the upper-left $2$ replaced with a $-1$, which we can call $\alpha'_{n-1}$. The minors of $\alpha'_{n-1}$ (for $n \geq 3$) are, in turn, $\alpha_{n-2}$ and $\alpha'_{n-2}$. Expanding the determinant in terms of these minors, we get the following recurrence relation:
					\[\left \lvert { \alpha_n } \right \lvert = 2\left \lvert { \alpha_{n-1} } \right \lvert  - \lvert \alpha_{n-2}\rvert - \lvert\alpha_{n-3}\rvert - \dots - \lvert \alpha_1\rvert\]
					Along with with the base case $\left \lvert { \alpha_1 } \right \lvert = 2$, we have
					\begin{align*}
						\lvert \alpha_{1}\rvert &= 2\\
						\lvert \alpha_{n} \rvert &= \lvert \alpha_{n-1}\rvert + 1
					\end{align*}
					Or, $\lvert \alpha_{n}\rvert = n+1$.
					\end{proof}
				\item $B_n$: Determinant is $2$. \begin{proof}
						The form of the matrix for $B_n$, $n \geq 2$ is:
						\[ \begin{bmatrix}
								2 & -1 & 0 & \dots & 0 & 0\\
								-1 & 2 & -1 & \dots & 0 & 0\\
								0 & -1 & 2 & \dots & 0 & 0\\
							\vdots & \vdots & \vdots & \ddots & \vdots & \vdots\\
							0 & 0 & 0 & \dots & 2 & 2\\
							0 & 0 & 0 & \dots & -1 & 2
					\end{bmatrix}\]
					The recurrence relations for this matrix, which we call $\beta_n$, are the same as the last, except for the case $\beta_{2}$, which is $2$ instead of $3$. Because the full expansion of the recurrence relation 
					\[\left \lvert { \beta_n } \right \lvert = 2\left \lvert { \beta_{n-1} } \right \lvert  - \lvert \beta_{n-2}\rvert - \lvert\beta_{n-3}\rvert - \dots - \lvert \beta_1\rvert\]
					Includes $n-1$ terms of $-\beta_{2}$, and $\beta_2 = \alpha_2 - 1$, we see that each term $\lvert \beta_n\rvert $ is less than $\lvert \alpha_n \rvert$ by $n_1$- i.e. that $\left \lvert { \beta_n } \right \lvert = \left \lvert { \alpha_n } \right \lvert - (n-1)$, which means that
					\[\beta_n = 2\]
					\end{proof}
				\item $C_n$: Determinant is $2$. \begin{proof}
						The matrix for $C_n$ is as follows:
						\[\begin{bmatrix}
								2 & -1 & 0 & \dots & 0 & 0\\
								-1 & 2 & -1 & \dots & 0 & 0\\
								0 & -1 & 2 & \dots & 0 & 0\\
							\vdots & \vdots & \vdots & \ddots & \vdots & \vdots\\
							0 & 0 & 0 & \dots & 2 & -1\\
							0 & 0 & 0 & \dots & 2 & 2
					\end{bmatrix}\]
					This is just the transpose of $B_n$, so it has the same determinant: $\left \lvert { C_n } \right \lvert  = 2$.
					\end{proof}
				\item $D_n$: Determinant is $4$. \begin{proof}
						The form of the matrix for $D_n$ is:
						\[ \begin{bmatrix}
								2 & -1 & 0 & \dots & 0 & 0 & 0\\
								-1 & 2 & -1 & \dots & 0 & 0 & 0\\
								0 & -1 & 2 & \dots & 0 & 0 & 0\\
							\vdots & \vdots & \vdots & \ddots & \vdots & \vdots & \vdots\\
							0 & 0 & 0 & \dots & 2 & -1 & -1\\
							0 & 0 & 0 & \dots &-1 & 2 & 0 \\
							0 & 0 & 0 & \dots & -1 & 0 & 2
					\end{bmatrix}\]
					Again, the recurrence relation for this matrix is the same, with a base case $\delta_3 = 4$. T
					\end{proof}
				\item $E_6$: Determinant is $3$. \begin{proof}
						We see that the matrix of $E_n$, for any $n$, is 
						\[
							\begin{bmatrix}
						f	
					\end{bmatrix}\]
					\end{proof}
				\item $E_7$: Determinant is $2$. \begin{proof}
						\textit{incomplete}
					\end{proof}
				\item $E_8$: Determinant is $1$. \begin{proof}
						\textit{incomplete}
					\end{proof}
				\item $F_4$: Determinant is $1$. \begin{proof}
						\textit{incomplete}
					\end{proof}
				\item $G_2$: Determinant is $1$. \begin{proof}
						\textit{incomplete}
					\end{proof}
			\end{itemize}
	\end{problem}
\end{section}
\begin{section}{Problems from the Course Website}
	\begin{problem}{1}
		Denote by $V_n$ the $(n+1)$-dimensional simple $\text{SU}(2)$ representation. For any non-negative integers $n$ and $m$ find the decomposition of the $\text{SU}(2)$ representation $V_n \otimes V_m$ as a direct sum of simple representations.
		\begin{proof}
			\textit{incomplete}
		\end{proof}
	\end{problem}
	\begin{problem}{2}
		Find a maximal torus and the corresponding roots and root spaces for $\text{SU}(n)$, $\text{SO}(n)$, $\text{Sp}(n)$.
		\begin{proof}
			A maximal torus for $SU(n)$ is the set of all unitary diagonal matrices with determinant $1$; that is, matrices of the form
			\[\begin{bmatrix}
					e^{2\pi \lambda_1 i} & 0 & \dots & 0\\
					0 & e^{2 \pi \lambda_{2} i} & \dots & 0\\
					\vdots & \vdots & \ddots & \vdots \\
					0 & 0 & \dots & e^{-2\pi (\lambda_1 + \lambda_2 + \dots + \lambda_{n-1})i }
		\end{bmatrix}\]
		The corresponding Cartan subalgebra of $\mathfrak{su}(n)$ is the set of all diagonal matrices $H_\lambda$ with entries $(H_\lambda)_{jj} = 2\pi \lambda_j i$, $\sum \lambda_j = 0$. 
		\par The root spaces are labelled by pairs $j, k$, $j \neq k$, of integers from $1$ to $n$, and each root $\alpha_{j, k}$ is given by 
		\[\alpha_{j, k}(H_\lambda) = 2\pi (\lambda_j - \lambda_k).\]
		\par Each root space $V_{\alpha_{j, k}}$ is the set of matrices whose entries are all $0$ except for the $j, k$th, which is arbitrary.
		\par \textit{Source: }\cite{woit}
		\par A maximal torus for $\text{SO}(n)$ is the set of all block-diagonal matrices with $2\times 2$ blocks, where each block is a rotation matrix, i.e. all matrices of the form
		\[\begin{bmatrix}
				R_1 & 0 & \dots & 0\\
				0 & R_2 & \dots & 0\\
				\vdots & \vdots & \ddots & \vdots \\
				0 & 0 & \dots & R_{n/2}\end{bmatrix}, \qquad R_i = \begin{bmatrix}
					\cos \theta_i & -\sin \theta_i\\
					\sin \theta_i & \cos \theta_i
				
	\end{bmatrix}\]
	Or, if $n$ is odd, the same matrices but with a $1$ in the lower-right corner. (i.e., a direction fixed). The Lie algebra of this representation is the set of all block-diagonal matrices with $2\times 2$ blocks $\begin{bmatrix}
		0 & \theta_i \\ -\theta_i & 0
	\end{bmatrix}$ on the diagonal. Then a basis for this Lie algebra is $H_1, \dots H_n$, where each $H_i$ is the matrix with $\theta_i = 1$ and all other $\theta$s zero.
	\textit{Source: }\cite{tapp}
		\end{proof}
	\end{problem}
\end{section}
\begin{thebibliography}{}
	\bibitem{brocker}{Br\"ocker \& tom Dieck, \textit{Representations of Compact Lie Groups}. Springer, 1985}
		%https://deepblue.lib.umich.edu/bitstream/handle/2027.42/70011/JMAPAQ-23-11-2019-1.pdf
	\bibitem{devito}{DeVito, J, Show that every compact Lie group contains a finitely generated dense subgroup. URL: https://math.stackexchange.com/q/2855928}
	\bibitem{tapp}{Tapp, Kristopher: Matrix Groups for Undergraduates. AMS, 2005}
	\bibitem{woit}{Woit, Peter. ``Topics in Representation Theory: $SU(n)$, Weyl Chambers and the Diagram of a Group}
\end{thebibliography}
\end{document}
