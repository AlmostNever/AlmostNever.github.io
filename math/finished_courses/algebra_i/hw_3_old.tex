% --------------------------------------------------------------
% Andrew Tindall
% --------------------------------------------------------------
 
\documentclass[12pt]{article}
 
\usepackage[margin=1in]{geometry} 
\usepackage{amsmath,amsthm,amssymb,enumitem,hyperref}

\newcommand{\N}{\mathbb{N}}
\newcommand{\Q}{\mathbb{Q}}
\newcommand{\Z}{\mathbb{Z}}
\newcommand{\C}{\mathbb{C}}
\newcommand{\R}{\mathbb{R}}
\newcommand{\mc}[1]{\mathcal{#1}}
\newcommand{\e}{\varepsilon}
\newcommand{\bs}{\backslash}
\newcommand{\PGL}{\text{PGL}}
\newcommand{\Sp}{\text{Sp}}
\newcommand{\tr}{\text{tr}}
\newcommand{\Lie}{\text{Lie}}
\newcommand{\rec}[1]{\frac{1}{#1}}
\newcommand{\toinf}{\rightarrow \infty}


\theoremstyle{definition}
\newtheorem{proofpart}{Part}
\newtheorem{theorem}{Theorem}
\makeatletter
\@addtoreset{proofpart}{theorem}
\makeatother


\newenvironment{problem}[2][Problem]{\begin{trivlist}
\item[\hskip \labelsep {\bfseries #1}\hskip \labelsep {\bfseries #2.}]}{\end{trivlist}}
 
\begin{document}
 
%\renewcommand{\qedsymbol}{\filledbox}
 
\title{Homework 3}
\author{Andrew Tindall\\
Algebra 1}
 
\maketitle
\begin{problem}{1}
	Show that a finite group generated by two involutions is dihedral.
	\begin{proof}
		We recall the definition of the dihedral group $D_n$ as
		\[D_n = \langle r, s \mid r^2 = s^n = e, rs = sr^{-1}\rangle.\]
		Let $G$ be a finite group generated by $a$, $b$, where $a$ and $b$ are involutions: 
		\[a^2 = b^2 = e\]
		Now, since $a = a^{-1}$ and $b = b^{-1}$, every element in $G$ can be written as a finite-length word in $a$ and $b$. Also, since $a^2 = b^2 = e$, we only need to consider words with no two identical adjacent letters. Therefore, the following two lists exhaust the group:
		\begin{align*}
			a&, ab, aba, abab, \dots\\
			b&, ba, bab, baba, \dots
		\end{align*}
		Since $G$ is finite, neither of these lists can be infinite. Let $e = \underbrace{aba\dots}_{\text{$m$ terms}} $ be the first occurrence of an identity element in the first list $a, ab, \dots$.
		\par It must be true that $e$ is of the form $(ab)^n$ for some $n$, because otherwise we could cancel on the left and right - say $e = aba\dots aba = (ab)^na$ for some $n$. Then 
		\begin{align*}
			e &= aea\\
			&= a(ab)^n a^2\\
			&= (ba)^{n-1}b
		\end{align*}
		But then we would have 
		\begin{align*}
			e &= beb\\
			&= b(ba)^{n-1}b^2\\
			&= (ab)^{n-2}a
		\end{align*}
		Which contradicts our choice of $\underbrace{aba\dots}_{\text{ $m$ terms}}$ as the shortest such word. So, $(ab)^{n} = e$ for some $e$.
		\par We see that this gives $(ba)^{n}$ as the shortest word in the second list which is equal to the identity, as well. First, we see that if we have $(ab)^n = e$, then it must also be true that $(ba)^n = e$:
		\begin{align*}
			(ba)^n &= a^2 (ba)^n\\
			&= a(ab)^n a\\
			&= a^2\\
			&= e
		\end{align*}
		Now, since $(ba)^n = e$ also implies that $(ab)^n =e$, it must be true that the smallest such $n$ are equal. So, there are at most $2n$ elements in $G$.
		\par In fact, there are exactly $2n$ elements in $G$.
	\end{proof}<++>
\end{problem}
\begin{problem}{2}
	What is the order of the largest cyclic subgroup of $S_n$?
	\begin{proof}
		We show that the order of the largest cyclic subgroup of $n$, which we denote $o_n$, is equal to the maximum Least Common Multiple of any of the sets of nonzero numbers which partition $n$ (Landau's Function on $n$).
		\par First, note that since $\left \lvert {  \langle x\rangle } \right \lvert  = \text{ord}(x)$, we are equivalently looking for the greatest order of any element $x \in S_n$. Now, let $x \in S_n$. The element $x$ has a unique cycle decomposition, up to reordering of the cycles and cyclic reordering of the elements in the cycles. The multiset of lengths of the cycles of an element is its \textit{cycle type}: for example, the cycle type of $x = (123)(4567)(89)(10,11)$ is $\left\{ 3,4,2,2 \right\}$. We also see that $x^k = 0$ if and only if $k \equiv 0$ modulo $3,4,2$, and $2$. The lowest such $k$ is exactly the least common multiple of the set $\left\{ 3,4,2 \right\}$.
		\par So, we see that every element has a unique multiset partitioning $n$ which determines its orderin $S_n$. We will also see that every multiset $S$ partitioning $n$ corresponds to at least one element of $S_n$: for instance, if $n=6$ and $S = \left\{ 1,3,2 \right\}$, then one such element with this multiset as its cycle class is $(23)(456) = (1)(23)(456)$, and the order of this element is $\text{Lcm}(\left\{ 1,2,3 \right\}) = 6$.
		So, for any $n$, the largest cyclic subgroup has order equal to the maximum possible $\text{Lcm}$ over all partitions $S$ of $n$. Letting $P(n)$ be the set of all partitions of $n$, we have shown that
		\[o_n = \text{max}_{S \in P(n)} (\text{Lcm}(S)).\]
	\end{proof}
\end{problem}
\begin{problem}{3}
	Frobenius' Thoerem states that if $n$ divides the order of a group then the number of elements whose order divide $n$ is a multiple of $n$.
	\begin{enumerate}[label = (\alph*)]
		\item Verify directly this theorem for the group $S_5$ and $n = 6$.
			\begin{proof}
				<++>
			\end{proof}
		\item Give an example of two nonisomorphic groups of order $m$ which have the same number of elements of order $d$, for each divisor $d$ of $m$.
			\begin{proof}
				<++>
			\end{proof}
	\end{enumerate}
\end{problem}
\begin{problem}{4}
	If $G$ is a non-Abelian $p$-group of order $p^3$, then $Z(G) = [G,G]$.
	\begin{proof}
		<++>
	\end{proof}
\end{problem}
\begin{problem}{5}
	If $p$ is an odd prime, then there are at most two nonabelian groups of order $p^3$.	
	\begin{proof}
		<++>
	\end{proof}
\end{problem}
\begin{problem}{6}
	If $P \trianglelefteq H \trianglelefteq G$ with $P$ a Sylow $p$-subgroup of $G$, then $P \trianglelefteq G$.	\begin{proof}
		<++>
	\end{proof}
\end{problem}
\begin{problem}{7}
	Prove that, if $p$ and $q$ are primes, a group of order $p^2 q$ cannot be simple.
	\begin{proof}
		<++>
	\end{proof}
\end{problem}
\begin{problem}{8}
	\begin{enumerate}[label=(\alph*)]
		\item Prove that there are no non-Abelian simple groups of order less than $60$.
			\begin{proof}
				<++>
			\end{proof}
		\item Prove that $A_5$ is the unique simple group of order $60$, up to isomorphism.
	\end{enumerate}
\end{problem}
\begin{problem}{9}
	Prove that $A_4$ is solvable but nilpotent.
	\begin{proof}
		<++>
	\end{proof}
\end{problem}
\begin{problem}{10}
	Show that $S_4$ is neither perfect, nor simple, nor nilpotent, but is solvable.
	\begin{proof}
	<++>	
	\end{proof}
\end{problem}
\begin{problem}{11}
	If $S \trianglelefteq G$ and $T$ are solvable subgroups of $G$, then $ST$ is a solvable subgroup of $G$.
	\begin{proof}
	<++>	
	\end{proof}
\end{problem}
\begin{problem}{12}
	The dihedral groups are solvable.
	\begin{proof}
		<++>
	\end{proof}
\end{problem}
\begin{problem}
	Let $p$ and $q$ be primes.
	\begin{enumerate}[label=(\alph*)]
		\item Prove that any groups of order $p^2q$ is solvable.
			\begin{proof}
				<++>
			\end{proof}
		\item If $p > q$, then any group of order $p^n q$ is solvable.
			\begin{proof}
				<++>
			\end{proof}
	\end{enumerate}
\end{problem}
\begin{problem}{14}
	The dihedral group of order $n$ is nilpotent if and only if $n$ is a power of $2$.
	\begin{proof}
		<++>
	\end{proof}
\end{problem}
\begin{problem}{15}
	If $m$ divides the order of a nilpotent group $G$, then $G$ contains a subgroup of order $m$.
	\begin{proof}
		<++>
	\end{proof}
\end{problem}
\begin{problem}{16}
	\begin{enumerate}[label=(\alph*)]
		\item If $H$ and $K$ are normal nilpotent subgroups of $G$, then $HK$ is a normal subgroup of $G$.
			\begin{proof}
				
			\end{proof}
		\item Any group $G$ contains a unique maximal normal nilpotent subgroup called the \textit{Fitting subgroup} of $G$.
	\end{enumerate}
	
\end{problem}
\begin{problem}{17}
	Upper triangular $n \times n$ matrices with entries in a Galois field and $1$'s on the main diagonal form a nilpotent group. Compute the class, center, and commutator subgroup of this group.
	\begin{proof}
		<++>
	\end{proof}
\end{problem}
\begin{problem}{18}
	A group is nilpotent if and only if every maximal subgroup is normal.
	\begin{proof}
		<++>
	\end{proof}
\end{problem}
\begin{problem}{19}
	Let $P$ be a $p$-group, and $\Phi(P)$ be the intersection of all maximal subgroups of $P$. Prove that, for $Q \trianglelefteq P$, the group $P/Q$ is elementary Abelian if and only if $\Phi(G) \leq Q$.
\end{problem}
\begin{problem}{20}
	\begin{enumerate}[label=(\alph*)]
		\item If $A$ is Abelian and $n$ divides the order of $A$, then $A$ contains a subgroup of order $n$.
			\begin{proof}
		<++>		
			\end{proof}
		\item If $B \leq A$ , then $A$ contains a subgroup isomorphic to $A/B$.
			\begin{proof}
				<++>
			\end{proof}
	\end{enumerate}
\end{problem}
\end{document}
