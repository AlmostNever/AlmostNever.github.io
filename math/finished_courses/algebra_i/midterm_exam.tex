% --------------------------------------------------------------
% Andrew Tindall
% --------------------------------------------------------------
 
\documentclass[12pt]{article}
 
\usepackage[margin=1in]{geometry} 
\usepackage{amsmath,amsthm,amssymb,enumitem,hyperref}

\newcommand{\N}{\mathbb{N}}
\newcommand{\Q}{\mathbb{Q}}
\newcommand{\Z}{\mathbb{Z}}
\newcommand{\C}{\mathbb{C}}
\newcommand{\R}{\mathbb{R}}
\newcommand{\mc}[1]{\mathcal{#1}}
\newcommand{\e}{\varepsilon}
\newcommand{\bs}{\backslash}
\newcommand{\PGL}{\text{PGL}}
\newcommand{\Sp}{\text{Sp}}
\newcommand{\tr}{\text{tr}}
\newcommand{\Lie}{\text{Lie}}
\newcommand{\rec}[1]{\frac{1}{#1}}
\newcommand{\toinf}{\rightarrow \infty}


\theoremstyle{definition}
\newtheorem{proofpart}{Part}
\newtheorem{theorem}{Theorem}
\makeatletter
\@addtoreset{proofpart}{theorem}
\makeatother


\newenvironment{problem}[2][Problem]{\begin{trivlist}
\item[\hskip \labelsep {\bfseries #1}\hskip \labelsep {\bfseries #2.}]}{\end{trivlist}}
 
\begin{document}
 
%\renewcommand{\qedsymbol}{\filledbox}
 
\title{Midterm Exam}
\author{Andrew Tindall\\
Algebra 1\\
Spring 2020}
 
\maketitle
\begin{section}{Midterm}
\begin{problem}{1}
	Let $H$ be a subgroup of a cyclic group $G$.
	\begin{enumerate}[label=(\alph*)]
		\item Prove that $H$ is characteristic in $G$. [Aside: Would this also be true if $G$ were abelian? Explain.]
			\begin{proof}
				First, we know that $H$ is cyclic, as every subgroup of a cyclic group is itself cyclic. The order of $H$ must divide the order of $G$, and in fact $H$ is the unique subgroup of $G$ of order $\left \lvert { H } \right \lvert $ (p. 135 of Dummit \& Foote). Now, let $\varphi: G \simeq G$ be an automorphism of $G$. Because the restriction of an automorphism is an automorphism, $\varphi\lvert_H: H \simeq \varphi(H) $ is a bijection. In particular, $\left \lvert { \varphi(H) } \right \lvert  = \left \lvert { H } \right \lvert $, meaning that $\left \lvert { \varphi(H) } \right \lvert $ is a subgroup of $G$ of order $\lvert H \rvert$. We know that $H$ is the unique subgroup with this property, so $\varphi(H) = H$, and $H$ is characteristic.
				\par This property does not hold for arbitrary abelian groups. For instance, the automorphism $\varphi: \Z_2 \times \Z_2$ which takes $(x,y)$ to $(y,x)$, takes the subgroup $\langle (1,0)\rangle$ to the subgroup $\langle (0,1)\rangle$.
			\end{proof}
		\item Show that every automotphism of $H$ is the restriction to $H$ of an automorphism of $G$.
			\begin{proof}	
				Let $G = \langle x\rangle$ be a cyclic group of order $n$, $H = \langle x^b\rangle$ be a subgroup of order $m$ (where $b$ = $n/m$ ), and let $\varphi$ be an automorphism of $H$. Then $\varphi(x^b) = (x^b)^d$, where $(d,m) = 1$. If we find a $c$ such that $bc = bd$ mod $n$, and $(c,n) = 1$, then $x \mapsto x^c$ would restrict to $\varphi$ on $H$, as $(x^b)^c = x^{bc} = x^{bd} = (x^{b})^d = \varphi(x^b)$.
				\par The existence of such a $c$ follows by a number theoretic argument, but I am not sure how.
			\end{proof}
	\end{enumerate}
\end{problem}
\begin{problem}{2}
	\begin{enumerate}[label=(\alph*)]
		\item (Working with the set $0, \dots 4$ so that we can do modular arithmetic)
			\par $G$ is the group generated by the $5$-cycle $\left( 0 \;1 \; 2 \; 3 \; 4 \right)$, show that the action of $G$ on the set $ S = \left\{ 0, 1,2,3,4\right\}$ is primitive but not doubly transitive.
			\begin{proof}
				We recall that a \textit{primitive} group action of $G$ on $S$ is a transitive action such that there is no nontrivial subgroup (more than $1$ element) $X$ of $S$, such that $X $ is disjoint from every other set in its orbit under $G$.
				\par Let $g$ be the generating cycle of $G$. It is immediate that $G$ is transitive, as it takes every element of $S$ to every other element, cyclically: $1$ to $2$ to $3$, and so on - for any two elements $x$, where $1 \leq x,y \leq 5$, then $g^{n}(x) = y$, where $n = (y - x \text{ mod } 5)$.
				\par Now, we want to show that $G$ is primitive. Let $X \subset S$ be a block of $S$. We want to show that $X$ is either a singleton or the whole set. Assume that $\left \lvert {  X } \right \lvert > 1$, and let $x, y$ be two elements of $X$. Let $n = y - x$ mod $5$, so that $y = x + n$, and $n \neq 0$ mod $5$.
				\par Then $g^n(x) = y$, and so $X$ and $g^n \cdot X$ are not disjoint. Since $X$ is a block, $g^n \cdot X$ must be equal to $X$, so we have $g^n(y) \in X$ as well. Since \begin{align*}g^n(y) &= y + n \text{ mod } 5\\
				&=  x + 2n \text{ mod }5,\end{align*}
				we must have $x + 2n \in X$. Since $5$ is prime, $2n \neq 0$ mod $5$, and so $x + 2n \neq x$. So, $\left\{ x, x + n, x+ 2n \right\} \subset X$. 
				\par Again, $g^n(x + 2n) = x + 3n \in X$, and $x + 4n \in X$ as well. So the set $\left\{ x, x + n, x + 2n, x+ 3n, x + 4n \right\} \subset X$. Since $n \neq 0$ mod $5$, these are five distinct elements, and so $\lvert X \rvert \geq 5$. Therefore $X$ is the whole set $S$, and there are no nontrivial blocks of $S$ under the action of $G$. So, the action is primitive.
			\end{proof}
		\item Give an example of an imprimitive group action.
			\begin{proof}
				Any transitive action which is not primitive is called imprimitive. For instance, the action of $\langle (1 \; 2 \; 3 \; 4 \; 5 \; 6)\rangle $ on the set $\left\{ 1, 2, 3, 4, 5, 6 \right\}$ is imprimitive. It admits the block $\left\{ 0, 2, 4 \right\}$: the only sets in the orbit of $\left\{ 0, 2, 4 \right\}$ are $\left\{ 1, 3, 5 \right\}$ and the set itself, which are disjoint.
			\end{proof}
	\end{enumerate}
\end{problem}
\begin{problem}{3}
	\begin{enumerate}[label=(\alph*)]
		\item In the group $G = \langle x\rangle \times S_3$, with $\langle x\rangle$ a cyclic group of order $4$  and $S_3$ the symmetric group on $3$ letters, list the elements of the subgroup generated by the element $(a^2, (123))$. Call this subgroup $C$. Show that $C$ is normal in $G$.
			\begin{proof}
				The elements of this subgroup are:
				\begin{align*}
					&(a^2, (123)) &(e, (132))\\
					&(a^2, e) &(e, (123))\\
					&(a^2, (132)) & (e,e )
				\end{align*}
				So we see that $\left \lvert { C } \right \lvert  = 6$. Since, for general product groups,  $\langle (x, y)\rangle \subset \langle x\rangle \times \langle y\rangle$,  we know that $C \subset \langle a^2\rangle \times \langle (123)\rangle$. The group $\langle a^2\rangle \times \langle (123)\rangle $ also has order $6$, so it must be true that $C = \langle a\rangle \times \langle (123)\rangle$.
				\par It is also true that, for a general product group $G \times H$, if $A \trianglelefteq G$ and $B \trianglelefteq H$, then $A \times B \trianglelefteq G \times H$. Since $\langle a^2\rangle$ has index $2$ in $\langle a\rangle$, and $\langle (123)\rangle$ also has index $2$ in $S_3$, they are both normal subgroups in their respective groups. Therefore, their direct product $C$ is also normal in $\langle a\rangle \times S_3$.
			\end{proof}
		\item List the elements of the quotient group $G/ C$. Is $G/C$ a cyclic group? Explain.
			\begin{proof}
			The elements of $G / C$ are as follows:
			\begin{align*}
				&(e, e) \cdot C & (a, e) \cdot C\\
				&(e, (12)) \cdot C &(a, (12)) \cdot C
			\end{align*}
			This is the Klein $4$-group $\Z_2 \times \Z_2$, which we can see by the fact that every element has order $2$. That these elements are distinct is not as easy to see, but we can see by inspection that they are pairwise unequal: for instance, $(e,e) + C \neq (a, e) + C$, because there is no element $(x,y)$ of $C$ such that $e\cdot x = a$.
		\end{proof}	
	\item Find the centralizer of $C$ in $G$.
		\begin{proof}
			The centralizer of any subgroup of $G$ is contained in the center of $G$. Since $G$ is a direct product, its center is the direct product of the centers of its factors, which in this case is $\langle a\rangle \times \left\{ e \right\} \simeq \Z_4$. So, we know that $\langle a\rangle \times \left\{ e \right\} \subset C_G(C)$. 
			\par Further, we can see that any element of $G \backslash \langle a\rangle \times e$ cannot be in the centralizer of $C$: let $(x,y)$ be an element of $G \backslash \langle a \rangle \times e$, so that $y \neq e$. Then $(x,y) \cdot (e, (123)) = (e, (123)) \cdot (x,y)$, implying that $(123)y = y(123)$. The only elements of $S_3$ which commute with $(123)$ are $(123)$ itself and $(123)^{-1} = (132)$, so we see that $C_G(C) \subset \langle a\rangle \times \langle (123)\rangle$. 
			\par In fact, every element of this group commutes with every element of $C$: let $x = (a^n, (123)^m) \in \langle a\rangle \times \langle (123)\rangle$, and $ y = (a^{2k}, (123)^j) \in C$. Then 
			\[xy = (a^{n + 2k}, (123)^{m+j}) = yx.\]
			So, we see that $C_G(C) = \langle a \rangle \times \langle (123)\rangle$.
		\end{proof}
	\end{enumerate}
\end{problem}
\begin{problem}{4}
	Up to isomorphism, list all abelian groups of order $72 = 2^3 \times 3^2$.
	\begin{proof}
		The following are all abelian groups of order $72$:
		\begin{align*}
			&\Z_8 \times \Z_9\\ &\Z_8 \times \Z_3 \times \Z_3\\
		&\Z_4 \times \Z_2 \times \Z_9 \\&\Z_4 \times \Z_2 \times \Z_3 \times \Z_3\\
		&\Z_2 \times \Z_2 \times \Z_2 \times \Z_9 \\&\Z_2 \times \Z_2 \times \Z_2 \times \Z_3 \times \Z_3
		\end{align*}
	\end{proof}
\end{problem}
\begin{problem}{5}
	If $p^n$ divides $\left \lvert { G } \right \lvert $ and $H \leq G$ has order $p^m$, with $m \leq n$, then the number of subgroups of $G$ of order $p^n$ that contain $H$ is equal to $1$ modulo $p$.
	\begin{proof}
		\textit{Incomplete}
	\end{proof}
\end{problem} 
\begin{problem}{6}
	Find the full automorphism group of the quaternion group $Q_8$.	
	\begin{proof}
		The quaternion group has three order-$4$ subgroups, $\langle i\rangle$, $\langle j\rangle$, and $\langle k\rangle$. Any automorphism of $Q_8$ must map each of these subgroups to another, giving a homomorphism $\text{Aut}(Q_8) \to S_3$. This homomorphism is surjective, since for each permutation of $i$, $j$, and $k$, there is a corresponding automorphism of $Q_8$: for example, for the permutation $(ij)$ there is the automorphism $i \mapsto j$, $j \mapsto i$, and $k \mapsto -k$, and for the permutation $(jk)$ there is the automorphism $i \mapso -i$, $j \mapsto k$, and $k \mapsto j$. These two permutations generate $S_3$, so the map is surjective.
	\par The kernel of this homomorphism is the set of all automorphisms which preserve the groups $\langle i\rangle$, $\langle j\rangle$, and $\langle k\rangle$. Because these are cyclic groups, an element of the kernel is any automorphism of $Q_8$ which maps $i$ into $\left\{ -i, i \right\}$, $j$ into $\left\{ -j, j \right\}$, and $k$ into $\left\{ -k, k \right\}$. Since $ij=k$, the value at  $i$ and $j$ determines the morphism, and any of the four images of $\left\{ i, j \right\}$ gives an automorphism of $Q_8$:
	\begin{align*}
		&(i, j) & (i, -j)\\
		&(-i, j) & (-i, -j)
	\end{align*}
	This group of $4$ automorphisms is of course the Klein $4$-group. 
	\end{proof}
\end{problem}
\begin{problem}{7}
	Prove that $GL(3,2)$ is a simple group, perhaps as follows:
	\begin{enumerate}[label=(\alph*)]
		\item Show that $\left \lvert { G } \right \lvert  = 168$
			\begin{proof}
				For any $3 \times 3$ matrix over $F_2$, there are $2^3 - 1 = 7$ choices for the first column; everything but the zero vector. For the second column, we then have the choice of everything but the zero vector, and all multiples of the first column. In $F_2$, multiples are trivial, so this gives only one less choice: $2^3 - 2 = 6$. And, for the third column, we have the choice of everything but any linear combnination of the first two columns - there are $4$ such linear combinations, including the zero vector. This gives $2^3 - 4 = 4$ choices. Any sequence of these choices gives a different element of $GL(3,2)$, so we have $7 \cdot 6 \cdot 4 = 168$ elements of $GL(3,2)$.
			\end{proof}
		\item Show that $V$, the $3$-dim vector space over $F_2$, has $7$ $1$-dimensional subspaces, this establishes a natural injection of $G$ into $S_7$.
			\begin{proof}
				Any one-dimensional linear subpsace over $V$ is the set of all $F_2$-multiples of a given nonzero vector. However, $F_2$-multiples are trivial, so the set of linear subspaces is in bijection with the set of nonzero elements; there are $2^3 - 1 = 7$ such elements of $V$. Since an element of $GL(3,2)$ is invertible, it is a bijection on the set of one-dimensional subspaces of $V$, and so induces a permutation on the set with $7$ elements.
			\end{proof}
		\item Use Jordan canonical forms to establish that $G$ has six conjugacy classes with $1$, $21$, $42$, $56$, $24$, $24$ elements respectively. Deduce that $G$ is simple.
			\begin{proof}
				\textit{incomplete}
			\end{proof}
	\end{enumerate}
\end{problem}
\begin{problem}{8}
	Give an example of groups $G$ and $H$, perhaps of order $8$, such that $G'$ is isomorphic to $H'$ and $G/G'$ is isomorphic to $H/H'$, but $G$ is not isomorphic to $G$. Here $K'$ denotes the commutator subgroup of $K$
	\begin{proof}
		Let $G= Q_8$ and $H = D_4$. The commutator subgroup of $G$ is generated by the elements $[x,y]$, for $x, y \in \left\{ \pm 1, \pm i, \pm j, \pmk \right\}$. The order of $x$ and $y$ does not matter for the genration of $G'$, and the commutator of $1$ or $-1$ with anything is $1$, since they are in the center. Finally, $[-x, y] = -[x,y]$, again since $-1$ is in the center and can be factored out. So, the three elements $[i,j]$, $[j, k]$, and $[k,i]$ generate $G'$. Each of these elements is equal to $-1$:
		\begin{align*}
			[i,j] &= iji^{-1}j^{-1} = ij(-i)(-j) = (ij)^{2} = k^2 = -1\\
			[j, k] &= jkj^{-1}k^{-1} = jk(-j)(-k) = (jk)^{2} = i^2 = -1\\
		[k, i] &= kik^{-1}i^{-1} = ki(-k)(-i) = (ki)^{-1} = j^2 = -1
		\end{align*}
		So, the commutator subgroup is generated by $-1$, and is isomorphic to $\Z_2$. The quotient $G/G'$ is the set $\left\{ [1], [i], [j], [k] \right\}$, which has $3$ degree two subgroups and is therefore equal to $\Z_2 \times \Z_2$.
		\par The commutator subgroup of $H = \langle r, s \mid r^4 = s^2 = e, rs = sr^3\rangle$ must be the smallest normal subgroup $H' \subset H$ such that the quotient $H/H'$ is abelian. Since the quotient $H/\left\{ e \right\} = H$ is nonabelian, the commutator subgroup must be nontrivial. But the group $\left\{ e, r^2 \right\}$ is normal, and $H / \left\{ e, r^2 \right\}$ is the set $\left\{ [1], [r], [s], [rs] \right\}$, which is isomorphic to $\Z_2 \times \Z_2$, and is abelian. So, the commutator subgroup of $H$ must be $\left\{ e, r^2 \right\} \simeq \Z_2$.
		\par So, we see that the commutator subgroups $G'$ and $H'$ are isomorphic to $\Z_2$, and $G/G'$ and $H/H'$ are isomorphic to $Z_2 \times \Z_2$, yet the two groups $G$ and $H$ are nonisomorphic.
	\end{proof}
\end{problem}
\begin{problem}{9}
	If a finite group $G$ has at most $n$ elements of order dividing $n$, for each $n\in \N$, then $G$ is cyclic. \textit{Hint}: Look at Sylow-subgroups first.
	\begin{proof}
		Let $\left \lvert {  G } \right \lvert  = p_1^{n_1}p_2^{n_2} \cdots p_i^{n_i}$. Then for each $1 \leq j \leq i$ it has a Sylow subgroup of order $p_j^{n_j}$. Since there are at most $p_j^{n_j}$ elements of $G$ whose order divide $p_j^{n_j}$, and every element of a Sylow $p_j$ group must divide $p_j^{n_j}$, the group is necessarily unique, and therefore normal. Since each Sylow group in $G$ is normal, it must be the direct product of its Sylow groups.
		\par Now, we show that for $\left \lvert { G } \right \lvert = p^m$, $G$ must be cyclic. Let $G$ be a $p$-group of order $p^m$. Then $G$ 
	\end{proof}
\end{problem}
\begin{problem}{10}
	Explicitly construct two normal field extensions of $\mathbb Q$, with Galois groups $C_2 \times C_2$ and $C_4$ respectively. Establish the Galois correspondence in each case.
	\begin{proof}
		\textit{incomplete}
	\end{proof}
\end{problem}
\end{section}
\end{document}
