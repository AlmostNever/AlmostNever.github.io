% --------------------------------------------------------------
% Andrew Tindall
% --------------------------------------------------------------
 
\documentclass[12pt]{article}
 
\usepackage[margin=1in]{geometry} 
\usepackage{amsmath,amsthm,amssymb,enumitem,hyperref}

\newcommand{\N}{\mathbb{N}}
\newcommand{\Q}{\mathbb{Q}}
\newcommand{\Z}{\mathbb{Z}}
\newcommand{\C}{\mathbb{C}}
\newcommand{\R}{\mathbb{R}}
\newcommand{\mc}[1]{\mathcal{#1}}
\newcommand{\e}{\varepsilon}
\newcommand{\bs}{\backslash}
\newcommand{\PGL}{\text{PGL}}
\newcommand{\Sp}{\text{Sp}}
\newcommand{\tr}{\text{tr}}
\newcommand{\Lie}{\text{Lie}}
\newcommand{\rec}[1]{\frac{1}{#1}}
\newcommand{\toinf}{\rightarrow \infty}


\theoremstyle{definition}
\newtheorem{proofpart}{Part}
\newtheorem{theorem}{Theorem}
\makeatletter
\@addtoreset{proofpart}{theorem}
\makeatother


\newenvironment{problem}[2][Problem]{\begin{trivlist}
\item[\hskip \labelsep {\bfseries #1}\hskip \labelsep {\bfseries #2.}]}{\end{trivlist}}
 
\begin{document}
 
%\renewcommand{\qedsymbol}{\filledbox}
 
\title{Homework 3}
\author{Andrew Tindall\\
Algebra 1}
 
\maketitle
\begin{problem}{1}
    Let $G$ be a group with $\left \lvert { G } \right \lvert = mn$, for coprime $m$ and $n$. For any $g\in G$, there exist unique $x, y \in G$ such that $xy = g = yx$ and $x^m = 1 = y^n$.
    \begin{proof}
        ( proof found on stackexchange) Fix some $g \in G$. Since $m$ and $n$ are coprime, there must exist some positive integers $a$ and $b$ such that $am + bn = 1$. Let $x = g^{bn}$ and $y = g^{am}$. It is immediate that $xy = yx = g^{(am + bn)}$, and by our choice of $a$ and $b$, we see that $g^{am + bn} = g^1 = g$.
        \par Finally, since the order of every element of $G$ divides $mn$, we see that $x^m = g^{b(mn)} = e$ and that $y^n = g^{a(mn)} = e$.
    \end{proof}
\end{problem}
\begin{problem}{2}Let $G$ be solvable. Can $G$ have a subgroup that is nonabelian simple? Can a homomorphic image of $G$ be nonabelian simple?
    \begin{proof}
        No. This is a direct consequence of the fact that every subgroup of a solvable group is solvable, and that no nonabelian simple group may be solvable, because it has no nontrivail subnormal series and is itself not abelian.
        \par We can show quickly from some manipulation of group isomorphism theories that every subgroup of a solvable gorup is solvable. Let $G$ be solvable, with some subnormal series 
        \[
           1 \simeq G_0 \trianglelefteq G_1 \trianglelefteq \cdots \trianglelefteq G_n \trianglelefteq G,        
        \]
        where each quotient $G_i / G_{i-1}$ is abelian. We show that the series
        \[
         (G_0 \cap H) \leq (G_1 \cap H) \leq \cdots \leq (G_n \cap H) \leq H
        \]
        Is in fact subnormal, and that each quotient in the series is abelian.
        \par First, we see that $(G_{i -1} \cap H)$ is normal in $(G_i \cap H)$.
    \end{proof}
\end{problem}
\begin{problem}{3}
	Verify the Bruhat decomposition for $G = \text{GL}(2,2)$; what is $G$ abstractly? Check that $G$ is generated by diagonal matrices and transvections. 
	\begin{proof}
	<++>	
	\end{proof}
\end{problem}
\begin{problem}{4}
	$H$ normal in $G$, $K$ normal in $G$, $G = HK$ imply $\frac{ G}{H \cap K} = \frac{H}{H \cap K} \times \frac{K}{H \cap K}$.
	\begin{proof}
		<++>
	\end{proof}
\end{problem}
\begin{problem}{5}
	If $G$ is the semidirect product of $N$ by $H$ and $N \leq K \leq G$, then $K$ is the semidirect product of $N$ by $H \cap K$.
\end{problem}
\begin{problem}{6}
	Show that $Q_8$ cannot be expressed as a semidirect product.
	\begin{proof}
	 If $Q_8$ were a semidirect product of two nontrivial subgroups $N$ and $H$, then the identity $\lvert N \rvert \cdot \lvert H \rvert = \lvert Q_8 \rvert = 8$ shows that one subgroup would have to have order $4$ and the other order $2$.
	 \par The definition of a semidirect product also requires that $N \cap H = \{e\}$. However, there is a unique element of order $2$ in $Q_8$, which is $-1$, and we will see that both $N$ and $H$ must have an element of order $2$.
	 \par The group of order $2$ must be isomorphic to $\Z_2$, so its nonidentity element must have order $2$. On the other hand, the group of order $4$ must be isomorphic to either $\Z_4$, which has an  element $2$ of order $2$, or to $\Z_2 \times \Z_2$, which has $3$ elements of order $2$, contradicting the presence of only one element of order $2$ in $Q_8$. Therefore, the group of order $4$ would have to be $\Z_4$.
	 \par So, the two groups $N$ and $H$ must both contain the single element in $Q_8$ of order $2$, contradicting the requirement that they have nontrivial intersection. Therefore, $Q_8$ cannot be constructed as a semidirect product.
	\end{proof}
\end{problem}
\end{document}
